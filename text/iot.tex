\chapter{Internet věcí}

Cílem Internetu věcí (IoT - Internet of Things) je propojení zařízení, systémů a služeb za účelem poskytnutí více dat, která mohou být převedena na informace a informace potom na znalosti, které mohou být následně aplikovány. Princip IoT tedy je: na začátku je sběr dat, ty jsou následně uložena a analyzována a poté dojde ke sdílení výsledků. V rámci IoT se vytvořily dva hlavní směry, průmyslový internet věcí (iIoT - Industry IoT) a spotřebitelský internet věcí (cIoT - Customer IoT) \cite{iot_svet_hardware_internet_veci, uvod_pohanka_internet_veci}. Rozdíly obou směrů jsou shrnuty v Tab. \ref{TableIOT}.

\section{Spotřebitelský Internet věcí}
Spotřebitelský Internet věcí se zaměřuje na spotřebitelská zařízení, IT a telekomunikační zařízení a další. Jsou zde využívána zařízení zjednodušující každodenní život pomocí automatizace v domácnosti, chytrých zařízení nebo pomocí nositelné elektroniky. Hlavní výhodou je zvýšení uživatelského zážitku (QoE - Quality of Experience).

\section{Průmyslový Internet věcí}
Průmyslový Internet věci vychází z M2M (Machine to Machine) a rozšiřuje komuikaci o možnost uložení, analýzy a zobrazení dat. Jedná se o IoT zařízení a systémy, které jsou používány v průmyslových odvětvích, jako jsou průmyslová automatizace, energetický průmysl a zdravotnictví. Hlavním zaměřením je efektivnější využívání zdrojů, snížení provozních nákladů, zvýšení efektivity či bezpečnosti. V praxi může sloužit například pro zajištění bezpečnosti pracovníků či automatizaci údržby. 


\begin{table}[!ht]
\caption{Porovnání průmyslového a spotřebitelského IoT \cite{iot_svet_hardware_internet_veci, uvod_pohanka_internet_veci}}
\label{TableIOT}
\begin{center}
\small
\begin{tabular}{|c|c|c|}
\hline
 & \textbf{Průmyslový IoT} & \textbf{Spotřebitelský IoT} \\ \hline
\textbf{Zaměření} & Průmysl & Spotřebitel \\ \hline
\textbf{Zařízení} & \begin{tabular}[c]{@{}c@{}}Stroje, zařízení\\  a průmyslová automatizace.\end{tabular} & \begin{tabular}[c]{@{}c@{}}Chytré zařízení\\ a nositelná elektronika.\end{tabular} \\ \hline
\textbf{Důležitost} & \begin{tabular}[c]{@{}c@{}}Jedná se o životně\\ důležité systémy.\end{tabular} & \begin{tabular}[c]{@{}c@{}}Nejedná se o životně \\ důležité systémy.\end{tabular} \\ \hline
\textbf{Využití} & \begin{tabular}[c]{@{}c@{}}Lepší využívání zdrojů, \\  snížení provozních nákladů, \\ zvýšení efektivity či bezpečnosti.\end{tabular} & \begin{tabular}[c]{@{}c@{}}Zvýšení uživatelského\\ zážitku.\end{tabular} \\ \hline
\end{tabular}
  \end{center}
\end{table}









Jelikož téma práce zahrnuje práci s modulem pro protokol Wireless M-Bus, sloužící především k elektroenergetice, tato práce se svým zaměřením řadí do průmyslového Internetu věcí.

\colorbox[rgb]{1,0,0}{Sem jeste prijde nejaky text na IIoT nebo Prumysl4 (22/3)}

