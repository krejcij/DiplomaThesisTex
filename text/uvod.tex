\chapter*{Úvod}
\phantomsection
\addcontentsline{toc}{chapter}{Úvod}

Fenoménem dneška je~propojování Internetu věcí (IoT - Internet of~Things), služeb (IoS - Internet of~Services) a~lidí (IoP - Internet of~People) a~s~ním související vývoj komunikací stroj-stroj (M2M - Machine to~Machine), člověk-stroj (H2M - Human to~Machine) nebo člověk-člověk (H2H - Human to~Human). Internet věcí, služeb a~lidí se rozšiřuje závratným tempem a~proniká tak do~odvětví, ve~kterých se~rostoucím tempem využívají komunikační nízkovýkonové (embedded) zařízení a~roste potřeba rozšíření těchto zařízení o~nové komunikační protokoly a~technologie. Vzniknou sítě založené na~propojených zařízeních, které budou schopny samostatné výměny informací, vyvolání potřebných akcí v~reakci na~momentální podmínky a vzájemné nezávislé kontroly. Senzory, přístroje a~IT systémy budou vzájemně propojeny a~budou na~sebe pomocí standardních komunikačních protokolů vzájemně reagovat a~analyzovat data, aby mohly předvídat případné chyby či~poruchy, konfigurovat samy sebe a~v~reálném čase se přizpůsobovat změněným 
podmínkám \cite{uvod_prumysl_4_pdf,uvod_prumysl_4_web}.

Tato práce vychází z~požadavku na~implementaci Wireless M-Bus protokolu do~produktu UniPi NEURON. K~tomuto účelu bylo zvoleno nízkovýkonové (embedded) zařízení RaspberryPi a~jeho rozšiřující modul UniPi. Pro M2M komunikaci byl zvolen protokol Wireless M-Bus, jelikož je jedním z~nejrozšířenějších a~navíc je založen na~protokolu M-Bus, který je osvědčený a~velmi rozšířený (měření~a regulace topných systémů, plynu, odběru vody a~elektrické energie). V teoretické části práce jsou popsány jednotlivé rodiny jednodeskových počítačů a~jejich vlastnosti, popis rozšiřujících desek UniPi~a samotného komunikačního modulu pro~Wireless M-Bus a~popis komunikačního protokolu Wireless M-Bus. Teoretickou část uzavírá přehled vyčítaných měřících zařízení protokolu Wireless M-Bus. 

Praktická část se~zaměřuje na~implementaci Wireless M-Bus protokolu v~zařízení RaspberryPi pomocí rozšiřujícího modulu UniPi a~komunikačního modulu Wireless M-Bus. Tato implementace vyčítání dat ze~vzdálených zařízení je realizována v~jazyku Python a~následně jsou získaná data vizualizována pomocí Google Chart API~\cite{uvod_google_charts_api}.


 