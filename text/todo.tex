\chapter{ToDo Poznámky}

\colorbox[rgb]{0,1,0}{neni realna kapitola. tady mam nejake poznamky k dodelani, abych to nemel po papircich.}

\begin{itemize}
	\item doplnit kecy o AES sifrovano (norma FIPS 197), popsat jednozlive mody, napsat logiku desifrovani, popsat tvorbu inicializacniho vektoru
	\item v kapitole 7.2.5 popsat jak se to prakticky obecne desiforvalo, dat ukazku desifrovani AES128 v modu CBC do prace [CLANEK!]
	\item dopsat druhou aplikaci pro vizualizaci, popsat nastaveni a instalaci veci k google sheetum
	\item sepsat proc se tu bavime pouze o ramci typu A (jedeme pouze v modu T zatim) a pripadne zminit nebo sepsat i ramec B
	\item do popisu vrstev doplnit prehledy moznosti dle dane specifikace
	\item doplnit vzorec pro vypocet RSSI a VENDOR\_ID
	\item popsat ulozeni dat LSB vs MSB
	\item Pikkerton+ZPA+Kamstrup presazet jejich teoreticke kapitoly a datasheery
	\item prekreslit vsechny potrebne obrazky, vcetne toho vyvojoveho diagramu. 
	\item prekreslit vsechny ukazky telegramu - narvat je to vektoru nebo tabulky.
	\item presazet vsechny prikazy pro linux a zachycene telegramy do cislovanych rovnic [CLANEK!]
	\item doplnit neco malo k uvodni strance [CLANEK!]
	\item projit praci a sjednotit reference Obr. vs obrázek
	\item projit praci a sjednotit sniffer-skener, MUC->koncentrator, Meter->Meric
	\item korektura tentokrat zavcas!
	\item ocitovat zdroje:
		\begin{itemize}
			\item FIPS197
			\item OMS 1-4 vrstvy, telegram example, primary comunication
			\item Datashety: Bonega, Kamstup, ZPA
		\end{itemize}
\end{itemize}
