\chapter{ToDo Poznámky}

\colorbox[rgb]{0,1,0}{neni realna kapitola. tady mam nejake poznamky k dodelani, abych to nemel po papircich.}

\begin{itemize}
	\item doplnit kecy o AES sifrovano (norma FIPS 197), popsat jednozlive mody, napsat logiku desifrovani, popsat tvorbu inicializacniho vektoru
	\item v kapitole 7.2.5 popsat jak se to prakticky obecne desiforvalo, dat ukazku desifrovani AES128 v modu CBC do prace
	\item okomentovat dukladne navrhnute schema aplikace, nezapomenout vysvetlit logiku:zasifruj desifrovane, zkontroluj, desifruj znova
	\item sepsat proc se tu bavime pouze o ramci typu A (jedeme pouze v modu T zatim) a pripadne zminit nebo sepsat i ramec B
	\item do popisu vstev wm-busu zminit i obalku od IQRF modulu a UART sbernice...co a proc to tam je
	\item do popisu vrstev doplnit prehledy moznosti dle dane specifikace
	\item doplnit vzorec pro vypocet RSSI a VENDOR\_ID
	\item popsat ulozeni dat LSB vs MSB
	\item provazat to cele se specifikaci OSMS 3 a popsat proc se o tom vubec bavime
	\item dopsat nekam ze iqrf umi aes pouze jako jednosmery meter. muc implementace stoji za hovno, snifferova taky, ale ta jde castecne obejit
	\item Pikkerton merak je predsazeny, vlozit az bude jasne co s tim, respektive jestli bude otestovano
	\item ZPA merak netusim, co s tim? zeptat se Maska na schuzce
	\item Kamstup merak je k dispozici datasheet, cekam na RCInvest jestli me to umozni desifrovat a validovat
	\item dopsat ty veci okolo ukladani do DB - bude se odvijet dle nasledne vizualizace
	\item doresit vycitani casu u Bonegy. aktualne sem se v tom zamotal.
	\item narvat ty zdrojaky do GITu, uz se v tom ztracim :)
	\item prekreslit vsechny potrebne obrazky, vcetne toho vyvojoveho diagramu. 
	\item prekreslit vsechny ukazky telegramu - narvat je to vektoru nebo tabulky.
	\item projit praci a sjednotit reference Obr. vs obrázek
	\item projit praci a sjednotit sniffer-skener, MUC->koncentrator, Meter->Meric
	\item korektura tentokrat zavcas!
	\item ocitovat zdroje:
		\begin{itemize}
			\item FIPS197
			\item OMS 1-4 vrstvy, telegram example, primary comunication
			\item Datashety: Bonega, Kamstup, ZPA?
		\end{itemize}
\end{itemize}
