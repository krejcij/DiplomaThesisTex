\chapter{Závěr}

V~této diplomové práci byla popsána problematika M2M (Machine to~Machine) komunikace pomocí protokolu Wireless M-bus a~její implementace do~produktu UniPi NEURON.


V~první části práce byla popsána M2M komunikace z pohledu spotřebitelského a průmyslového Internetu věcí.

Druhá část se~zabývá embedded zařízeními pro~IoT (Internet of Things), přináší přehled nejznámějších z~nich, popisuje jejich možnosti, uvádí možnosti připojení senzorů a~zmiňuje nedostatky zařízení. Zařízení RaspberryPi je~následně použité k~samotné implementaci v praktické části. Jsou zde popsány předchozí verze, důvod výběru konkrétního modelu, design i~kroky potřebné k~implementaci.

Třetí část obsahuje popis rozšiřující desky UniPi a~zařízení UniPi NEURON. Popisuje blíže parametry obou zařízení, možnosti jejich konektivity a~softwarového vybavení. Zařízení bylo vyvinuto primárně jako rozhraní pro~příjem vstupních signálů, jejich vyhodnocení a~realizaci výstupní reakce na~základě naprogramovaných algoritmů. Je~vhodné pro~monitorování, sběr a~ukládání dat na~vzdálený server, nebo~jako výkonná a~plně vybavená brána pro~ostatní zařízení.

Čtvtrá část se~zabývá Wireless M-Bus modulem TR-72D-WMB výrobce IQRF, komunikující přes~sběrnici UART, a~popisuje strukturu příkazů a~formát dat pro~komunikaci s~tímto modulem.

Pátá část se~zaměřila na~protokol Wireless M-Bus, konkrétně na~princip komunikace, režimy přenosu a~jednotlivé vrstvy. 
Díky nutnosti znalosti fyzické a~linkové vrstvy pro~pozdější analýzu zachytávaných dat byly tyto vrstvy rozebrány podrobněji. 

V šesté části byly popsány vyčítaná zařízení a~struktura dat jejich telegramů.

Závěrečná (sedmá) část obsahuje návrh a~samotnou implementaci vzorové aplikace pro~vyčítání dat. Jsou popsány jednotlivé kroky nutné ke~zprovoznění komunikace mezi RaspberryPi a~vyčítaným senzorem, provedeno zachycení vzorového telegramu, jeho analýza a~následné předání zvolených informací. Z~výstupu aplikace je patrné, že~pakety obsahují příslušná data, komunikace mezi modulem a~zařízením funguje, data ze~senzoru se~přenášejí, následně vyčítají a~přehledně zobrazují v~implementované vizualiaci pomocí Google Charts API.

Na~zakládě této realizace je~vytvořená aplikace analyzovat jen předem známá zařízení. Pokud by~řešení mělo být plně využitelné pro Internet věcí, bylo by nutné implementovat velké množství parsovacích schémat, nebo vytvořit obecný algoritmus pro~parsování datových jednotek všech výrobců dodržující platnou specifikaci.


