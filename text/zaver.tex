\chapter{Závěr}

V této diplomové práci byla popsána problematika M2M (Machine to Machine) komunikace pomocí protokolu Wireless M-bus a její implementace do produktu UniPi NEURON.


V první části práce byla popsána M2M komunikace z pohledu spotřebitelského a průmyslového Internetu věcí.

Druhá část se zabývá embedded zařízeními pro IoT (Internet of Things), přináší přehled nejznámějších z nich, popisuje jejich možnosti, uvádí možnosti připojení senzorů a zmiňuje nedostatky zařízení. Zařízení RaspberryPi je následně použité k samotné implementaci v praktické části. Jsou zde popsány předchozí verze, důvod výběru konkrétního modelu, design i kroky potřebné k implementaci.

Třetí část obsahuje popis rozšiřující desky UniPi a zařízení UniPi NEURON. Popisuje blíže parametry obou zařízení, možnosti jejich konektivity a softwarového vybavení. Zařízení bylo vyvinuto primárně jako rozhraní pro příjem vstupních signálů, jejich vyhodnocení a realizaci výstupní reakce na základě naprogramovaných algoritmů. Je vhodné pro monitorování, sběr a ukládání dat na vzdálený server, nebo jako výkonná a plně vybavená brána pro ostatní zařízení.

Čtvtrá část se zabývá Wireless M-Bus modulem TR-72D-WMB výrobce IQRF, komunikující přes sběrnici UART, a popisuje strukturu příkazů a formát dat pro komunikaci s tímto modulem.

Pátá část se zaměřila na protokol Wireless M-Bus, konkrétně na princip komunikace, režimy přenosu a jednotlivé vrstvy. 
Díky nutnosti znalosti fyzické a linkové vrstvy pro pozdější analýzu zachytávaných dat byly tyto vrstvy rozebrány podrobněji. 

\colorbox[rgb]{1,0,0}{AKTUALIZOVAT}

V šesté části bylo popsáno čidlo WEPTECH a struktura dat jeho telegramu.

Závěrečná (sedmá) část obsahuje návrh a samotnou implementaci vzorové aplikace pro vyčítání dat. Jsou popsány jednotlivé kroky nutné ke zprovoznění komunikace mezi RaspberryPi a vyčítaným senzorem, provedeno zachycení vzorového telegramu, jeho analýza a následné předání zvolených informací. Z výstupu aplikace (viditelném v konzoli) je patrné, že pakety obsahují příslušná data, komunikace mezi modulem a zařízením funguje, data ze senzoru se přenášejí, následně vyčítají a zobrazují.


Nakonec bych rád zmínil další možnosti vývoje aplikace. V aktuální verzi je aplikace schopná zachytávat nešifrovaný přenos dat od Wireless M-Bus zařízení výrobce WEPTECH. V navazující diplomové práci proto bude přistoupeno k rozšíření podpory senzorů od více výrobců, AES kódování přenášených dat či vizualizaci zachytávaných dat.



\colorbox[rgb]{1,0,0}{VNORIT DO ZAVERU + ta cestin!}
V teto zaverecne casti se budeme zabyvat aspekty, kterym jsme celili v prubehu vyvoje naseho reseni. 
Behem implementace aplikace jsme museli vyresit nekolik problemu:
\begin{itemize}
	\item Implementace byla provedena na embeded zarizeni RaspberryPi verze 3, bohuzel pristup k seriovemu portu tohoto zarizeni je odlisny od RaspberryPi predchozich verzi. Byly nutne zasahy do konfigurace na urovni zavadeni operacniho systemu.
	\item Bylo nutne implementovat UART komunikaci s bezdratovym modulem IRQF pomoci daneho protokolu, ktery je stale ve fazi vyvoje a jehoz dokumentace neni nyni kompletni.
	\item V pripade odposlechu zasifrovanych dat bylo nutne vyresit jejich zasifrovani AES klicem bezdratoveho modulu IRQF zpet do desifrovatelneho prenosoveho stavu a opetovne desifrovani prislusnym klicem.
	\item Pro zvolena merici zarizeni bylo potreba provest analyzu zachycenych telegramu a urcit schema vyparsovani zachycenych dat. To stezuje situaci pro plosene nasazeni daneho scenare.
	\item Pri vizualizaci dat pomoci Google bylo nezbytne implementovat autorziacni a komunikacni knihovny tretich stran.
\end{itemize}
Reseni techto dilcich problemu umoznilo postupne dosahnout techto cilu:
\begin{itemize}
	\item V pripade vyrobcu WepTech, ZPA, Kamstrup a Bonega jsme schopni vycitat jakekoliv zarizeni, vcetne zarizeni podporujici pouze sifrovany prenos.	
	\item Nejvyznamejsim dosazenym cilem bylo dosazeni analyzy, ulozeni a vizualizace prijatych dat.
\end{itemize}

Na zaklade nasich zjisteni jsme schopni ve skutecnosti analyzovat jen predem znama zarizeni. Pokud bychom chteli nase reseni plne vyuzit pro internet veci, bylo by nutne implementovat velke mnozstvi parsovacich schemat. Z toho vyplyva, ze dane reseni pro tento okamzik nema potrebnou funkcnost internetu veci v realnem svete.


