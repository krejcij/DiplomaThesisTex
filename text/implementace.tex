\chapter{Návrh implementace}
Jak již bylo zmíněno na začátku práce, samotná implementace je rozdělena do dvou částí: \colorbox[rgb]{0,1,0}{nezapomenout zpetne upravit bod dva}
\begin{enumerate}
	\item Komunikace RaspberryPi přes rozšiřující desku UniPi s bezdrátovým modulem a pomocí něj s poskytnutými WM-Bus zařízeními.
	\item Implementace této komunikace do nezávislé knihovny funkcí jako rozšiřující modul pro software Mervis, případně tvorba vlatní vizualizace zachytávaných dat. 
\end{enumerate}

Jelikož žádný z dostupných softwarů pro UniPi nepodporuje daný bezdrátový modul, ani UART zařízení obecně, je nutné tuto komunikaci implementovat již na úrovni operačního systému.

%%%%%%%%%%%%%%%%%%%%%%%%%%%%%%%%%%%%%%%%%%%%%%%%%%%%%%%%%%%
%%%%%%%%%%%%%%%%%%%%%%%%%%%%%%%%%%%%%%%%%%%%%%%%%%%%%%%%%%%
%%%%%%%%%%%%%%%%%%%%%%%%%%%%%%%%%%%%%%%%%%%%%%%%%%%%%%%%%%%
%%%%%%%%%%%%%%%%%%%%%%%%%%%%%%%%%%%%%%%%%%%%%%%%%%%%%%%%%%%

\section{Výběr OS}
Jako operační sytém je využita aktuální verze Raspbianu Jessie s datem vydání 2017-01-11. UART rozhraní se na RaspberryPi verze 1 a 2 nachází v /dev/ttyAMA0. To se ale v případě RaspberryPi 3 odkazuje na integrovaný BT modul a původní sériový port je zde v /dev/ttyS0. Samotné UART rozhraní je ale ve výchozím nastavení Raspbianu zakázáno.

Pro zpřístupnění UART rozhraní je nutné provést drobné úpravy jeho konfigurace:


\begin{enumerate}
	\item Nejdříve je nutné provést kompletní aktualizaci Raspbianu, tedy v konzoli spustit posloupnost příkazů:
	
	\begin{lstlisting}[style=MyCodeBash]
		sudo apt-get update
		sudo apt-get upgrade
		sudo apt-get dist-upgrade
		sudo apt-get rpi-upgrade	
	\end{lstlisting}
					
	\item Poté je potřeba v /boot/config.txt změnit položku ENABLE\_UART na hodnotu 1. Tím dojde k zpřístupnení sběrnice UART. Tato položka může být v budoucnu při aktualizaci Raspbianu přepsána, proto při prvním náznaku nefunkčnosti, je potřeba tuto položku zkontrolovat jako první.
	\item V souboru /boot/cmdline.txt je potřeba odebrat úsek textu \textit{console=ttyAMA0, 115200}, aby při startování systému nedocházelo k výpisu do seríové linky. 
	\item V případě, že se jedná o RaspberryPi verze 3, je potřeba do /boot/config.txt dopsat položku \textit{dtoverlay=pi3-miniuart-bt}, která zakáže BT na mini-UART a provede přemapování zpět na /dev/ttyAMA0. Tento krok je takto řešený z důvodu kompatibility, kdy je sériová komunikace směrována přes /dev/ttyAMAO nezávisle na použité verzi RaspberryPi.
\end{enumerate}

Po každém z těchto kroků je doporučován restart zařízení. Kroky byly otestovány pouze na výše zmíněné verzi Raspbianu a v jiných distribucích se monou mírně lišit. Úspěšnost provedení těchto kroků lze zkontrolovat pomocí zadání příkazu konzole 
	\begin{lstlisting}[style=MyCodeBash]
			sudo dmesg | grep tty
	\end{lstlisting}

jehož výstup by měl být následující:
					
	\begin{lstlisting}[style=MyCodeBash]
[    0.000974] console [tty1] enabled
[    0.130442] 20201000.uart: ttyAMA0 at MMIO 0x20201000 (irq = 81, base_baud = 0) is a PL011 rev2
	\end{lstlisting}


%%%%%%%%%%%%%%%%%%%%%%%%%%%%%%%%%%%%%%%%%%%%%%%%%%%%%%%%%%%
%%%%%%%%%%%%%%%%%%%%%%%%%%%%%%%%%%%%%%%%%%%%%%%%%%%%%%%%%%%
%%%%%%%%%%%%%%%%%%%%%%%%%%%%%%%%%%%%%%%%%%%%%%%%%%%%%%%%%%%
%%%%%%%%%%%%%%%%%%%%%%%%%%%%%%%%%%%%%%%%%%%%%%%%%%%%%%%%%%%

\section{Výběr programovacího jazyka}
Jelikož primárním jazykem využívaným na platformě RaspberryPi je Python, který již obsahuje knihovny pro sériovou komunikaci, je současný kód napsán v jazyce Python 3.

%%%%%%%%%%%%%%%%%%%%%%%%%%%%%%%%%%%%%%%%%%%%%%%%%%%%%%%%%%%
%%%%%%%%%%%%%%%%%%%%%%%%%%%%%%%%%%%%%%%%%%%%%%%%%%%%%%%%%%%
%%%%%%%%%%%%%%%%%%%%%%%%%%%%%%%%%%%%%%%%%%%%%%%%%%%%%%%%%%%
%%%%%%%%%%%%%%%%%%%%%%%%%%%%%%%%%%%%%%%%%%%%%%%%%%%%%%%%%%%

\section{Nastavení komunikačního modulu a čidla}

Před samotným vyčítáním dat bylo potřeba zjistit či nastavit přenosové parametry všech použitých zařízení:

\begin{itemize}
	\item Komunikační modul IQRF nastaven do módu T ve funkci skeneru.
	\item Čidlo Weptech je nastaveno do módu T1 s intervalem zasílání 1 minuta. 
	\item Modul Bonega je nastaven do módu T1 se zapnutým šifrováním AES128 v módu 5 a s intervalem zasílání 20-24 sekund v odpočtovém období a 4 intervalem minuty mimo odpočtové období.
	\item \colorbox[rgb]{0,1,0}{bude tu neco dal? ZPA, Pikkerton, Kamstrup?}
\end{itemize}

%%%%%%%%%%%%%%%%%%%%%%%%%%%%%%%%%%%%%%%%%%%%%%%%%%%%%%%%%%%
%%%%%%%%%%%%%%%%%%%%%%%%%%%%%%%%%%%%%%%%%%%%%%%%%%%%%%%%%%%
%%%%%%%%%%%%%%%%%%%%%%%%%%%%%%%%%%%%%%%%%%%%%%%%%%%%%%%%%%%
%%%%%%%%%%%%%%%%%%%%%%%%%%%%%%%%%%%%%%%%%%%%%%%%%%%%%%%%%%%

\section{Zajištění dedikovaného běhu}
Pro zajištění běhu aplikace nezávisle na typu provozu RaspberryPi bude daný program spouštěn ihned po startu operačního systému pomocí příkazu screen. Je tedy nutné ho doinstalovat:
 
\begin{lstlisting}[style=MyCodeBash]
		sudo apt-get update
		sudo install screen		
	\end{lstlisting}


%%%%%%%%%%%%%%%%%%%%%%%%%%%%%%%%%%%%%%%%%%%%%%%%%%%%%%%%%%%
%%%%%%%%%%%%%%%%%%%%%%%%%%%%%%%%%%%%%%%%%%%%%%%%%%%%%%%%%%%
%%%%%%%%%%%%%%%%%%%%%%%%%%%%%%%%%%%%%%%%%%%%%%%%%%%%%%%%%%%
%%%%%%%%%%%%%%%%%%%%%%%%%%%%%%%%%%%%%%%%%%%%%%%%%%%%%%%%%%%

\section{Zajištění podpory šifrování}
Některá ze zařízení používají pro přenos dat šifrování AES. Pro zajištění podpory šifrování byla zvolena knihovna PyCrypto, která podporuje jak šifrování DES tak i AES. 
Umoňuje pohodlnou implementaci AES128 pomocí jazyku Python3. Na rozdíl od ostatních knihoven není závislá na balíčku OpenSSL a je součástí repozitářů Raspbianu. 

Je nutné doinstalovat nezbytné balíčky:	
 
\begin{lstlisting}[style=MyCodeBash]
		sudo apt-get update
		sudo install python-crypto
		sudo install python-dev
	\end{lstlisting}

%%%%%%%%%%%%%%%%%%%%%%%%%%%%%%%%%%%%%%%%%%%%%%%%%%%%%%%%%%%
%%%%%%%%%%%%%%%%%%%%%%%%%%%%%%%%%%%%%%%%%%%%%%%%%%%%%%%%%%%
%%%%%%%%%%%%%%%%%%%%%%%%%%%%%%%%%%%%%%%%%%%%%%%%%%%%%%%%%%%
%%%%%%%%%%%%%%%%%%%%%%%%%%%%%%%%%%%%%%%%%%%%%%%%%%%%%%%%%%%

\section{Zpracování dat}

\subsection{Nešifrovaný přenos}

Jednoduchým spuštěním komunikačního modulu v módu snifferu, byl zachycen telegram

\begin{verbatim}
	32002E44B05C10000000021B7A620800002F2F0A6699010AFB1
	A930202FD971D01002F2F2F2F2F2F2F2F2F2F2F2F2F879e0D0A
\end{verbatim}

který byl pomoci datasheetu použitého komunikačního modulu \cite{iqrfmodul} a čidla \cite{WeptechCidlo} analyzován, a přehledně zobrazen do tabulky \ref{PacketTableAnalysis}.

\subsection{Šifrovaný přenos}

V okamžiku kdy bylo zařízení přepnuto do šifrovaného módu dle tabulky \ref{TablukaSETUP} byl zachycen šifrovaný telegram

\begin{verbatim}
	32002e44b05c10000000021b7ac40820053ed44a38a9c3c86f5
	8210f9b979353c39dc1d5e0c873eb81994d28c099ef1d55b008
\end{verbatim}

který byl pomoci datasheetů použitého komunikačního modulu \cite{iqrfmodul} a čidla \cite{WeptechCidlo} analyzován a byly vyparsovány položky nezbytné pro dešifrování dat:
\begin{itemize}
	\item 30-33 pro informaci použitém šifrování,
	\item 8-25 pro sestavení inicializačního vektoru a
	\item 38-93 pro šifrovanou část dat.
\end{itemize}

Poté byla daná data v souladu s normou \colorbox[rgb]{0,1,0}{REF-KODA} dešifrována:


..\colorbox[rgb]{0,1,0}{popsat prakticky jak se to dešifrovalo}..


 a byl získán dešifrovaný telegram:

\begin{verbatim}
	32002e44b05c10000000021b7ac40820052F2F0A6699010AFB1
	A930202FD971D01002F2F2F2F2F2F2F2F2F2F2F2F2F879e0D0A
\end{verbatim}

Poté lze dešifrovaná data vyparsovat jako při nešifrovaném přenosu popsaném v předchozí kapitole.

\begin{table}[]
\centering
\caption{Rozklíčovaný zachycený paket}
\resizebox{\textwidth}{!}{%
\label{PacketTableAnalysis}
\begin{tabular}{|c|c|c|l|l|c|c|l|}
\hline
\textbf{Pozice} & \textbf{\begin{tabular}[c]{@{}c@{}}Tele-\\ gram\end{tabular}} & \textbf{Vrstva} & \multicolumn{1}{c|}{\textbf{Pole}} & \multicolumn{1}{c|}{\textbf{Popis}} & \textbf{Hodnota} & \textbf{\begin{tabular}[c]{@{}c@{}}Číselné\\ vyjádření\end{tabular}} & \multicolumn{1}{c|}{\textbf{Význam pro uživatele}} \\ \hline
0 & 32 & \begin{tabular}[c]{@{}c@{}}IQRF\\ obálka\end{tabular} &  &  &  &  &  \\ \hline
2 & 0 & \begin{tabular}[c]{@{}c@{}}IQRF\\ obálka\end{tabular} &  &  &  &  &  \\ \hline
4 & 2E & \begin{tabular}[c]{@{}c@{}}Linková\\ vrstva\end{tabular} & L-Pole & Délka telegramu & 1Eh & 46 & Paket má 46 bytů \\ \hline
6 & 44 & \begin{tabular}[c]{@{}c@{}}Linková\\ vrstva\end{tabular} & C-Pole & Typ telegramu & 44h & 44 & \begin{tabular}[c]{@{}l@{}}Paket je typu \\ SND-NR\end{tabular} \\ \hline
8 & B0 & \begin{tabular}[c]{@{}c@{}}Linková\\ vrstva\end{tabular} & M-Pole & Výrobce zařízení & B0h & 5CB0 & \begin{tabular}[c]{@{}l@{}}Výrobcem \\ je WEPtech\end{tabular} \\ \hline
10 & 5C & \begin{tabular}[c]{@{}c@{}}Linková \\ vrstva\end{tabular} & M-Pole & Výrobce zařízení & 5Ch &  &  \\ \hline
12 & 10 & \begin{tabular}[c]{@{}c@{}}Linková\\ vrstva\end{tabular} & A-Pole & Sériové číslo & 11h & 10 & SN je 00000010 \\ \hline
14 & 00 & \begin{tabular}[c]{@{}c@{}}Linková\\ vrstva\end{tabular} & A-Pole & Sériové číslo & 47h &  &  \\ \hline
16 & 00 & \begin{tabular}[c]{@{}c@{}}Linková\\ vrstva\end{tabular} & A-Pole & Sériové číslo & 15h &  &  \\ \hline
18 & 00 & \begin{tabular}[c]{@{}c@{}}Linková \\ vrstva\end{tabular} & A-Pole & Sériové číslo & 08h &  &  \\ \hline
20 & 02 & \begin{tabular}[c]{@{}c@{}}Linková\\ vrstva\end{tabular} & A-Pole & Verze zařízení & 01h & 2 & \begin{tabular}[c]{@{}l@{}}Druhá verze \\ zařízení\end{tabular} \\ \hline
22 & 1B & \begin{tabular}[c]{@{}c@{}}Linková\\ vrstva\end{tabular} & A-Pole & Typ zařízení & 1Bh & 1B & \begin{tabular}[c]{@{}l@{}}Zařízení je \\ pokojové čidlo\end{tabular} \\ \hline
24 & 7A & \begin{tabular}[c]{@{}c@{}}Aplikační\\ vrstva\end{tabular} & Ci-Pole & Odpověd od zařízení & 7Ah & 7A & jedná se o M-Bus \\ \hline
26 & 62 & \begin{tabular}[c]{@{}c@{}}Aplikační\\ vrstva\end{tabular} & AccNo & Číslo přístupu & 41h & 214 & 214. přístup \\ \hline
28 & 08 & \begin{tabular}[c]{@{}c@{}}Aplikační\\ vrstva\end{tabular} & Status & Status zařízení & 00h & 8 &  \\ \hline
30 & 00 & \begin{tabular}[c]{@{}c@{}}Aplikační\\ vrstva\end{tabular} & \begin{tabular}[c]{@{}l@{}}Config.\\ word\end{tabular} & Šifrování AES & 00h &  &  \\ \hline
32 & 00 & \begin{tabular}[c]{@{}c@{}}Aplikační\\ vrstva\end{tabular} & \begin{tabular}[c]{@{}l@{}}Config.\\ word\end{tabular} & Šifrování AES & 00h &  &  \\ \hline
34 & 2F & Data & \begin{tabular}[c]{@{}l@{}}AES \\ encryption\end{tabular} & Šifrování AES & 2Fh &  &  \\ \hline
36 & 2F & Data & \begin{tabular}[c]{@{}l@{}}AES \\ encryption\end{tabular} & Šifrování AES & 2Fh &  &  \\ \hline
38 & 0A & Data & DR1 & DIF: 4 cifry BCD & 0Ah &  &  \\ \hline
40 & 66 & Data & DR1 & \begin{tabular}[c]{@{}l@{}}VIF: první \\ měřená veličina\end{tabular} & 66h & 66 & Teplota v \degree\,C\textsuperscript{-1} \\ \hline
42 & 99 & Data & DR1 & hodnota teploty & 99h & 0199 & Teplota je 19.9\degree\,C \\ \hline
44 & 01 & Data & DR1 & hodnota teploty & 01h &  &  \\ \hline
46 & 0A & Data & DR2 & DIF: 4 cifry BCD & 0Ah &  &  \\ \hline
48 & FB & Data & DR2 & \begin{tabular}[c]{@{}l@{}}VIF: První \\ rozšiřovací \\ tabulka\end{tabular} & FBh &  &  \\ \hline
50 & 1A & Data & DR2 & \begin{tabular}[c]{@{}l@{}}VIFE: druhá \\ měřená veličina\end{tabular} & 1Ah & 1A & Relativní vlhkost v \%\textsuperscript{-1} \\ \hline
52 & 93 & Data & DR2 & hodnota vlhkosti & 93h & 0293 & Vlhkost je 29.3\,\% \\ \hline
54 & 02 & Data & DR2 & hodnota vlhkosti & 02h &  &  \\ \hline
\end{tabular}}
\end{table}

\newpage

\begin{table}[t]
\centering
\resizebox{\textwidth}{!}{%
\begin{tabular}{|c|c|c|l|l|c|c|l|}
\hline
\textbf{Pozice} & \textbf{\begin{tabular}[c]{@{}c@{}}Tele-\\ gram\end{tabular}} & \textbf{Vrstva} & \multicolumn{1}{c|}{\textbf{Pole}} & \multicolumn{1}{c|}{\textbf{Popis}} & \textbf{Hodnota} & \textbf{\begin{tabular}[c]{@{}c@{}}Číselné\\ vyjádření\end{tabular}} & \multicolumn{1}{c|}{\textbf{Význam pro uživatele}} \\ \hline
56 & 02 & Data & DR3 & \begin{tabular}[c]{@{}l@{}}DIF: 16bit \\ integer/binary\end{tabular} & 02h &  &  \\ \hline
58 & FD & Data & DR3 & \begin{tabular}[c]{@{}l@{}}VIF: Druhá \\ rozšiřovací \\ tabulka\end{tabular} & FDh &  &  \\ \hline
60 & 97 & Data & DR3 & \begin{tabular}[c]{@{}l@{}}VIFE0: Chybové \\ stavy\end{tabular} & 97h &  &  \\ \hline
62 & 1D & Data & DR3 & VIFE1: Norma & 1Dh &  &  \\ \hline
64 & 01 & Data & DR3 & Příznak sabotáže & 00h & 1 & Čidlo bylo otevřeno \\ \hline
66 & 00 & Data & DR3 & \begin{tabular}[c]{@{}l@{}}Příznak vybité\\  baterie\end{tabular} & 00h & 0 & Baterie je nabitá \\ \hline
68 & 2F & Data & Fill & Výplňové byty & 2Fh &  &  \\ \hline
... & 2F & Data & Fill & \begin{tabular}[c]{@{}l@{}}Výplňové byty \\ (11x)\end{tabular} & 2Fh &  &  \\ \hline
92 & 2F & Data & Fill & Výplňové byty & 2Fh &  &  \\ \hline
94 & 87 & \begin{tabular}[c]{@{}c@{}}IQRF\\  obálka\end{tabular} & CRC & Kontrolní součet & 87h &  &  \\ \hline
96 & 9e & \begin{tabular}[c]{@{}c@{}}IQRF \\ obálka\end{tabular} & RSSI & \begin{tabular}[c]{@{}l@{}}Síla přijímaného \\ signálu\end{tabular} & 9Eh & 158 & \begin{tabular}[c]{@{}l@{}}Síla signálu \\ je -51dBm\end{tabular} \\ \hline
98 & 0D & \begin{tabular}[c]{@{}c@{}}UART \\ obálka\end{tabular} &  & Konec řádku & 0Dh &  & CR znak \\ \hline
100 & 0A & \begin{tabular}[c]{@{}c@{}}UART\\  obálka\end{tabular} &  & \begin{tabular}[c]{@{}l@{}}Příkaz pokračuj \\ dále jako Sniffer\end{tabular} & 0Ah &  &  \\ \hline
\end{tabular}}
\end{table}


Z tabulky je patrné, že nutné vyparsovat položky na následujících pozicích:
\begin{itemize}
	\item 8-23 pro informace o daném čidlu,
	\item 24-25 pro určení pořadí telegramu,
	\item 42-45 pro hodnotu naměřené teploty,
	\item 52-55 pro hodnotu naměřené vlhkosti,
	\item 64-67 pro kontrolu stavu čidla a
	\item 96 pro úroveň signálu.	
\end{itemize}

a jejich následnou správnou interpretací dle specifikace (zohlednění uložení LSB, převod hexadecimálních hodnot na dekadické) předat k dalšímu zpracování či uložení do databáze.


\section{Zajištění uložení dat}
Zachycená a naměřená data se ukládají do databáze k pozdějšímu zpracování. Zvolena byla databáze SqLite3 pro svoji jednoduchost, nenáročnost na sytémové prostředky a možností instalace z repozitáře Raspbianu:
 
\begin{lstlisting}[style=MyCodeBash]
		sudo apt-get update
		sudo apt-get install sqlite3
	\end{lstlisting}

Byla zvolena jedna databáze se třemi tabulkami:
\begin{itemize}
	\item DEVICES - evidence známých zařízení a jejich AES klíčů
	\item VALUES - uložení naměřených hodnot
	\item TELEGRAMS - uložení zachycených dat a AES klíče modulu
\end{itemize}

Struktura tabulek je popsána na obrázku... \colorbox[rgb]{0,1,0}{toto je stale tezce pracovni verze}
	

\section{Struktura aplikace}
Vzhledem k výše uvedeným požadavakům a technologiím byla zvolena struktura aplikace znázorněná na obrázku \ref{AplikaceDiagram}. Diagram je pro přehlednost odlišen barevnými bloky:
\begin{itemize}
	\item modrou barvou je znázorněna kostra programu,
	\item zelenou barvou je nekonečná smyčka naslouchání dat,
	\item růžovou barvou je případné dešifrování přenášených dat,
	\item červenou barvou jsou chyby znemožnující běh programu,
	\item oranžovou barvou jsou chyby znemožňující platnou analýzu či dešifrování daného telegramu a
	\item černou barvu je řízení samotného programu.
\end{itemize}

V následujících podkapitolách budou jednotlivé bloky aplikace představeny podrobněji.

\subsection{Start programu v rámci operačního systému}
Program je nyní spouštěn automaticky po startu operačního systému interpretem jazyka Python v příkazu screen. \colorbox[rgb]{0,1,0}{Zeptat se RŠ jak z toho udělat službu nebo tak něco.} Tím je zajištěna nezávilost na typu nasazení RaspberryPi a případných restartech zařízení. Ukončení programu nastává pouze násilným ukončením aplikace, restartem zařízení nebo závažnou chybou při startu programu. 

 \begin{figure}[!]
  \begin{center}
    \includegraphics[scale=0.6]{obrazky/aplikace_diagram}
  \end{center}
  \caption{Vývojový diagram aplikace pro vyčítání dat}
	\label{AplikaceDiagram}
\end{figure}

\subsection{Start programu z pohledu aplikace}
Program při startu kontroluje, zdali  má k dispozici všechny potřebné komponenty pro svůj běh. Program je závislý na knihovně PyCrypto, Serial nebo SQLite databázi.
Dále program kontroluje přítomnost a možnost otevření sériového portu. V případě úspěšného otevření portu je na něj zaslán příznak pro probuzení komunikačního modulu z úsporného režimu. Po probuzení následujě příkaz, který nastaví modul do režimu skeneru v komunikačním módu T1. Pokud některá z operací selže, je zaznamenán chybový stav a dojde k ukončení programu.

\colorbox[rgb]{0,1,0}{dál je jsou pouze poznámky, zatím ve výstavbě}

\subsection{Základní kontrola a parsování dat}
\begin{itemize}
\item - jak se resi datasheety specifikaci
\item - co se tam kontroluje (signatura, rozsifrovani, delka paketu, statove bajty)
\item - jak se zjisti jestli se ma desifrovat
\end{itemize}

\subsection{Dešifrování dat}
\begin{itemize}
\item - proc se to tak debilne desifruje
\item - kde se berou ty klice
\item - jak se pocita IV
\item - jak se kontroluje desifrovani
\end{itemize}

\subsection{Parsování dat}
\begin{itemize}
\item - co to umí
\item - jaká je aplikační logika pro RSSI a VendorID
\item - jak se to dělá pro jednotlivé výrobce
\item - co se tam kontroluje (signatura, rozsifrovani, delka paketu, statove bajty)
\end{itemize}

\subsection{Uložení dat}
\begin{itemize}
\item - jak se osetri ktere chyby
\item - co davame kam jak na vedomi
\item - co se zapise do logu a co do db
\end{itemize}

\subsection{Ošetření chyb a vyjímek}
\begin{itemize}
\item - jak se ukladaji vysledky
\item - jak to vypada v DB, kolik je tabulek s cim
\item - kdy se kam co uklada a v jakem formatu
\item - co se zapise do logu a co do db
\end{itemize}











