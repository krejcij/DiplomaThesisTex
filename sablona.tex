% Soubory musí být v kódování, které je nastaveno v příkazu \usepackage[...]{inputenc}

\documentclass[%
%  draft,    				  % Testovací překlad
  12pt,       				% Velikost základního písma je 12 bodů
  a4paper,    				% Formát papíru je A4
%  oneside,      			% Jednostranný tisk (výchozí)
%% Z následujicich voleb lze použít maximálně jednu:
%	dvipdfm  						% výstup bude zpracován programem 'dvipdfm' do PDF
%	dvips	  						% výstup bude zpracován programem 'dvips' do PS
%	pdftex							% překlad bude proveden programem 'pdftex' do PDF (výchozí)
%% Z následujících voleb lze použít jen jednu:
%english,            % originální jazyk je angličtina
czech              % originální jazyk je čeština (výchozí)
%slovak,             % originální jazyk je slovenčina
]{report}				    	% Dokument třídy 'zpráva'

\usepackage[utf8]		%	Kódování zdrojových souborů je Windows-1250
	{inputenc}					% Balíček pro nastavení kódování zdrojových souborů

\usepackage{graphicx} % Balíček 'graphicx' pro vkládání obrázků
											% Nutné pro vložení log školy a fakulty
	\RequirePackage[hyphens]{url}										
											
\usepackage[
	nohyperlinks				% Nebudou tvořeny hypertextové odkazy do seznamu zkratek
]{acronym}						% Balíček 'acronym' pro sazby zkratek a symbolů
											% Nutné pro použití prostředí 'seznamzkratek' balíčku 'thesis'

\usepackage[
	unicode,						% Záložky a informace budou v kódování unicode
	breaklinks=true,		% Hypertextové odkazy mohou obsahovat zalomení řádku
	hypertexnames=false % Názvy hypertextových odkazů budou tvořeny
											% nezávisle na názvech TeXu
]{hyperref}						% Balíček 'hyperref' pro sazbu hypertextových odkazů
											% Nutné pro použití příkazu 'nastavenipdf' balíčku 'thesis'

\usepackage{pdfpages} % Balíček umožňující vkládat stránky z PDF souborů
                      % Nutné při vkládání titulních listů a zadání přímo
                      % ve formátu PDF z informačního systému

\usepackage{enumitem} % Balíček pro nastavení mezerování v odrážkách
  \setlist{topsep=0pt,partopsep=0pt,noitemsep}

\usepackage{cmap} 		% Balíček cmap zajišťuje, že PDF vytvořené `pdflatexem' je
											% plně "prohledávatelné" a "kopírovatelné"

\usepackage{upgreek}	% Balíček pro sazbu stojatých řeckých písmem
											% např. stojaté pí: \uppi
											% např. stojaté mí: \upmu (použitelné třeba v mikrometrech)
											% pozor, grafická nekompatibilita s fonty typu Computer Modern!

%% Nastavení českého jazyka při sazbě v češtině.
% Pro sazbu češtiny je možné použít mezinárodní balíček 'babel', jenž
% použití doporučujeme pro nové instalace (MikTeX2.8,TeXLive2009), nebo
% národní balíček 'czech', který doporučujeme ve starších instalacích.
% Balíček 'babel' bude správně fungovat pouze ve spojení s programy
% 'latex', 'pdflatex', zatímco balíček 'czech' bude fungovat ve spojení
% s programy 'cslatex', 'pdfcslatex'.
% Varianta A:
\usepackage    				
  {babel}             % Balíček pro sazbu různojazyčných dokumentů; kompilovat (pdf)latexem!
  										% převezme si z parametrů třídy správný jazyk
\usepackage{lmodern}	% vektorové fonty Latin Modern, nástupce půvoních Knuthových Computern Modern fontů
\usepackage{textcomp} % Dodatečné symboly
\usepackage[T1]{fontenc}  % Kódování fontu - mj. kvůli správným vzorům pro dělení slov
% Varianta B:
%\usepackage{czech}   % Alternativní balíček pro sazbu v českém jazyce, kompilovat (pdf)cslatexem!

% Moje  vlastni formatovaci styly
\usepackage{xcolor}
\usepackage{listings}
\usepackage{gensymb}

\definecolor{BGgray}{rgb}{0.9,0.9,0.9}
\definecolor{BGblack}{rgb}{0.1,0.1,0.1}

\lstdefinestyle{numbers} {numbers=left, stepnumber=1, numberstyle=\tiny, numbersep=10pt}

\lstdefinestyle{MyFramePython}{}

\lstdefinestyle{MyFrameJava}{}

\lstdefinestyle{MyFrameBash}{backgroundcolor=\color{white}, basicstyle=\footnotesize\color{black},frame=shadowbox,aboveskip=5mm,belowskip=5mm,keywordstyle=\color{blue},commentstyle=\color{brown},stringstyle=\color{purple},breaklines=false,breakatwhitespace=true,showspaces=false, showstringspaces=false, numbers=none, showtabs=false,}

\lstdefinestyle{MyCodePython} {language=Python,style=numbers,numbers=right,style=MyFramePython,frame=single}
\lstdefinestyle{MyCodeBash} {basicstyle=\footnotesize,language=Bash,style=numbers,style=MyFrameBash,frame=single}
\lstdefinestyle{MyCodeJava} {language=Java,style=MyFrameJava,frame=none}

\lstset{language=Bash,frame=none,basicstyle=\tiny}

\lstset{language=Java,frame=none,basicstyle=\footnotesize,backgroundcolor=\color{white},aboveskip=5mm,belowskip=5mm,breaklines=true,breakatwhitespace=true,tabsize=20,showspaces=false,showstringspaces=false, showtabs=false}

\lstset{language=Python,frame=single,basicstyle=\footnotesize,aboveskip=5mm,belowskip=5mm,breaklines=true,breakatwhitespace=true,tabsize=4,showspaces=false,showstringspaces=false, showtabs=false, , keywordstyle=\color{blue},commentstyle=\color{brown},stringstyle=\color{purple}}

\renewcommand{\lstlistingname}{Kód}% Listing -> Algorithm
\renewcommand{\lstlistlistingname}{Seznam ukázek zdrojových kódů}% List of Listings -> List of Algorithms




\usepackage[%
%% Z následujících voleb lze použít pouze jednu
% left,               % Rovnice a popisky plovoucich objektů budou %zarovnány vlevo
  center,             % Rovnice a popisky plovoucich objektů budou zarovnány na střed (vychozi)
%% Z následujících voleb lze použít pouze jednu
%semestral						%	sazba zprávy semestrálního projektu
%bachelor						%	sazba bakalářské práce
diploma						 % sazba diplomové práce
%treatise            % sazba pojednání o dizertační práci
%phd                 % sazba dizertační práce
]{thesis}             % Balíček pro sazbu studentských prací
                      % Musí být vložen až jako poslední, aby
                      % ostatní balíčky nepřepisovaly jeho příkazy

%%%%%%%%%%%%%%%%%%%%%%%%%%%%%%%%%%%%%%%%%%%%%%%%%%%%%%%%%%%%%%%%%
%%%%%%      Definice informací o dokumentu             %%%%%%%%%%
%%%%%%%%%%%%%%%%%%%%%%%%%%%%%%%%%%%%%%%%%%%%%%%%%%%%%%%%%%%%%%%%%

%% Název práce:
%  První parametr je název v originálním jazyce,
%  druhý je překlad v angličtině nebo češtině (pokud je originální jazyk angličtina)
\nazev{Implementace komunikačních protokolů pro IoT s využitím rozšiřujícího modulu UniPi pro RaspberryPi}
{Implementation of IoT Communication Protocols Utilizing UniPi Module for Raspberry Pi}

%% Jméno a příjmení autora ve tvaru
%  [tituly před jménem]{Křestní}{Příjmení}[tituly za jménem]
\autor[PhDr.]{Jan}{Krejčí}

% Jméno a příjmení vedoucího včetně titulů
%  [tituly před jménem]{Křestní}{Příjmení}[tituly za jménem]
\vedouci[Ing.]{Pavel}{Mašek}

%% Označení oboru studia
% První parametr je obor v originálním jazyce,
% druhý parametr je překlad v angličtině nebo češtině
\oborstudia{Teleinformatika}{Teleinformatics}

%% Označení ústavu
% První parametr je název ústavu v originálním jazyce,
% druhý parametr je překlad v angličtině nebo češtině
\ustav{Ústav telekomunikací}{Department of Telecommunications} 

%% Rok obhajoby
\rok{2017}

%% Místo obhajoby
% Na titulních stránkách bude automaticky vysázeno VELKÝMI písmeny
\misto{Brno}

%% Abstrakt
\abstrakt{
Předkládaná diplomová práce je zaměřena na implementaci protokolu Wireless M-Bus do embedded zařízení RaspberryPi za pomocí rozšiřující desky UniPi. Protokol je implementován v jazyce Python a s Wireless M-Bus zařízeními komunikuje pomocí komunikačního modulu IQRF připojeného na sběrnici UART. Teoretická část práce se zaměřuje na přehled embedded zařízení pro IoT, možnosti jejich rozšíření, popisuje danou rozšiřující desku i Wireless M-Bus komunikační modul. Podrobněji se zaměřuje na vrstvy protokolu Wireless M-bus, čímž poskytuje základy potřebné pro porozumění praktické části. V praktické části je provedena implementace aplikace pro vyčítání dat z Wireless M-Bus senzorů a jejich následnou vizualizaci. Aplikace je schopna vyčítat i zařízení umožňující šifrovaný přenos.
}
{
Presented diploma thesis is focused on the implementation of Wireless M-Bus protocol to embedded device RaspberryPi with expansion board UniPi. The protocol is implemented in Python with Wireless M-Bus devices communicating via IQRF transceiver connected to the UART bus. The theoretical part is focused on an overview of embedded devices for the IoT, the possibility of their expansion, describes the UniPi expansion board and Wireless M-Bus transceiver. More specifically focuses on the Wireless M-bus layers, which provides a basic knowledge for understanding the practical part. In the practical part is the implementation of the sample application for retrieving data from a Wireless M-Bus sensors.
}


%% Klíčová slova
\klicovaslova{IoT, RaspberryPi, UniPi, Wireless M-Bus, EN 13757-4, Python, Pikkerton, Bonega, Weptech, ZPA}%
	{IoT, RaspberryPi, UniPi, Wireless M-Bus, EN 13757-4, Python, Pikkerton, Bonega, Weptech, ZPA}

%% Poděkování
\podekovanitext{Rád bych poděkoval vedoucímu diplomové práce panu Ing.\ Pavlu Maškovi za odborné vedení, konzultace, trpělivost a podnětné návrhy k~práci.}

%%%%%%%%%%%%%%%%%%%%%%%%%%%%%%%%%%%%%%%%%%%%%%%%%%%%%%%%%%%%%%%%%%%%%%%%

%%%%%%%%%%%%%%%%%%%%%%%%%%%%%%%%%%%%%%%%%%%%%%%%%%%%%%%%%%%%%%%%%%%%%%%%
%%%%%%     Nastavení polí ve Vlastnostech dokumentu PDF      %%%%%%%%%%%
%%%%%%%%%%%%%%%%%%%%%%%%%%%%%%%%%%%%%%%%%%%%%%%%%%%%%%%%%%%%%%%%%%%%%%%%
%% Při vloženém balíčku 'hyperref' lze použít příkaz '\nastavenipdf'
\nastavenipdf
%  Nastavení polí je možné provést také ručně příkazem:
%\hypersetup{
%  pdftitle={Název studentské práce},    	% Pole 'Document Title'
%  pdfauthor={Autor studenstké práce},   	% Pole 'Author'
%  pdfsubject={Typ práce}, 						  	% Pole 'Subject'
%  pdfkeywords={Klíčová slova}           	% Pole 'Keywords'
%}
%%%%%%%%%%%%%%%%%%%%%%%%%%%%%%%%%%%%%%%%%%%%%%%%%%%%%%%%%%%%%%%%%%%%%%%

%%%%%%%%%%%%%%%%%%%%%%%%%%%%%%%%%%%%%%%%%%%%%%%%%%%%%%%%%%%%%%%%%%%%%%%
%%%%%%%%%%%       Začátek dokumentu               %%%%%%%%%%%%%%%%%%%%%
%%%%%%%%%%%%%%%%%%%%%%%%%%%%%%%%%%%%%%%%%%%%%%%%%%%%%%%%%%%%%%%%%%%%%%%
\begin{document}


%% Vložení desek generovaných informačním systémem
%\includepdf[pages=1,offset=19mm 0mm]%
%  {pdf/student-desky}% název souboru nesmí obsahovat mezery!
% nebo vytvoření desek z balíčku
%\vytvorobalku
%\setcounter{page}{1} %resetovani citace stranek - desky se necisluji

%% Vložení titulního listu generovaného informačním systémem
%\includepdf[pages=1,offset=19mm 0mm]{pdf/student-titulka}% název souboru nesmí obsahovat mezery!
% nebo vytvoření titulní stránky z balíčku
%\vytvortitulku
   
%% Vložení zadání generovaného informačním systémem
%\includepdf[pages=1,offset=19mm 0mm]{pdf/student-zadani}% název souboru nesmí obsahovat mezery!
% nebo lze vytvořit prázdný list příkazem ze šablony
%\stranka{}%
%	{\sffamily\Huge\centering ZDE VLOŽIT LIST ZADÁNÍ}%
%	{\sffamily\centering Z~důvodu správného číslování stránek}

%% Vysázení stránky s abstraktem
%\vytvorabstrakt

%% Vysázení prohlaseni o samostatnosti
%\vytvorprohlaseni

%% Vysázení poděkování
%\vytvorpodekovani

%% Vysázení poděkování projektu SIX
% ----------- zakomentujte pokud neodpovida realite
%\vytvorpodekovaniSIX

%% Vysázení obsahu
\obsah

%% Vysázení seznamu obrázků
\seznamobrazku

%% Vysázení seznamu tabulek
\seznamtabulek

%% Vložení souboru 'text/uvod.tex'
\chapter*{Úvod}
\phantomsection
\addcontentsline{toc}{chapter}{Úvod}

Fenoménem dneška je~propojování Internetu věcí (IoT - Internet of~Things), služeb (IoS - Internet of~Services) a~lidí (IoP - Internet of~People) a~s~ním související vývoj komunikací stroj-stroj (M2M - Machine to~Machine), člověk-stroj (H2M - Human to~Machine) nebo člověk-člověk (H2H - Human to~Human). Internet věcí, služeb a~lidí se rozšiřuje závratným tempem a~proniká tak do~odvětví, ve~kterých se~rostoucím tempem využívají komunikační nízkovýkonové (embedded) zařízení a~roste potřeba rozšíření těchto zařízení o~nové komunikační protokoly a~technologie. Vzniknou sítě založené na~propojených zařízeních, které budou schopny samostatné výměny informací, vyvolání potřebných akcí v~reakci na~momentální podmínky a vzájemné nezávislé kontroly. Senzory, přístroje a~IT systémy budou vzájemně propojeny a~budou na~sebe pomocí standardních komunikačních protokolů vzájemně reagovat a~analyzovat data, aby mohly předvídat případné chyby či~poruchy, konfigurovat samy sebe a~v~reálném čase se přizpůsobovat změněným 
podmínkám \cite{uvod_prumysl_4_pdf,uvod_prumysl_4_web}.

Tato práce vychází z~požadavku na~implementaci Wireless M-Bus protokolu do~produktu UniPi NEURON. K~tomuto účelu bylo zvoleno nízkovýkonové (embedded) zařízení RaspberryPi a~jeho rozšiřující modul UniPi. Pro M2M komunikaci byl zvolen protokol Wireless M-Bus, jelikož je jedním z~nejrozšířenějších a~navíc je založen na~protokolu M-Bus, který je osvědčený a~velmi rozšířený (měření~a regulace topných systémů, plynu, odběru vody a~elektrické energie). V teoretické části práce jsou popsány jednotlivé rodiny jednodeskových počítačů a~jejich vlastnosti, popis rozšiřujících desek UniPi~a samotného komunikačního modulu pro~Wireless M-Bus a~popis komunikačního protokolu Wireless M-Bus. Teoretickou část uzavírá přehled vyčítaných měřících zařízení protokolu Wireless M-Bus. 

Praktická část se~zaměřuje na~implementaci Wireless M-Bus protokolu v~zařízení RaspberryPi pomocí rozšiřujícího modulu UniPi a~komunikačního modulu Wireless M-Bus. Tato implementace vyčítání dat ze~vzdálených zařízení je realizována v~jazyku Python a~následně jsou získaná data vizualizována pomocí Google Chart API~\cite{uvod_google_charts_api}.


 

%% Vložení kapitoly s kecama o IoT
\chapter{Internet věcí}
\label{ChapterInternetVeci}

Cílem Internetu věcí (IoT - Internet of Things) je propojení zařízení, systémů a~služeb za účelem poskytnutí více dat, která mohou být převedena na informace a~informace potom na znalosti, které mohou být následně aplikovány. Princip IoT je tedy sběr dat, ty jsou následně uložena a analyzována a poté dojde ke~sdílení výsledků. V rámci IoT se vytvořily dva hlavní směry, průmyslový Internet věcí (iIoT - Industry IoT) a spotřebitelský Internet věcí (cIoT - Customer IoT) \cite{iot_svet_hardware_internet_veci, iot_pohanka_internet_veci}. Rozdíly obou směrů jsou shrnuty v Tab.~\ref{TableIOT}.

\section{Spotřebitelský Internet věcí}
Spotřebitelský Internet věcí se zaměřuje na spotřebitelská zařízení, IT a telekomunikační zařízení a další. Jsou zde využívána zařízení zjednodušující každodenní život pomocí automatizace v domácnosti, chytrých zařízení nebo pomocí nositelné elektroniky. Hlavní výhodou je zvýšení uživatelského zážitku (QoE - Quality of Experience).

\section{Průmyslový Internet věcí}
Průmyslový Internet věcí vychází z M2M (Machine-to-Machine) a rozšiřuje komunikaci o možnost uložení, analýzy a zobrazení dat. Jedná se o IoT zařízení a systémy, které jsou používány v průmyslových odvětvích, jako jsou průmyslová automatizace, energetický průmysl a zdravotnictví. Hlavním zaměřením je efektivnější využívání zdrojů, snížení provozních nákladů, zvýšení efektivity či bezpečnosti. V praxi může sloužit například pro zajištění bezpečnosti pracovníků či automatizaci údržby. 


\begin{table}[!ht]
\caption{Porovnání průmyslového a spotřebitelského IoT \cite{iot_svet_hardware_internet_veci, iot_pohanka_internet_veci}}
\vspace{-10pt}
\label{TableIOT}
\begin{center}
\small
\begin{tabular}{|c|c|c|}
\hline
 & \textbf{Spotřebitelský IoT} &  \textbf{Průmyslový IoT} \\ \hline \hline
\textbf{Zaměření} & Spotřebitel. & Průmysl. \\ \hline
\textbf{Zařízení} & \begin{tabular}[c]{@{}c@{}}Chytré zařízení\\ a nositelná elektronika.\end{tabular} & \begin{tabular}[c]{@{}c@{}}Stroje, zařízení\\  a průmyslová automatizace.\end{tabular} \\ \hline
\textbf{Důležitost} & \begin{tabular}[c]{@{}c@{}}Nejedná se o životně \\ důležité systémy.\end{tabular} & \begin{tabular}[c]{@{}c@{}}Jedná se o životně\\ důležité systémy.\end{tabular} \\ \hline
\textbf{Využití} & \begin{tabular}[c]{@{}c@{}}Zvýšení uživatelského\\ zážitku.\end{tabular} & \begin{tabular}[c]{@{}c@{}}Lepší využívání zdrojů, \\  snížení provozních nákladů, \\ zvýšení efektivity či bezpečnosti.\end{tabular} \\ \hline \hline
\end{tabular}
\vspace{-10pt}
  \end{center}
\end{table}

\subsection{Průmysl 4.0}

Současný trend digitalizace a s ní související automatizace výroby je označován jako Průmysl 4.0. Koncept vychází z dokumentu, který byl představen na veletrhu v~Hannoveru v roce 2013. Předpokládá se, že v horizontu následujících 10 až 15 let nastane příchod čtvrté průmyslové revoluce, která přinese radikální změnu ve srovnání s nynějším výrobním procesem. Podle této myšlenky vzniknou chytré továrny, které budou využívat kyberneticko-fyzikální systémy. Ty převezmou opakující se~a~jednoduché činnosti, které do té doby vykonávali lidé. Má zahrnovat kompletní (viz Obr.~\ref{SchemaPrumysl4}) digitalizaci, robotizaci a automatizaci většiny současných lidských činností pro zajištění větší rychlosti a efektivity výroby přesnějších, osobitějších, spolehlivějších a levnějších produktů, současně pro efektivnější využití materiálů a ekologičtější průmysl i lidský život.

\vspace{-10pt}	
	\begin{figure}[!ht]
  \begin{center}
   \includegraphics[scale=0.4]{obrazky/iot_industry4}
  \end{center}
	\vspace{-20pt}	
  \caption{Schéma odvětví Průmyslu 4.0}
	\label{SchemaPrumysl4}
\vspace{-10pt}	
\end{figure}



Na průmyslové úrovni má jít o nahrazení manuální lidské práce robotizací, současné manuální zadávání výrobních dat a postupů má být nahrazeno automatickým elektronickým předáváním informací mezi jednotlivými výrobními komponentami a~materiálmi. Významné změny mají i ve spojistosti s automatizovaným průmyslem nastat v oblasti domácností a běžného bydlení, kde mají být jednotlivé domácí systémy vzájemně elektronicky propojeny a jejich vzájemná koordinovaná spolupráce bude maximalizovat efektivitu a současně minimalizovat spotřebu médií.

V reflexi na tento trend v září 2015 vydalo Ministerstvo průmyslu a obchodu Národní iniciativu Průmysl 4.0~\cite{uvod_prumysl_4_pdf}, podle které bude revoluce příležitostí pro růst a~konkurenceschopnost českých firem a České republiky vůbec.











%% Vložení kapitoly se srovnanim HW
\chapter{Embedded zařízení pro IoT}
\label{ChapterEmbeddedZarizeni}

V současnosti existuje velké množství zařízení v roli výpočetní jednotky, využitelných pro chytrou domácnost či Internet věcí. Tato kapitola představí nejznámější z~nich, popíše jejich možnosti, uvede možnosti připojení senzorů a zmíní jejich nedostatky. 

Mezi nejznámější nízkovýkonové (embedded) zařízení patří open-source Arduino (Kap.~\ref{KapArduino}), RaspberryPi (Kap.~\ref{KapRaspi}) a~jejich klony (Kap.~\ref{KapArduinoKlony} a \ref{KapRaspiKlony}). Poté budou zmíněny desky předních firem výrobců procesorů Intel (Kap.~\ref{KapIntel}) a AMD (Kap.~\ref{KapAMD}) a~v~neposlední řadě budou představeny desky firmy CubieBoard (Kap.~\ref{KapCubie}), HardKernel (Kap.~\ref{KapKernel}) a další. 

Z výše uvedených byly vybrány jen nízkovýkonová zařízení, využitelná pro chytrou domácnost či Internet věcí, s vyvedenými GPIO piny a dostatečnou dokumentací.

Jejich vlastnosti jsou přehledně shrnuty v tabulce přílohy \ref{PrilohaTabulkaTabulka}.

%%%%%%%%%%%%%%%%%%%%%%%%%%%%%%%%%%%%%%%%%%%%%%%%%%%%%%%%%%%%%%%%%%%%%%%%%%%%%%%%%%%%%%%%%%%%%%%%%%%%%%%%%%%%%%%%%
%%%%%%%%%%%%%%%%%%%%%%%%%%%%%%%%%%%%%%%%%%%%%%%%%%%%%%%%%%%%%%%%%%%%%%%%%%%%%%%%%%%%%%%%%%%%%%%%%%%%%%%%%%%%%%%%%
%%%%%%%%%%%%%%%%%%%%%%%%%%%%%%%%%%%%%%%%%%%%%%%%%%%%%%%%%%%%%%%%%%%%%%%%%%%%%%%%%%%%%%%%%%%%%%%%%%%%%%%%%%%%%%%%%
%%%%%%%%%%%%%%%%%%%%%%%%%%%%%%%%%%%%%%%%%%%%%%%%%%%%%%%%%%%%%%%%%%%%%%%%%%%%%%%%%%%%%%%%%%%%%%%%%%%%%%%%%%%%%%%%%
%%%%%%%%%%%%%%%%%%%%%%%%%%%%%%%%%%%%%%%%%%%%%%%%%%%%%%%%%%%%%%%%%%%%%%%%%%%%%%%%%%%%%%%%%%%%%%%%%%%%%%%%%%%%%%%%%
%%%%%%%%%%%%%%%%%%%%%%%%%%%%%%%%%%%%%%%%%%%%%%%%%%%%%%%%%%%%%%%%%%%%%%%%%%%%%%%%%%%%%%%%%%%%%%%%%%%%%%%%%%%%%%%%%

\section{Arduino}
\label{KapArduino}

Ardiuno je skupina několika jednodeskových počítačů založených na~mikrokontrolérech. Nejedná se však o klasický stolní počítač IBM PC, ale o protypovací desku, ke~které se~spíše jak ovládací a|zobrazovací periferie připojují senzory, moduly, serva a~displeje. Projekt je od svého počátku šířen jako open-source, příručka jazyka a~externí knihovny jsou pak šířeny pod licencí Creative Commons.
	
Výrobce těchto desek vytvořil vývojové prostředí shodné pro všechny produkty Ardiuno. To se nazývá Arduino IDE, je dostupné zdarma na webu výrobce a podporuje jazyk Wiring~\cite{embed_about_wiring_2011}, což je upravená verze jazyka C. Prostředí zároveň obsahuje i~Serial Monitor, který slouží k oboustranné sériové komunikaci mezi Arduinem a~PC. Alternativou ještě může být prostředí Processing~\cite{embed_about_processing_2015} využívající stejnojmenný jazyk, umožňující vytváření grafických multiplatformních aplikací.
	
Na deskách bývá několik diod, resetovací tlačítko, různé přídavné sběrnice, konektory pro ICSP (In Circuit Serial Programming) programování, napájecí konektor, oscilátor a obvod zprostředkovávající komunikaci po USB.
	
Arduino podporuje připojení rozšiřujících karet. Ty se u Arduina nazývají Shieldy, mají převážně stejný tvar jako deska Arduina a připojují se pomocí dlouhých pinů. Zabírají celou plochu, ale většina z nich dále zpřístupňuje GPIO (General Purpose Input/Output) piny, lze je tedy skládat na sebe. Stejně jako Arduino desek existuje i celá řada shieldů. Samozřejmě lze k Arduinu připojit i samotné moduly nebo senzory, přímým připojením na dané piny. Je však třeba dbát na to, že Arduino pracuje s 5\,V logikou, zatímco například RaspberryPi pracuje s 3,3\,V logikou.
	
		\subsection{Arduino Duemilanove} Arduino Duemilanove je vývojová jednoprocesorová deska s mikroprocesorem ATMega168 od firmy Atmel, tedy platformě Atmel AVR. 
		Parametry zařízení jsou: ATmega168 s~16\,MHz krystalem, 16\,KB flash, 1\,KB SRAM (Static Random Access Memory), 512B EEPROM (Electrically Erasable Programmable Read-Only Memory). 	Konektivita: 14 digitálních vstupně/výstupných pinů, z toho 6 z nich může být využito i PWM (Pulse Width Modulation), vstupních analogových pinů (10\,bit A/D převodník, 0-5\,V), I2C (Inter-Integrated Circuit) sběrnici, UART (Universal Asynchronous Receiver/Transmitter) sběrnici, ICSP rozhraní, USB (Universal Serial Bus) rozhraní~\cite{ArduinoDuemilanove}.	
	
		\subsection{Arduino Uno} Arduino Uno je v současnosti asi nejpoužívanější typ desky. Arduino Uno je vývojová jednoprocesorová deska s mikroprocesorem ATMega328. Od roku 2011 je nástupcem Arduina Duemilanove. Změny oproti předchůdci jsou pouze v použitém mikrokontroléru, došlo k zdvojnásobení velikosti paměti na 32\,KB flash, 2\,KB SRAM, 1\,KB EEPROM~\cite{ArduinoUno}.
	
		\subsection{Arduino Leonardo} 
		Arduino Leonardo vzhledem navazuje na Arduino Uno, liší se pouze v použitém čipu ATmega32u4~\cite{ArduinoLeonardo} a využitím SMD (Surface Mount Device) součástek.


\begin{figure*}[!ht]
    \centering
			\subfigure[Arduino Duemilanove]{\includegraphics[width=0.3\textwidth,height=3cm,keepaspectratio]{obrazky/emded_arduino_duemilanove}\label{emded_arduino_duemilanove}}
			\hspace*{5mm}
			\subfigure[Arduino Uno]{\includegraphics[width=0.3\textwidth,height=3cm,keepaspectratio]{obrazky/emded_arduino_uno}\label{emded_arduino_uno}}
			\hspace*{5mm}
			\subfigure[Arduino Leonardo]{\includegraphics[width=0.3\textwidth,height=3cm,keepaspectratio]{obrazky/emded_arduino_leonardo}\label{emded_arduino_leonardo}}
    \caption{Arduino Duemilanove, Uno a Leonardo}
		\vspace{-10pt}	
\end{figure*}


\newpage
	
					\subsection{Arduino Mega} 
					Arduino Mega je deska pro náročnější projekty. Oproti klasickému Arduinu má~Arduino Mega rychlejší procesor (16\,MHz) a také více vstupních a výstupních pinů. K~dispozici je 54 digitálních pinů, 14 PWM výstupů, 16 analogových vstupů a~4~hardwarové sériové porty. Dále má 256\,KB flash paměti, 8\,KB RAM paměti a 4\,KB EEPROM paměti~\cite{ArduinoMega}.	
			
		\subsection{Arduino Due} 
		Arduino Due je nástupcem Arduino Mega a je to první karta Arduino, na~níž~je~umístěn 32-bitový řadič (32-bitový ARM procesor
		Atmel SAM3X8E). Vysoká taktovací rychlost 84\,MHz ve spojení s celkem 54 I/O piny umožňuje realizaci značně rozsáhlých projektů. K 54 pinům mimo jiné patří 12 PWM výstupů a 12 analogových vstupů, 4 UARTy, 2 I2C a dvojitý digitálně-analogový měnič. Vlastní USB Host poskytuje kartě vedle standardů jako JTAG (Joint Test Action Group), SPI (Serial Peripheral Interface) a Micro USB širší možnosti konektivity~\cite{ArduinoDue}.	

\begin{figure*}[!ht]
    \centering
			\subfigure[Arduino Mega]{\includegraphics[width=0.45\textwidth,height=3.5cm,keepaspectratio]{obrazky/emded_arduino_mega}\label{emded_arduino_mega}}
			\hspace*{5mm}
			\subfigure[Arduino Due]{\includegraphics[width=0.45\textwidth,height=3.5cm,keepaspectratio]{obrazky/emded_arduino_due}\label{emded_arduino_due}}
		\caption{Arduino Mega a Due}
		\vspace{-20pt}	
\end{figure*}


	\subsection{Arduino Mini} 
	Arduino Mini je nejmenší oficiální verze Arduina, navržená pro úsporu místa. Daní za malé rozměry je~však absence USB portu. K programování je tedy nutné použít externí USB2Serial převodník. Jeho výkon však nijak nezaostává za většími deskami. Běží na procesoru ATmega328 s taktem 16\,MHz. Pro své malé rozměry je~vhodný k použití například v chytrých vypínačích, či dálkových ovladačích~\cite{ArduinoMini}.	
	
	\subsection{Arduino Micro} 
	Arduino Micro je jedna z desek, která má čip obsahující předprogramovaný převodník ATmega32u4~\cite{ArduinoMicro}.	 
		
	\subsection{Arduino Nano} 
	Arduino Nano navíc obsahuje ještě USB port a převodník~\cite{ArduinoNano}.	
		
\begin{figure*}[!ht]
	\vspace{-10pt}	
    \centering
			\subfigure[Arduino Mini]{\includegraphics[width=0.3\textwidth,height=3.5cm,keepaspectratio]{obrazky/emded_arduino_mini}\label{emded_arduino_mini}}
			\hspace*{5mm}
			\subfigure[Arduino Micro]{\includegraphics[width=0.3\textwidth,height=3.5cm,keepaspectratio]{obrazky/emded_arduino_micro}\label{emded_arduino_micro}}
			\hspace*{5mm}
			\subfigure[Arduino Nano]{\includegraphics[width=0.3\textwidth,height=3.5cm,keepaspectratio]{obrazky/emded_arduino_nano}\label{emded_arduino_nano}}
		\caption{Arduino Mini, Micro a Nano}
		\vspace{-30pt}	
\end{figure*}
	
		
		\subsection{Arduino Fio} 
		Arduino Fio je přizpůsobená k připojení různých bezdrátových modulů (například ZigBee nebo XBee moduly). Základem je procesor ATmega328P s frekvencí 8\,MHz. Napětí je zde kvůli kompatibilitě s moduly sníženo oproti většině ostatních desek z 5\,V na 3,3\,V~\cite{ArduinoFio}.	
		
	\subsection{Arduino MKR1000}
	Arduino MKR1000 je postavené na čipu ATSAMW25 od Atmelu, který v sobě spojuje ARMové jádro SAMD21 Cortex-M, Wi-Fi čip a šifrovací a autentizační čip ECC508. Tento čip nabízí ECDH (Diffie-Hellman s využitím eliptických křivek) a~ECDSA (Elliptic Curve Digital Signature Algorithm). Dále pak generátor náhodných čísel, unikátní 72bitové sériové číslo nebo SHA-256 s volitelným HMAC.
	
	
	\begin{figure*}[!ht]
    \centering
			\subfigure[Arduino Fio]{\includegraphics[width=0.33\textwidth,height=3.5cm,keepaspectratio]{obrazky/emded_arduino_fio}\label{emded_arduino_fio}}
			\hspace*{5mm}
			\subfigure[Arduino MKR1000]{\includegraphics[width=0.33\textwidth,height=3.5cm,keepaspectratio]{obrazky/emded_arduino_mkr1000}\label{emded_arduino_mkr1000}}
		\caption{Arduino Fio a MKR1000}
		\vspace{-30pt}	
\end{figure*}
	
		\subsection{Lilypad Arduino} 
		Lilypad Arduino je postaveno na ATmega168V (energeticky úsporná verze ATmega168) nebo ATmega328V. Je určeno pro wearables projekty, zejména pro implementaci do textilií, kdy jsou spoje vytvořeny pomocí vodivých nití. Není však pratelná. Existuje více variant této desky~\cite{ArduinoLilipad}.	
	
	\subsection{Arduino Yun} 
	Arduino Yun je deska založená na ATmega32u4 (architektura ARM) a Atheros AR9331 (architektiura x86), a umožňuje běh odlehčeného linuxu Linino. Obsahuje softwarový můstek zajišťující komunikaci obou čipů. Procesor Atheros podporuje linuxové distribuce založené na OpenWrt s názvem OpenWrt-Yun. Deska má vestavěný Ethernet a WiFi modul, USB-A port, slot pro MicroSD kartu. Dále disponuje 20 digitálními I/O piny, z toho 7 mohou být použito jako výstupy PWM a 12 jako analogové vstupy~\cite{ArduinoYun}.	

	\begin{figure*}[!ht]
    \centering
			\subfigure[Lilypad Arduino ]{\includegraphics[width=0.55\textwidth,height=3.5cm,keepaspectratio]{obrazky/emded_arduino_lilipad}\label{emded_arduino_lilipad}}
			\hspace*{5mm}
			\subfigure[Arduino Yun ]{\includegraphics[width=0.35\textwidth,height=3.5cm,keepaspectratio]{obrazky/emded_arduino_yun}\label{emded_arduino_yun}}
			\caption{Lilypad Arduino a Arduino Yun}
			\vspace{-10pt}	
\end{figure*}
	
%%%%%%%%%%%%%%%%%%%%%%%%%%%%%%%%%%%%%%%%%%%%%%%%%%%%%%%%%%%%%%%%%%%%%%%%%%%%%%%%%%%%%%%%%%%%%%%%%%%%%%%%%%%%%%%%%
%%%%%%%%%%%%%%%%%%%%%%%%%%%%%%%%%%%%%%%%%%%%%%%%%%%%%%%%%%%%%%%%%%%%%%%%%%%%%%%%%%%%%%%%%%%%%%%%%%%%%%%%%%%%%%%%%
%%%%%%%%%%%%%%%%%%%%%%%%%%%%%%%%%%%%%%%%%%%%%%%%%%%%%%%%%%%%%%%%%%%%%%%%%%%%%%%%%%%%%%%%%%%%%%%%%%%%%%%%%%%%%%%%%
%%%%%%%%%%%%%%%%%%%%%%%%%%%%%%%%%%%%%%%%%%%%%%%%%%%%%%%%%%%%%%%%%%%%%%%%%%%%%%%%%%%%%%%%%%%%%%%%%%%%%%%%%%%%%%%%%
%%%%%%%%%%%%%%%%%%%%%%%%%%%%%%%%%%%%%%%%%%%%%%%%%%%%%%%%%%%%%%%%%%%%%%%%%%%%%%%%%%%%%%%%%%%%%%%%%%%%%%%%%%%%%%%%%
%%%%%%%%%%%%%%%%%%%%%%%%%%%%%%%%%%%%%%%%%%%%%%%%%%%%%%%%%%%%%%%%%%%%%%%%%%%%%%%%%%%%%%%%%%%%%%%%%%%%%%%%%%%%%%%%%

\section{Arduino klony}
\label{KapArduinoKlony}

	Jelikož je projekt Arduino open-source, vzniklo množství klonů od dalších firem i jednotlivců. Klony jsou s původním Arduinem kompatibilní, ve většině případů konfigurací odpovídají některému z Arduino modelu, většinou Arduino UNO. Kity, které nemají shodné rozložení pinů neumožňují připojení Arduino shieldů. V této podkapitole je uveden krátký přehled těch nejznámějších. Rozsáhlý přehled kompatibilních klonů lze nalézt na oficiálních stránkách Arduina~\cite{ArduinoClonesWeb}.
	
	\subsection{Freeduino} 
	Freeduino je klon Arduina, vycházející z Arduino Duemilanove.
	
	\subsection{LABduino} 
	LABduino je český klon Arduina vytvořený z otevřené elektronické stavebnice MLAB.
	
	\subsection{Arduelo Libero}	
	Arduelo Libero je mírně vylepšený český free klon Arduino Duemilanove.
	
	\subsection{Bare Bones Board} 
	Bare Bones Board je kompatibilní deska, tvarově nepřipomínající žádný Arduino produkt. Kvůli rozložení pinů nepodporuje shieldy. Vyráběná a prodávaná jako kit firmou Modern Device Company.
	
	\subsection{Freaduino} 
	Freaduino je kompatibilní deska, tvarově shodná s Arduino UNO, vyráběná a prodávaná firmou ElecFreak jako kit The Freaduino Uno. Podporuje 3,3\,V logiku a~napájení. Má piny na připojení modulů (XBee). Napájecí piny zvládají zátěž až 2\,A.

	\begin{figure*}[!ht]
	\vspace{-10pt}	
    \centering
			\subfigure[Bare Bones Board]{\includegraphics[width=0.45\textwidth,height=3.5cm,keepaspectratio]{obrazky/embed_barebones}\label{emded_barebones}}
			\hspace*{5mm}
			\subfigure[Freaduino]{\includegraphics[width=0.45\textwidth,height=3.5cm,keepaspectratio]{obrazky/embed_freaduino}\label{emded_freaduino}}
					\caption{Bare Bones Board a Freaduino}
					\vspace{-20pt}	
	\end{figure*}	
	
	\subsection{Runtime} 
	Runtime je kompatibilní deska, tvarově nepřipomínající žádný Arduino produkt. Kvůli rozložení pinů nepodporující shieldy. Vyráběná a prodávaná jako kit firmou NKC Electronics.
	
	\subsection{Nanode} 
	Nanode je kompatibilní deska, tvarově nepřipomínající žádný Arduino produkt. Tvarově připomíná Arduino UNO, rozložení pinů je kompatibilní.
	
	\subsection{Seeeduino} 
	Seeeduino je kompatibilní deska, vzhledem připomínající Arduino UNO, parametricky shodná s Arduino Mega.
	
	\subsection{Teensy}
	Teensy je kompletní vývojový mikrokontrolérový systém na velmi malé desce bez~osazených pinů, který je schopen realizovat mnoho typů projektů. Softwarově je~kompatibilní s Arduinem, programuje se však pomocí doplňku do Arduino IDE nebo pomocí WinAVR~\cite{ArduinoTeensy}.
	
	\subsection{Diavolino} 
	Diavolino je free klon Arduina, vzhledově i parametricky podobný Arduino UNO, bez~vyvedených konektorů. Vyráběná a prodávané jako kit firmou Evil Mad Scientist.
		
	\subsection{Boarduino} 
	Boarduino je levnější klon Arduina Diecimila s piny pro zapojení rovnou do nepájivého pole.

\begin{figure*}[!ht]
    \centering
			\subfigure[Diavolino]{\includegraphics[width=0.45\textwidth,height=3.5cm,keepaspectratio]{obrazky/embed_diavolino}\label{emded_diavolino}}
			\hspace*{5mm}
			\subfigure[Boarduino]{\includegraphics[width=0.45\textwidth,height=3.5cm,keepaspectratio]{obrazky/embed_boarduino}\label{emded_boarduino}}
					\caption{Diavolino a Boarduino}
					\vspace{-30pt}	
	\end{figure*}	

%%%%%%%%%%%%%%%%%%%%%%%%%%%%%%%%%%%%%%%%%%%%%%%%%%%%%%%%%%%%%%%%%%%%%%%%%%%%%%%%%%%%%%%%%%%%%%%%%%%%%%%%%%%%%%%%%
%%%%%%%%%%%%%%%%%%%%%%%%%%%%%%%%%%%%%%%%%%%%%%%%%%%%%%%%%%%%%%%%%%%%%%%%%%%%%%%%%%%%%%%%%%%%%%%%%%%%%%%%%%%%%%%%%
%%%%%%%%%%%%%%%%%%%%%%%%%%%%%%%%%%%%%%%%%%%%%%%%%%%%%%%%%%%%%%%%%%%%%%%%%%%%%%%%%%%%%%%%%%%%%%%%%%%%%%%%%%%%%%%%%
%%%%%%%%%%%%%%%%%%%%%%%%%%%%%%%%%%%%%%%%%%%%%%%%%%%%%%%%%%%%%%%%%%%%%%%%%%%%%%%%%%%%%%%%%%%%%%%%%%%%%%%%%%%%%%%%%
%%%%%%%%%%%%%%%%%%%%%%%%%%%%%%%%%%%%%%%%%%%%%%%%%%%%%%%%%%%%%%%%%%%%%%%%%%%%%%%%%%%%%%%%%%%%%%%%%%%%%%%%%%%%%%%%%
%%%%%%%%%%%%%%%%%%%%%%%%%%%%%%%%%%%%%%%%%%%%%%%%%%%%%%%%%%%%%%%%%%%%%%%%%%%%%%%%%%%%%%%%%%%%%%%%%%%%%%%%%%%%%%%%%

\section{RaspberryPi}
\label{KapRaspi}

RaspberryPi reprezentuje jednodeskový počítač o velikosti zhruba platební karty. Byl vyvinut v roce 2012 s cílem podpořit výuku informatiky a seznámit studenty s~řízením různých zařízení přes počítač~\cite{Raspi}. 

Primárním operačním systémem je Linux, k dispozici je několik jeho distribucí, případně lze použít Windows 10 IoT Core. Na rozdíl do Arduina obsahuje RaspberryPi plnohodnotný operační systém, ARM mikrokontrolér, USB pro připojení myši a klávesnice, Ethernet konektor pro připojení sítě, grafický výstup HDMI (High-Definition Multimedia Interface) a kompozitní video, DSI (Display Serial Interface) pro připojení displeje, CSI (Camera Serial Interface) pro připojení kamery a čtečku pamětových karet, tedy působí spíše jako menší počítač, než vývojová platforma. 

Všechny další rozšiřující sběrnice (UART, I2C, SPI, PWM, digitální vstup a~výstup, analogový vstup) jsou vyvedeny do 26 nebo 40 pinového GPIO konektoru. Na~rozdíl od Arduina je možné RaspberryPi pomocí GPIO kontaktů použít nejen k ovládání různých zařízení, ale i~k~samotnému vývoji příslušných aplikací. Lze ho také použít jako multimediální přehrávač videa nebo hudby nebo i jen pro přístup k Internetu.

RaspberryPi stejně jako Arduino podporuje připojení rozšiřujících karet:

\begin{itemize}
	\item \textbf{Pi T-Cobbler} je pasivní elektronický přípravek, který se k RaspberryPi připojuje pomocí 40 žilového plochého kabelu a slouží k vyvedení pinů do vývojové desky breadboard. Zde na konektorové desce jsou již jednotlivé piny popsány.
\item \textbf{Gertboard} je rozšiřující deska autora Gerta Van Loo, který rozšiřuje I/O možnosti RaspberryPi. K ní se připojuje pomocí 40 žilového plochého kabelu a~rozšiřuje možnosti o 8/10/12-bitový dvoukanálový D/A převodník, 10-bitový dvoukanálový A/D převodník, obvody pro řízení motoru, předprogramovaný Atmel AVR ATmega 328P, 6 výstupů s otevřeným kolektorem a dalších 12 IO pinů~\cite{GertBoard}.  
\item \textbf{UniPi} je rozšiřující deska která rozšiřuje I/O možnosti RaspberryPi. K~ní~se~připojuje pomocí 26 žilového plochého kabelu a dle typu připojeného UniPi zařízení poskytuje I/O funkce navíc. Rozšiřujícími moduly UniPi se bude blíže zabývat následující kapitola (Kap.~\ref{KapUnipi}).
\item \textbf{RaspberryPi to Arduino Shield} je rozšiřující deska, která umožňuje propojení RasbperryPi a vybraných modelů Arduino.
\end{itemize}
Samozřejmě lze k Arduinu připojit i samotné moduly nebo senzory, přímým připojením na dané piny GPIO konektoru. Je však třeba dbát na to, že RaspberryPi pracuje s 3,3\,V logikou, zatímco například Arduino pracuje s 5\,V logikou. Popis GPIO konektoru včetně možností připojení je součástí Kap.~\ref{KapGPIO}.

	\subsection{RaspberryPi}
	Původní model RaspberryPi byl zveřejněn v únoru roku 2012. Obsahuje jednojádrový procesor o frekvenci 700\,MHz. U této verze existovaly tři modely~\cite{RaspiOne}:
	
		\begin{itemize}
		\item \textbf{Model A+} je odlehčená levná verze modelu B. Nemá žádný pamětový slot. Disponuje 256\,MB RAM. Neobsahuje USB port. Má 40 GPIO pinů.
		\item \textbf{Model B} byl původní RaspberryPi. Má slot na SD kartu. Disponuje 512\,MB RAM. Obsahuje 1 USB port. Má 26 GPIO pinů. Má samostatný výstup kompozitního videa.
		\item \textbf{Model B+} obsahuje slot na MicroSD kartu. Disponuje 512\,MB RAM. Obsahuje 2 USB porty. Má 40 GPIO pinů.
	\end{itemize}
	
\begin{figure*}[!ht]
    \centering
			\subfigure[Model A+]{\includegraphics[height=3.5cm,keepaspectratio]{obrazky/embed_raspi_1a}\label{embed_raspi_1a}}
			\hspace*{2mm}
			\subfigure[Model B]{\includegraphics[height=3.5cm,keepaspectratio]{obrazky/embed_raspi_1b}\label{embed_raspi_1b}}
			\hspace*{2mm}
			\subfigure[Model B+]{\includegraphics[height=3.5cm,keepaspectratio]{obrazky/embed_raspi_1bp}\label{embed_raspi_1bp}}
		\caption{RaspberryPi prvních verzí}
\end{figure*}
	
	\subsection{RaspberryPi 2}
	RaspberryPi 2 je pokračováním RaspberryPi, které přináší zejména vyšší výkon. Díky čtyřjádrovému procesoru BCM2836 o taktu 900\,MHz by měl být 3-6× rychlejší než jeho předchůdce. Tento model disponuje 1\,GB paměti a má 4 USB porty~\cite{RaspiTwo}.


\subsection{RaspberryPi 3}
		RaspberryPi 3, dostupný od roku 2016 je vybaven čtyřjádrovým 64bitovým procesorem ARM Cortex-A53 o taktu 1,2\,GHz. Oproti předchozímu modelu přináší integraci WiFi a Bluetooth modulů přímo na desce a měl by být dvakrát rychlejší~\cite{RaspiThree}.
		
\subsection{RaspberryPi Zero}
		RaspberryPi Zero je nejúspornější varianta RaspberryPi, ideální pro použití v IoT. Vychází z modelu A+, ve srovnání s ním nabízí procesor s frekvencí 1\,GHz a 512\,MB paměti. Má přibližně poloviční velikost, nemá však vyvedené piny GPIO konektoru, USB konektory má ve verzi micro a HDMI ve verzi mini~\cite{RaspiZero}.

\begin{figure*}[!ht]
    \centering
			\subfigure[RaspberryPi 2]{\includegraphics[width=0.3\textwidth,height=3.5cm,keepaspectratio]{obrazky/embed_raspi_2}\label{embed_raspi_2}}
			\hspace*{5mm}
			\subfigure[RaspberryPi 3]{\includegraphics[width=0.3\textwidth,height=3.5cm,keepaspectratio]{obrazky/embed_raspi_3}\label{embed_raspi_3}}
			\hspace*{5mm}
			\subfigure[RaspberryPi Zero]{\includegraphics[width=0.3\textwidth,height=3.5cm,keepaspectratio]{obrazky/embed_raspi_zero}\label{embed_raspi_zero}}
		\caption{RaspberryPi následujících verzí}
\end{figure*}


%%%%%%%%%%%%%%%%%%%%%%%%%%%%%%%%%%%%%%%%%%%%%%%%%%%%%%%%%%%%%%%%%%%%%%%%%%%%%%%%%%%%%%%%%%%%%%%%%%%%%%%%%%%%%%%%%
%%%%%%%%%%%%%%%%%%%%%%%%%%%%%%%%%%%%%%%%%%%%%%%%%%%%%%%%%%%%%%%%%%%%%%%%%%%%%%%%%%%%%%%%%%%%%%%%%%%%%%%%%%%%%%%%%
%%%%%%%%%%%%%%%%%%%%%%%%%%%%%%%%%%%%%%%%%%%%%%%%%%%%%%%%%%%%%%%%%%%%%%%%%%%%%%%%%%%%%%%%%%%%%%%%%%%%%%%%%%%%%%%%%
%%%%%%%%%%%%%%%%%%%%%%%%%%%%%%%%%%%%%%%%%%%%%%%%%%%%%%%%%%%%%%%%%%%%%%%%%%%%%%%%%%%%%%%%%%%%%%%%%%%%%%%%%%%%%%%%%
%%%%%%%%%%%%%%%%%%%%%%%%%%%%%%%%%%%%%%%%%%%%%%%%%%%%%%%%%%%%%%%%%%%%%%%%%%%%%%%%%%%%%%%%%%%%%%%%%%%%%%%%%%%%%%%%%
%%%%%%%%%%%%%%%%%%%%%%%%%%%%%%%%%%%%%%%%%%%%%%%%%%%%%%%%%%%%%%%%%%%%%%%%%%%%%%%%%%%%%%%%%%%%%%%%%%%%%%%%%%%%%%%%%

\section{RaspberryPi klony}
\label{KapRaspiKlony}

Vzrůstající popularita RaspberryPi dala stejně jak u Arduina vzniknout celé řadě klonů. Tyto klony odvozují ze základního sestavení RaspberryPi a určitým způsobem ho rozřišují. Jelikož označení „RaspberryPi“ je registrovanou ochrannou známkou, mají podobně navžené počítače odvozené názvy, jako BananaPi a~OrangePi. Zmíněné klony patří k nejznámějším a každý z nich již existuje v několika verzích, v této podkapitole budou představeny ty nejznámější s uvedením jejich hlavních odchylek od RaspberryPi.

%%% ======================================= BANANA PI ============================== %%%
	\subsection{BananaPi}
		Původní BananaPi, ze kterého vychází řada dalších modelů, je malý jednodeskový počítač, který se na první pohled podobá RaspberryPi.  Obsahuje dvoujádrový procesor a 512\,MB RAM. Na rozdíl od RaspberryPi obsahuje BananaPi také SATA řadič, mikrofon, který je připájen přímo na desce, gigabitový Ethernet, USB 2.0 OTG (On The Go), IR (Infrared Radiation) přijímač, tlačítko reset a power. Počítač podporuje SATA disky až~do~velikosti 2\,TB. GPIO konektor je vždy kompatibilní s některou verzi RaspberryPi. Za~výrobou všech počítačů BananaPi stojí čínská firma SinoVoip CO., Limited~\cite{BananaPi}.
		
	\textbf{BananaPi BPI-M2} je klon RaspberryPi 2, obsahuje taktéž čtyřjádrový procesor běžící na 1 GHz, má již integrovanou WiFi (Wireless Fidelity), ale neobsahuje SATA (Serial Advanced Technology Attachment) port.
	
	\textbf{BananaPi BPI-M3} obsahuje osmijádrový procesor 1,8\,GHz s 2\,GB RAM, dále zahrnuje WiFi b/g/n a integrované Bluetooth 4. Obsahuje SATA port.
	
	\textbf{BananaPi BPI-M64} obsahuje oproti modelu M3 čtyřjádrový 64 bitový SoC procesor Allwinner A64.
	
	%%% ======================================= ORANGE PI ============================== %%%
	\subsection{OrangePi}
	OrangePi je alternativa pro RaspberryPi vznikající v posledních dvou letech. Všechny modely jsou založeny na architektuře ARM Cortex-A7 s SoC Allwinner H3 s čtyřjádrovým CPU, výjimkou jsou OrangePi a OrangePi Mini, které mají SoC Allwinner A20 s dvojádrovým CPU. Grafickým čipem je u všech modelů ARM Mali-400 MP2. Všechny modely podporují HDMI CEC~\cite{OrangePi}.

		\textbf{OrangePi} je základní model z rodiny OrangePi, obsahuje čtyřjádrový procesor Allwinner A20 na 1\,GHz a 1\,GB RAM. Oproti RaspberryPi má navíc pouze mikrofon, IR port, USB OTG, ale nemá DSI rozhraní.
		
		\textbf{OrangePi Plus } má procesory běžící na 1,6\,GHz, 1\,GB RAM a 8\,GB EMMC Flash. Oproti RaspberryPi má gigabitový Ethernet, integrovaný mikrofon, USB-OTG konektor, integrovaný WiFi modul, IR přijímač. Obsahuje SATA port, který je připojený přes USB převodník.
		
		\textbf{OrangePi Plus2} oproti předchozí verzi došlo k navýšení pamětí na 2\,GB RAM a 16\,GB eMMC (embedded MultiMedia Card) Flash a doplnění CSI (Camera Serial Interface) konektoru.
		
		\textbf{OrangePi One} vznikla jako reakce na odlehčenou verzi RaspberryPi Zero. Jedná se o čtyřjádrový procesor na frekvenci 1,2\,GHz postavený na čipu ARM Cortex-A7 s grafickým čipem Mali400 MP2. Operační paměť je 512\,MB. K dispozici je pouze 10/100\,Mbps Ethernet a jeden port USB 2.0.

	\begin{figure*}[!ht]
	\vspace{-10pt}
    \centering
			\subfigure[BananaPi BPI-M2]{\includegraphics[height=4cm]{obrazky/embed_bananapim2}\label{embed_bananapim2}}
			\hspace*{5mm}
			\subfigure[OrangePi Plus2]{\includegraphics[height=4cm]{obrazky/embed_orangepiplus2}\label{embed_orangepiplus2}}
			\caption{BananaPi BPI-M2 a OrangePi Plus2}
			\vspace{-10pt}	
\end{figure*}
	
	

%%% ======================================= CUBIE BOARD ============================== %%%
	\subsection{CubieBoard}
	\label{KapCubie}
			CubieBoard je alternativou k RaspberryPi z roku 2012. Ačkoliv jsou vzhledově i~parametricky velmi podobné, není Cubieboard s RaspberryPi kompatibilní. Jsou postaveny na AllWinner A10 SoC čipu. Výrobce poskytuje vlastní sadu modulů a rozšiřujících desek. Cubieboardy poskytují pinové rozhraní, obsahující základní sběrnice (I2C, SPI, UART) ale i rozšiřující jako LVDS (Low-Voltage Differential Signaling). Desky obsahují navíc SATA konektor~\cite{CubieBoards}.
	
	\textbf{CubieBoard 1} je výkonná nízkopříkonová deska s ARM A8 o taktu 1\,GHz s~1\,GB RAM, 4\,GB NAND flash a Mali400 GPU. Obsahuje LAN port a dvojici USB portů. Deska má 96 pinů, které zahrnují sběrnice GPIO, I2C, UART, LVDS (Low Resolution Analog to Digital Converter), PWM, SPI, CSI, VGA a jiné. Dále obsahuje 100Mbps Ethernet a dva USB HOST porty, mini USB OTG, čtečku micro SD, HDMI, IR, line in, line out a SATA port.
	
	\textbf{CubieBoard 2} představuje nástupce CubieBoardu1, je s ním zpětně kompatibilní a~od~předchozí verze se liší pouze dvoujádrovým provedením CPU a GPU.
		
	\textbf{CubieBoard 3} oproti předchozím verzím přínaší vylepšení jako 2\,GB RAM, 8\,GB NAND flash, VGA konektor přímo na desce, gigabitový Ethernet, WiFi a~Bluetooth integrované přímo na desce. Pinové rozhranní je zde redukováno na 54 pinů obsahující I2S (Inter-Integrated Sound), I2C, SPI, CVBS (Color Video Blanc Sync), UART, PWM a GPIO.
	
		\textbf{CubieBoard 4} je nástupce CubieBoardu 3, je zpětně kompatibilní a oproti předchůdci přináší čtyřjádrové CPU ARM A15x a GPU PowerVR G6230. Dále má microUSB 3.0 OTG a audio konektory umístěné přímo na desce.
		
			\textbf{CubieBoard 5 }nabízí osmijádrový procesor Allwinner H8, který doplňují 2\,GB RAM. Navíc oproti předchozím verzím má kromě HDMI i DP (Display Port), přináší taktéž konektor pro připojení externí baterie. Došlo k navýšení GPIO pinů na 70, které navíc přináší LRADC (Low Resolution Analog to Digital Converter) a PS2 (Personal System/2). SATA konektor pomocí speciální desky podporuje připojení dvou SATA disků s podporou RAIDu.
		
CubieBoardy již poskytují dostatečný výkon pro embeeded zařízení, přináší oproti RaspberryPi mnoho rozšiřujících sběrnic, avšak pro nedostatečnou podporu či zastoupení v evropských zemí a velmi častou nedostupnost webu výrobce, včetně dostupnosti anglické dokumentace pro  programování jednotlivých rozhraní, není moc vhodná pro IoT. Hodí se spíše pro aplikace jako multimediální centrum či nízkonákladový počítač. 	

%%% ======================================= UP BOARD ============================ %%%

\subsection{UpBoard}
		UpBoard představuje miniaturní jednodeskový počítač na platformě Intel s čtyřjádrovým procesorem Intel Atom. Vzhledově je velice podpobný RaspberryPi 3. 
		Tento počítač obsahuje čtyřjádrový procesor Intel Atom x5-Z8300 na frekvenci 1,84 GHz s~TDP 2\,W. Obsahuje 1\,GB RAM a 16\,GB flash eMMC (Embedded MultiMedia Card). 40 pinové rozhraní je totožné jako u RaspberryPi 2 s níž je částečně kompatibilní. Navíc obsahuje gigabitový Ethernet port, 5 USB 2.0 a jedno USB 3.0. Čip má hardwarovou podporu šifrování AES (Advanced Encryption Standard), je tedy vhodný pro IoT projekty s vyšším zabezpečením. Podporuje Android 5.0, Linux či Windows 10 IoT Core. Dokumentace pro programování GPIO v současnosti neexistuje, dokumentaci tvoří pouze popis GPIO konektoru~\cite{UpBoard}.

	\begin{figure*}[!ht]
	\vspace{-10pt}	
    \centering
			\subfigure[CubieBoard1]{\includegraphics[height=4cm]{obrazky/embed_cubie_1}\label{embed_cubie_1}}
			\hspace*{5mm}
			\subfigure[UpBoard1]{\includegraphics[height=4cm]{obrazky/embed_upboard}\label{embed_upboard}}
			\caption{CubieBoard1 a UpBoard1}
			\vspace{-10pt}	
\end{figure*}
		

%%% ======================================= PINE64 ============================== %%%
	\newpage{}
	\subsection{PINE64}
Pine64 je rodina tří jednodeskových počítačů společnosti PINE64. Tyto počítače byly navrženy tak, aby konkurovaly RaspberryPi ve výkonu a ceně. Všechny verze obsahují 64bitový čtyřjádrový procesor 1,152\,GHz Cortex-A53 a liší se pouze velikostí operační paměti a použitelným operačním systémem. Oproti RaspberryPi obsahují gigabitový Ethernet, WiFi, Bluetooth a port pro připojení dotykového panelu. Mají GPIO konektor shodný s danou verzi Raspberry, jsou s ní tedy do jisté míry kompatibilní. Zvlášností těchto desek je Eulerova sběrnice, která navyšuje počty sběrnic SPI, UART, GPIO~\cite{Pine64}.

\textbf{PINE A64 512MB} má 512\,MB paměti a podporuje pouze Arch Linux a Debian Linux.

\textbf{PINE A64+ 1GB} má 1\,GB paměti a podporuje i Android, Remix OS, Ubuntu a Windows IoT.

\textbf{PINE A64+ 2GB} má oproti předchozí verzi 2\,GB operační paměti.

	\begin{figure*}[!ht]
    \centering
			\subfigure[PINE A64+ 2GB]{\includegraphics[height=4cm]{obrazky/embed_pine}\label{embed_pine}}
			\hspace*{5mm}
			\subfigure[HardKernel Odroid-C2]{\includegraphics[height=4cm]{obrazky/embed_odroidc2}\label{embed_odroidc2}}
			\caption{PINE A64+ 2GB a HardKernel Odroid-C2}
			\vspace{-10pt}
\end{figure*}

%%% ======================================= ODROID ============================== %%%
\newpage{}
\subsection{HardKernel Odroid}
	\label{KapKernel}
ODROID je řada jednodeskových počítačů od společnosti HardKernel. Název je odvozen z \textbf{O}pen An\textbf{droid}, ale podporovány jsou i linuxové distribuce. Desky disponují 40 pinovým GPIO kompatibilním s RaspberryPi, ale open-source již nejsou. Desky jsou postaveny na SoC platformě Samsung Exynos. Zvláštností desek je sériové rozhraní s 1,8\,V~\cite{HardKernel}.
	
	\textbf{ODROID-C1} je reakcí na RaspberryPi 1. Nabízí čtyřjádrové SoC Cortex A5 s frekvencí 1,5\,GHz a 1\,GB RAM. Dále má gigabitový Ethernet a připojení flash úložiště typu eMMC. 

	\textbf{ODROID-C2} je reakcí na RaspberryPi 3. Obsahuje čtyřjádrový 64bitový procesor ARMv8 taktovaný na 2\,GHz, 2\,GB paměti a gigabitový Ethernet. Má podporu sběrnice I2S. Hlavní změnou je podpora HDMI 2.0 a schopnost přehrávat 4K video ve formátu H.265. Podporuje Ubuntu 16.04 nebo Android 5.1. 
	
	\textbf{ODROID-XU4} je výkonejší řada desek, obsahují čtyřjádrový procesor Samsung Exynos5 ARM Cortex-A15 na frekvenci 2\,GHz a čtyřjádrový procesor Cortex-A7 Quad 1,3\,GHz, bohužel vzhledem k výkonu je zde již aktivní chlazení. Deska disponuje grafickým čipem Mali-T628 MP6 a 2\,GB RAM paměti.

	
%%% ======================================= BeagleBoard ============================== %%%
\subsection{BeagleBoard }
BeagleBoard je skupina jednodeskových počítačů produkovaných společností Texas Instruments, navržených na čipu Texas Instrument's OMAP3530 SoC, ten obsahuje ARM Cortex-A8 CPU, který může provozovat Linuxové distribuce, BSD nebo Android. Desky obsahují dva 46pinové GPIO konektory, oproti ostatním přináší podporu CAN (Controller Area Network) sběrnice. Výrobce poskytuje vlastní řadu kompatibilních rozšiřujicích desek, nazýva je \uv{capes} a současně lze připojit až~4~takovéto desky. Výhodou desek je jejich nízká spotřeba, využívají maxikmálně 2\,W elektrické energie a mohou být napájeny i ze samostatného napájení. Vzhledem k nízké spotřebě energie nejsou nutné žádné přídavné chladiče~\cite{BeagleBone}.

\textbf{BeagleBoard} obsahuje procesor Sitara ARM Cortex-A8 na frekvenci 720\,MHz a disponuje dle revize 128 nebo 256\,MB RAM. Obsahuje 256\,MB NAND paměti.

\textbf{BeagleBone} obsahuje procesor Sitara ARM Cortex-A8 na frekvenci 720\,MHz a~disponuje 256\,MB RAM.

\textbf{BeagleBoard-X15} je založen na šestiádrovém procesoru Sitara AM5728 s dvěma jádry ARM Cortex-A15 na frekvenci 1,5\,GHz a dvěma jádry ARM Cortex-M4 na frekvenci 212\,MHz a dvěma jádry TI C66x DSP na frekvenci 700\,MHz. Disponuje 2\,GB RAM. Použitý procesor přináší podporu HDMI 2.1, gigabitového Ethernetu a grafického dvoujádrového čipu SGX544 na frekvenci 532\,MHz. 

	\begin{figure}[!ht]
  \begin{center}
    \includegraphics[height=4cm]{obrazky/embed_beaglebone_black}
  \end{center}
	\vspace{-20pt}
  \caption{BeagleBone Black~\cite{BeagleBone}}
\end{figure}

\textbf{BeagleBone Black} má oproti předchůdci zvýšenou pamět na 512\,MB, frekvenci procesoru na 1\,GHz a 2\,GB eMMC flash paměti.

%%%%%%%%%%%%%%%%%%%%%%%%%%%%%%%%%%%%%%%%%%%%%%%%%%%%%%%%%%%%%%%%%%%%%%%%%%%%%%%%%%%%%%%%%%%%%%%%%%%%%%%%%%%%%%%%%
%%%%%%%%%%%%%%%%%%%%%%%%%%%%%%%%%%%%%%%%%%%%%%%%%%%%%%%%%%%%%%%%%%%%%%%%%%%%%%%%%%%%%%%%%%%%%%%%%%%%%%%%%%%%%%%%%
%%%%%%%%%%%%%%%%%%%%%%%%%%%%%%%%%%%%%%%%%%%%%%%%%%%%%%%%%%%%%%%%%%%%%%%%%%%%%%%%%%%%%%%%%%%%%%%%%%%%%%%%%%%%%%%%%
%%%%%%%%%%%%%%%%%%%%%%%%%%%%%%%%%%%%%%%%%%%%%%%%%%%%%%%%%%%%%%%%%%%%%%%%%%%%%%%%%%%%%%%%%%%%%%%%%%%%%%%%%%%%%%%%%
%%%%%%%%%%%%%%%%%%%%%%%%%%%%%%%%%%%%%%%%%%%%%%%%%%%%%%%%%%%%%%%%%%%%%%%%%%%%%%%%%%%%%%%%%%%%%%%%%%%%%%%%%%%%%%%%%
%%%%%%%%%%%%%%%%%%%%%%%%%%%%%%%%%%%%%%%%%%%%%%%%%%%%%%%%%%%%%%%%%%%%%%%%%%%%%%%%%%%%%%%%%%%%%%%%%%%%%%%%%%%%%%%%%

\section{Intel}
\label{KapIntel}

Společnosti Intel přináší dva jednodeskové počítače založené na platformě mikroprocesoru x86. Jsou navrženy jak pro vývojáře tak k výuce výpočetní techniky.

		\subsection{Intel Galileo}
		Intel Galileo je jednočipový počítač, vyvinutý společností Intel, postavený na architektuře x86. Obsahuje procesor Intel Quark x86 na frekvenci 400\,MHz. Má 256\,MB RAM. Byl navržen pro výuku výpočetní techniky. Jedná se o první zařízení od~Intelu, které je hardwarově i softwarově kompatibilní s Arduinem. Lze k němu připojovat Arduino shieldy i moduly a využívat vývojové prostředí Arduina, včetně jeho knihoven. 
		Tento počítač obsahuje 14 digiálních I/O pinů, z toho 6 z nich lze využít jako PWM výstupy. Dále obsahuje 6 analogových vstupů, UART sběrnici, I2C sběrnici, SPI sběrnici, Ethernet konektor, slot na MicroSD kartu. Dále obsahuje 2~USB konektory, jeden USB-host, druhý USB-klient. Druhá generace této desky pak přináší podporu PoE (Power over Ethernet)a další drobné změny~\cite{IntelGalileo,ArduinoGalileo}.
\begin{figure}[!ht]
  \begin{center}
    \includegraphics[height=5cm]{obrazky/embed_intel_galileo}
  \end{center}
	\vspace{-20pt}
  \caption{Intel Galileo~\cite{IntelGalileo}}
	\vspace{-10pt}
\end{figure}
		
		
		\subsection{Intel Edison} 
		Intel Edison je druhý jednočipový počítač architektury x86 vyvinutý společností Intel. Má velikost SD karty a je určený pro nositelnou elektroniku. Obsahuje dvoujádrový procesor Intel Quark x86 na frekvenci 400 MHz. Dále obsahuje 1\,GB RAM a 4\,GB flash paměti. Konektivita je zajištěna pomocí 70 pinového Hirose DF40 konektoru, který v sobě sdružuje veškerá dostupná rozhraní (USB, GPIO, SPI, I2C a~PWM). Jsou k dispozici dvě rozšiřující desky~\cite{IntelEdison}:
			
			\begin{itemize}
				\item Arduino board - Arduino board je plně kompatibilní s Arduinem, včetně podpory Arduino shieldů a modulů. Dále tato deska zpřístupňuje 20 digitálních I/O pinů, z toho 4 z nich lze využít jako PWM výstupy. Dále obsahuje 6 analogových vstupů, UART sběrnici, I2C sběrnici, SPI sběrnici. Dále obsahuje 2~USB konektory, jeden pro napájení, druhý připojený k UART sběrnici a slot na SD kartu.
				
				\begin{figure}[!ht]
  \begin{center}
    \includegraphics[height=4.5cm]{obrazky/embed_intel_edison2}
  \end{center}
	\vspace{-20pt}
  \caption{Arduino board pro Intel Edison}
	\label{embed_intel_edison2}
	\vspace{-10pt}
\end{figure}
				
				\item Intel breakout board -  Tato deska je díky svým malým rozměrům vhodná pro~prototypování nositelné elektroniky či pro Internet věcí. Obsahuje pájitelnou mřížku pro zpřístupnění věškerých dostupých rozhraní. Na desku jsou vyvedeny pouze dva USB konektory, jeden pro napájení a druhý připojený k UART sběrnici.
\end{itemize}

\begin{figure}[!ht]
  \begin{center}
    \includegraphics[height=2.5cm]{obrazky/embed_intel_edison1}
  \end{center}
	\vspace{-20pt}
  \caption{Intel breakout board}
	\label{embed_intel_edison1}
	\vspace{-10pt}
\end{figure}

Desky Intel se hodí spíše pro větší typy projektů, kdy již vývojové prostředí Arduina nestačí a je potřeba využít plného potenciálu operačního systému.

%%%%%%%%%%%%%%%%%%%%%%%%%%%%%%%%%%%%%%%%%%%%%%%%%%%%%%%%%%%%%%%%%%%%%%%%%%%%%%%%%%%%%%%%%%%%%%%%%%%%%%%%%%%%%%%%%
%%%%%%%%%%%%%%%%%%%%%%%%%%%%%%%%%%%%%%%%%%%%%%%%%%%%%%%%%%%%%%%%%%%%%%%%%%%%%%%%%%%%%%%%%%%%%%%%%%%%%%%%%%%%%%%%%
%%%%%%%%%%%%%%%%%%%%%%%%%%%%%%%%%%%%%%%%%%%%%%%%%%%%%%%%%%%%%%%%%%%%%%%%%%%%%%%%%%%%%%%%%%%%%%%%%%%%%%%%%%%%%%%%%
%%%%%%%%%%%%%%%%%%%%%%%%%%%%%%%%%%%%%%%%%%%%%%%%%%%%%%%%%%%%%%%%%%%%%%%%%%%%%%%%%%%%%%%%%%%%%%%%%%%%%%%%%%%%%%%%%
%%%%%%%%%%%%%%%%%%%%%%%%%%%%%%%%%%%%%%%%%%%%%%%%%%%%%%%%%%%%%%%%%%%%%%%%%%%%%%%%%%%%%%%%%%%%%%%%%%%%%%%%%%%%%%%%%
%%%%%%%%%%%%%%%%%%%%%%%%%%%%%%%%%%%%%%%%%%%%%%%%%%%%%%%%%%%%%%%%%%%%%%%%%%%%%%%%%%%%%%%%%%%%%%%%%%%%%%%%%%%%%%%%%

 \section{AMD Gizmo}
\label{KapAMD}

	Gizmo Board a Gizmo Board 2 od firmy AMD jsou alternativou k počítačům RaspberryPi, nabízející však platformu IBM PC a 64bitovou architekturu. Umožňuje tedy běh klasických operačních systémů, včetně Windows.
	
		\textbf{Gizmo Board 1} beží na dvoujádrovém APU G-T40E od firmy AMD na frekvenci 1GHz při příkonu 10\,W. Součástí procesoru je grafický čip Radeon HD 6250. K dispozici je 1\,GB RAM. Deska dále obsahuje dvojici USB, VGA, audio výstup, SATA a Ethernet konektor. Další sběrnice jako GPIO, SPI, I2C, UART a~PWM jsou dostupné po připojení rozšiřující karty přes LowSpeed~\cite{AmdGizmo1}.
	
	\textbf{ Gizmo Board 2} je vybaven APU AMD GX210HA na~frekvenci 1 GHz, s~integrovaným GPU AMD Radeon HD 8210E s frekvencí 300\,MHz. Příkon je 9\,W. Tento model má také 1\,GB RAM. Tato verze disponuje 4 USB, HDMI výstupem, MicroSD slotem a Ethernetovým portem. Mezi další rozhraní patří PCI Express (Peripheral Component Interconnect Express), GPIO, SPI, I2C, UART, DAC/ADC (Digital to Analog Converter/Analog to Digital Converter) nebo PWM \cite{AmdGizmo2}.


\begin{figure*}[!ht]
    \centering
			\subfigure[Gizmo Board 1]{\includegraphics[width=0.45\textwidth]{obrazky/embed_amd_gizmo1}\label{embed_amd_gizmo1}}
			\hspace*{5mm}
			\subfigure[Gizmo Board 2]{\includegraphics[width=0.45\textwidth]{obrazky/embed_amd_gizmo2}\label{embed_amd_gizmo2}}
		\caption{AMD Gizmo}
\end{figure*}



	
Oba počítače již poskytují dostatečný výkon pro embedded zařízení, avšak druhá verze zařízení již využívá aktivní chlazení a je hlučnější. Obě zařízení jsou větších rozměrů a nemají dostatečnou dokumentaci k přístupu a programování jednotlivých rozhraní. Hodí se spíše pro~aplikace jako multimediální centrum či jednodušší počítač, než pro IoT nebo~průmyslovou automatizaci. Komunita okolo AMD Gizmo prakticky neexistuje.
	
	





%% Vložení kapitoly o UniPi
\chapter{Rozšiřující deska UniPi}
\label{KapUnipi}

UniPi je česká firma, nyní dceřiná společnost Faster.cz, původně její oddělení měření a regulace, která se zaměřuje na inteligentní stavební řešení, domácí automatizaci a~Internet věcí. Dále provozuje výzkum a vývoj rozšiřujicích desek UniPi, včetně jejich softwarového vybavení \cite{UniPi}.

UniPi je taktéž pojmenování pro přídavné rozšiřující desky pro RaspberryPi, se~kterou je plně kompatibilní ve všech verzích. Je vybavena řadou komponent, jako jsou například digitální galvanicky oddělené vstupy s LED signalizací, 0\,-\,10\,V analogové vstupy, 0\,-\,10\,V analogové výstupy, spínací relé, jednokanálová 1Wire sběrnice, I2C sběrnice, UART sběrnice, SPI sběrnice a RS-485 sběrnice.

UniPi je název, odvozený od slov \uv{RaspberryPi} a \uv{univerzální}, protože jednoduchost a univerzálnost jsou základní charakteristiky této desky. Deska původně vznikla pro potřeby řízení energetických hodnot vlastního datacentra Zelená Data \cite{ZelenaData}, ale škála odvětví, kde je možné UniPi nasadit je rozsáhlá, pro představu výrobce uvádí několik příkladů \cite{UniPi}:
\begin{itemize}
	\item Docházkové a přístupové systémy.
	\item Bezpečnostní systémy.
	\item Topné, chladící prvky i řízení.
	\item Větrání, rekuperace.
	\item Řídící systémy, které nejsou kompletní.
	\item Dohledové systémy.
	\item Ovládání světelných prvků.
	\item Datové vypínače.
	\item Řízení pivovarnických technologií.
	\item Zavlažovací systémy.
	\item Wellness systémy – vířivé vany, bazény, sauny.
	\item Solární systémy.
\end{itemize}

\vspace{5mm}
\noindent
 V současnosti existují dvě verze rozšiřující desky UniPi:


\begin{itemize}
	\item UniPi (verze 1)\footnote[1]{Dostupné z: \href{http://unipi.technology/product/unipi/}{http://unipi.technology/product/unipi/}}.
	\item UniPi Neuron (verze 2)\footnote[2]{Dostupné z: \href{http://unipi.technology/product/unipi-neuron-s103/}{http://unipi.technology/product/unipi-neuron-s103/}}.
\end{itemize}

\vspace{5mm}
Desky se liší svými vstupně-výstupními možnostmi, rozměry a jsou dostupné v několika variantách.



%%%%%%%%%%%%%%%%%%%%%%%%%%%%%%%%%%%%%%%
%%%%%%%%%%%%%%%%%%%%%%%%%%%%%%%%%%%%%%%
%%%%%%%%%%%%%%%%%%%%%%%%%%%%%%%%%%%%%%%

\section{UniPi v1}
\label{KapitolaUnipi1}

Deska UniPi je prezentována jako nejlevnější a nejjednodušší řešení pro inteligentní budovy a IoT. Je navržena pro maximální kompatibilitu s embedded zařízením RaspberryPi, který se dnes dá pořídit již za několik stokorun.  Zařízení bylo vyvinuto primárně jako rozhraní pro příjem vstupních signálů, jejich vyhodnocení a~realizaci výstupní reakce na základě naprogramovaných algoritmů \cite{UniPiBoard1}.

Disponuje (viz Obr.~\ref{ObrazekUnipiV1}) osmi relé pro střídavý proud, čtrnácti digitálními vstupy, jedním jednokanálovým 1Wire rozhraním, dvěma 0\,-\,10\,V analogovými vstupy a jedním 0\,-\,10\,V analogovým výstupem. Zajímavou součástí desky je také modul reálného času. Druhý I2C port na RaspberryPi v sobě navíc ukrývá 5\,V měnič napětí a ochranu ESD (ElectroStatic Discharge), umožňující tak připojení dalších zařízení. Pro jednoduché připojení jednotlivých sběrnic jsou na desce umístěny konektory.

 \begin{figure}[!ht]
  \begin{center}
    \includegraphics[scale=0.75]{obrazky/unipi_v1}
  \end{center}
	\vspace{-20pt}
  \caption{Popis UniPi v1 \cite{UniPiBoard1}}
	\label{ObrazekUnipiV1}
	\vspace{-30pt}
\end{figure}

\subsubsection{Popis desky}
\begin{itemize}
	\item 14 digitálních vstupů 5\,–\,24\,V. 
	\item 1Wire sběrnice pro měření teploty a vlhkosti. 
	\item 8 přepínacích relé 250\,V/5\,A AC nebo 24\,V/5\,A DC.
	\item 1 Analogový výstup 0\,–\,10\,V.
	\item 2 Analogové vstupy 0\,–\,10\,V.
	\item Modul reálného času.
	\item I2C sběrnice.
	\item EEPROM paměť.
	\item UART sběrnice.
	\item Notifikační diody pro zobrazení stavu jednotlivých portů.
\end{itemize}

\vspace{10pt}
Velkou výhodou řídicí jednotky UniPi je zabudovaný čip pro obsluhu teplotních čidel na sběrnici 1Wire. Digitální teploměry mají svou adresu, není tedy nutné je~jakkoliv kalibrovat či nastavovat, stačí zapojit.

S RaspberryPi je deska UniPi propojena 26\,žilovým kabelem přes GPIO konektor. Jak bylo popsáno v kapitole o konektoru GPIO, toto propojení je z důvodu kompatibility shodné pro všechny verze RaspberryPi. Vnitřní uspořádání desky je~řešeno pomocí funkčních celků  (znázorněno na Obr.~\ref{SchemaBlok1}), které jsou propojeny pomocí I2C sběrnice. Výstupy jednotlivých celků jsou poté vyvedeny na konektory desky.
 
\begin{figure}[!ht]
\vspace{-20pt}
  \begin{center}
    \includegraphics[scale=0.85]{obrazky/unipi_schema1}
  \end{center}
	\vspace{-30pt}
  \caption{Blokové schéma UniPi v1}
	\label{SchemaBlok1}
	
\end{figure}

Napájení desky je řízeno jumperem JP1 a může být řešeno dvěma způsoby:
\begin{itemize}
	\item Adaptérem 5\,V/2\,A do UniPi, s distribucí 5\,V/750\,mA do RaspberryPi.
	\item Samostatným nápajením obou desek.
\end{itemize}

\vspace{10pt}
Pro účely testování a implementace byla zapůjčena deska UniPi s počítačem RaspberryPi 2. Vývoj této desky byl již ukončen a nahrazen druhou verzí, označovanou jako UniPi NEURON.



%%%%%%%%%%%%%%%%%%%%%%%%%%%%%%%%%%%%%%%
%%%%%%%%%%%%%%%%%%%%%%%%%%%%%%%%%%%%%%%
%%%%%%%%%%%%%%%%%%%%%%%%%%%%%%%%%%%%%%%

\section{UniPi v2 - Neuron}
\label{KapitolaUnipi2}

UniPi Neuron představuje modulární PLC (Programmable Logic Controller) pro~chytrou domácnost a inteligentní systémy budov, řízení a průmyslovou automatizaci. Díky modulární a kompaktní konstrukci nabízí jedinečnou variabilitu funkcí. UniPi Neuron je univerzální řídící jednotka. Neuron lze použít k řízení chytrého domu nebo jako domácí server. Je vhodný pro monitorování, sběr a ukládání dat na vzdálený server, nebo jako výkonná a plně vybavená brána pro ostatní zařízení \cite{UniPiBoard2}.

 \begin{figure}[!ht]
  \vspace{-30pt}
  \begin{center}
    \includegraphics[scale=0.25]{obrazky/unipi_unipi_deska}
  \end{center}
	\vspace{-20pt}
  \caption{UniPi rozšiřující deska}
	\label{UnipiV2DeskaUnipi}
	\vspace{-10pt}
\end{figure}

UniPi Neuron je na rozdíl od první verze, kdy se rozšiřujicí deska RaspberryPi distribuovala zvlášt, již hotové řešení, které se skládá z RaspberryPi, rozšiřující desky UniPi verze 2 (viz Obr.~\ref{UnipiV2DeskaUnipi}), propojovací desky pro komunikační moduly (viz Obr.~\ref{unipi_osazeny}) a diodového panelu. To vše propojené a uzavřené v modrém plechovém pouzdru s možností montáže na DIN lištu. K dostání je tedy pouze jako hotový výrobek.


\subsubsection{Popis desky}
\begin{itemize}
\item Digitalní vstup 4\,-\,24\,V (počet závislý na konkrétním modelu).
\item Tranzistorový výstup 50V/750 mA (počet závislý na konkrétním modelu).
\item Analogový výstup 0\,-\,10\,V.
\item Analogový vstup 0\,-\,10\,V.
\item 1Wire sběrnice.
\item RS-485 .
\item Modul reálného času.
\item Notifikační diody pro zobrazení stavu jednotlivých portů.
\end{itemize}

\vspace{10pt}
UniPi Neuron existuje v několika verzích (viz Obr.~\ref{UnipiNeuronModels}), rozlišených počtem digitálních vstupů a výstupů, parametrů procesoru a velikosti paměti RAM. Do budoucna by měly být také k dostání verze s jedním konkrétním modulem (Wireless M-Bus, ZigBee, GPRS, \ldots) uvnitř.

\begin{figure*}[!ht]
    \centering
			\subfigure[S103]{\includegraphics[width=0.3\textwidth]{obrazky/unipi_neuron_s}\label{unipi_neuron_s}}
			\hspace*{5mm}
			\subfigure[M203]{\includegraphics[width=0.3\textwidth]{obrazky/unipi_neuron_m}\label{unipi_neuron_m}}
			\hspace*{5mm}
			\subfigure[L303]{\includegraphics[width=0.3\textwidth]{obrazky/unipi_neuron_l}\label{unipi_neuron_l}}
			\caption{Unipi Neuron~\cite{UniPiBoard2}}
			\label{UnipiNeuronModels}
\end{figure*}

\vspace{10pt}
Standardní modely NEURON mají proměnlivé množství digitálních vstupů a~reléových výstupů. Jejich počet je uveden v Tab.~\ref{TableUnipiIO}.

\begin{table}[!ht]
\caption{Porovnání modelů UniPi NEURON dle I/O \cite{UniPiBoard2}}
\vspace{-20pt}
\label{TableUnipiIO}
	\begin{center}
	\resizebox{\textwidth}{!}{%
	\begin{tabular}{|l|c|c|c|}
		\hline
		\textbf{Model} & \multicolumn{1}{l|}{\textbf{\begin{tabular}[c]{@{}l@{}}Počet digitálních vstupů\end{tabular}}} & \multicolumn{1}{l|}{\textbf{\begin{tabular}[c]{@{}l@{}}Počet digitálních výstupů\end{tabular}}} & \multicolumn{1}{l|}{\textbf{\begin{tabular}[c]{@{}l@{}}Velikost na DIN liště\end{tabular}}} \\ \hline \hline
		\textbf{S10x} & 4 & 0 & 4 moduly \\ \hline
		\textbf{M10x} & 12 & 8 & 8 modulů \\ \hline
		\textbf{M20x} & 20 & 14 & 8 modulů \\ \hline
		\textbf{M30x} & 34 & 0 & 8 modulů \\ \hline
		\textbf{M40x} & 4 & 28 & 8 modulů \\ \hline
		\textbf{L20x} & 36 & 28 & 12 modulů \\ \hline
		\textbf{L30x} & 64 & 0 & 12 modulů \\ \hline
		\textbf{L40x} & 4 & 56 & 12 modulů \\ \hline \hline
	\end{tabular}}
	\end{center}
\end{table}

\newpage
Písmeno x v Tab.~\ref{TableUnipiVar} bývá nahrazeno číslem 1-3 dle osazeného typu RaspberryPi:

\begin{table}[!ht]
\caption{Varianty modelů UniPi NEURON dle CPU a RAM \cite{UniPiBoard2}}
\vspace{-10pt}
\label{TableUnipiVar}
	\begin{center}
\begin{tabular}{|c|c|c|c|c|}
\hline
x & \textbf{Osazená deska} & \textbf{CPU} & \textbf{RAM} & \textbf{Další vlastnosti} \\ \hline \hline
\textbf{1} & RaspberryPi B+ & 700\,MHz & 512\,MB &  \\ \hline
\textbf{2} & RaspberryPi 2 & 4x900\,MHz & 1\,GB &  \\ \hline
\textbf{3} & RaspberryPi 3 & 4x1200\,MHz & 1\,GB & BT 4.1, WiFi 802.11n \\ \hline \hline
\end{tabular}
	\end{center}
\end{table}

\newpage

S RaspberryPi je deska UniPi propojena, obdobně jako u první verze, pomocí 26 pinové desky propoující GPIO port na RaspberryPi s konektorem na rozšiřující desce UniPi. Na samotné propojující desce (viz Obr.~\ref{unipi_propojbrain}) je vyvedena UART a~I2C sběrnice. 

\begin{figure*}[!ht]
		\vspace{-10pt}
    \centering
			\subfigure[Propojení desek]{\includegraphics[width=0.45\textwidth]{obrazky/unipi_propojbrain}\label{unipi_propojbrain}}
			\hspace*{5mm}
			\subfigure[Osazená UniPi deska]{\includegraphics[width=0.45\textwidth]{obrazky/unipi_osazeny}\label{unipi_osazeny}}
			\caption{Detaily UNiPi desky}
			\vspace{-5pt}
\end{figure*}

Na I2C sběrnici je dále připojen panel (viz Obr.~\ref{unipi_propojuvnitr}) se signalizačními diodami. UART sběrnice je zde připravena pro připojenů dalších modulů. Tyto desky jsou k~dostání v několika verzích, přizpůsobené konektorem pro konkrétní komunikační modul (viz Obr.~\ref{unipi_osazeny}).
 
Celá sestava desek je poté uložena v kovové krabičce s označením vstupů a výstupů (viz Obr.~\ref{unipi_unipi_box}).


\begin{figure*}[!ht]
		\centering
			\subfigure[Uložení v krabičce]{\includegraphics[width=0.45\textwidth]{obrazky/unipi_unipi_box}\label{unipi_unipi_box}}
			\hspace*{5mm}
			\subfigure[Připojení diodového panelu]{\includegraphics[width=0.45\textwidth]{obrazky/unipi_propojuvnitr}\label{unipi_propojuvnitr}}
			\caption{Detaily vnitřního uspořádání UniPi Neuronu S103}
			\vspace{-5pt}
\end{figure*}

Vnitřní uspořádání desky je řešeno pomocí funkčních celků (viz Obr.~\ref{BlokUniPi2Schema}), které jsou propojeny pomocí I2C sběrnice. Výstupy jednotlivých celků jsou poté vyvedeny na konektory desky.
 \begin{figure}[!ht]
	\vspace{-10pt}
  \begin{center}
    \includegraphics[scale=0.75]{obrazky/unipi_schema2}
  \end{center}
	\vspace{-30pt}
	\caption{Blokové schéma UniPi v2}
	\label{BlokUniPi2Schema}
	\vspace{-10pt}
\end{figure}

Napájení desky je pomocí 24\,V/1.5\,A adaptéru přímo na rozšiřující desku UniPi. Pro účely testování a implementace byla zapůjčena deska UniPi Neuron S103 vybavená počítačem RaspberryPi3.

 %\begin{figure}[!h]
 % \begin{center}
 %   \includegraphics[scale=1.0]{obrazky/unipi_v2}
 % \end{center}
 % \caption{UniPi Neuron S10x (v2) [docasne prevzeti]}
%\end{figure}

%%%%%%%%%%%%%%%%%%%%%%%%%%%%%%%%%%%%%%%
%%%%%%%%%%%%%%%%%%%%%%%%%%%%%%%%%%%%%%%
%%%%%%%%%%%%%%%%%%%%%%%%%%%%%%%%%%%%%%%

\section{Srovnání obou verzí}

Jak bylo popsáno v předchozím textu (Kap.~\ref{KapitolaUnipi1}~a~\ref{KapitolaUnipi2}), obě verze UniPi se liší svými parametry a využitím. I když vývoj UniPi byl nahrazen vývojem UniPi NEURONu, stále se najdou aplikace vhodné pouze pro původní desku:


\begin{itemize}
	\item Deska UniPi má reléově spínané výstupy, lze pomocí ní spínat i silové výstupy do 250\,V. Zatímco UniPi NEURON má spínané tranzistorové výstupy pouze do 50\,V, pro spínání vyšších napětí je nutné připojit reléový modul.
	\item Deska UniPi má zpřístupněnou I2C a UART sběrnici, zatímco na UniPi Neuronu je I2C využita pouze pro adresování vnitřních bloků a UART sběrnice je~alokována pro rozšiřující komunikační moduly.
	\item Software EVOK a software postavené na něm jsou v tomto okamžiku plně funkční pouze na desce první verze.	
\end{itemize}

\vspace{10pt}
Na některé aplikace však již tato deska vhodná není a je lepší využít UniPi NEURON:

\begin{itemize}
	\item I když UniPi první verze má sběrnici UART a teoretecky do ní lze připojit stejné rozšiřující komunikační moduly jako do UniPi Neuronu, součástí vývoje budou jen rozšiřující desky pro UniPi NEURON, jejichž nabídka má obsahovat spoustu dostupných technologií.
	\item Vzhledem k rozsáhlé nabídce modelů UniPi NEURON lze zvolit řešení na míru, včetně další konektivity.
	\item UniPi NEURON disponuje sběrnici RS-485 s protokolem ModBUS.
	\item UniPi NEURON má na kontaktech vysouvací svorky a celý modul zabírá méně místa.	
\end{itemize}

\vspace{10pt}
Vzhledem k tomu, že UniPi NEURON je v době vypracování práce jediná vývojem podporovaná verze, bude implementaci WM-Bus protokolu provedena na této verzi, avšak lze bez jakýchkoliv softwarových modifikací a pouze s jednou hardwarovou modifikací implementovat WM-Bus protokol i na UniPi první verze. Stačí pouze propojit příslušné piny IQRF modulu s piny modulárního konektoru UART sběrnice.


%%%%%%%%%%%%%%%%%%%%%%%%%%%%%%%%%%%%%%%
%%%%%%%%%%%%%%%%%%%%%%%%%%%%%%%%%%%%%%%
%%%%%%%%%%%%%%%%%%%%%%%%%%%%%%%%%%%%%%%

\section{Sběrnice na UniPi}
Jak je patrné z předhozích kapitol, jednodeskové počítače i rozšiřující moduly disponují množstvím komunikačních sběrnic. V této kapitole budou stručně představeny všechny výše zmíněné a zaměříme se na sběrnici UART, která bude sloužit pro~komunikaci mezi RaspberryPi a WM-Bus modulem.

\subsection{UART}
UART je synchronní a asynchronní sériové rozhraní pro přenos dat mezi zařízeními v obou směrech. Používá se pro komunikaci mezi mikrokontroléry, počítači a dalšími zařízeními podporující tento standard. Využívá dvouvodičovou sběrnici, vysílá data na pinu označovaném obvykle jako TX, přijímá na pinu RX.

 \begin{figure}[!ht]
	\vspace{-10pt}
  \begin{center}
    \includegraphics[scale=1.0]{obrazky/sbernice_uart}
  \end{center}
	\vspace{-30pt}
  \caption{UART rámec}
	\vspace{-5pt}
\end{figure}
Pro přenos se používají rámce, které mohou mít 5 až 9 bitů a jsou od sebe odděleny jedním start bitem a jedním nebo dvěma stop bity. Každý rámec může obsahovat ještě paritní bit pro kontrolu rámce. 

Dále je možné nastavit rychlost přenosu dat od 1~200\,bps až do 250\,kbps. Lze nastavit buď pro asynchronní režim, označovaný jako SCI, například pro RS-232 či RS-485, anebo pro synchronní režim, běžně označovaný jako SPI. Tato sběrnice je~ve~verzi 1 vyvedená do modulárního konektoru na desce, ve verzi 2 již není vyvedená na kontakty, ale je součástí desky plošného spoje, na kterém se přímo nachází slot pro komunikační modul.




\subsection{SPI}
SPI je sériové periferní rozhraní. Používá se pro komunikaci mezi řídícími mikroprocesory a ostatními integrovanými obvody. Jednotlivé obvody jsou propojeny čtyřmi vodiči:
\begin{itemize}
	\item Datový výstup MOSI (Master Out, Slave In) obvodu Master je připojen na~vstupy MOSI všech obvodů Slave.
	\item Datový vstup MISO (Master In, Slave Out) obvodu Master je propojen s~výstupy MISO všech obvodů Slave.
	\item Výstup hodinového signálu SCK je připojen na vstupy SCK všech obvodů Slave.
	\item Každý obvod Slave má vstup SS (Slave Select) pro výběr obvodu.
\end{itemize}
Komunikace je realizována pomocí společné sběrnice, je typu master-slave. Adresace se provádí pomocí zvláštních vodičů, které při logické nule aktivují příjem a vysílání zvoleného zařízení. Tato sběrnice se ani v jedné z desek UniPi nepoužívá ani není vyvedena ven.

\subsection{RS-485}
RS-485 se používá především v průmyslovém prostředí. Vyznačuje se dvouvodičovým propojením jednotek. Tyto vodiče se označují písmeny A a B. Přenos je poloduplexní, a proto se vyžaduje řízení přenosu dat. Pomocí dvouvodičové linky je možné připojit až 32 zařízení. Tato sběrnice není součástí první verze desky, v druhé verzi je vyvedena na kontakty.

\subsection{I2C}
I2C je interní datová sběrnice sloužící pro komunikaci a přenos dat mezi jednotlivými integrovanými obvody většinou v rámci jednoho zařízení. Sběrnice je duplexní a dvoudrátová. Na jednu sběrnici může být připojeno více obvodů, v základní sedmibitové verzi až 128 obvodů.
Vodiče jsou označeny jako serial data (SDA) a serial clock (SCL). Sběrnice je typu master-slave. Master při přenosu generuje hodinový signál na vodiči SCL. Když jeden obvod vysílá, všechny ostatní poslouchají a pouze podle adresy určují, zda jsou data určena jim. Obvod, který chce vyslat/přijmout data musí nejprve definovat adresu čipu, s kterým chce komunikovat a zda půjde o příjem nebo vysílání - tedy o čtení nebo zápis. To určuje bit, který je součástí adresy. 

Tato sběrnice je součástí obou verzí desky, využívá se pro propojení vnitřních funkčních bloků (EEPROM, RTC modul, AD převodník, 1Wire master, ...), v první verzi je i vyvedená do modulárního konektoru na desce, v druhé verzi již ne.

\subsection{1Wire}
Sběrnice 1Wire, navržená firmou Dallas Semiconductor, umožňuje připojit několik zařízení k řídící jednotce prostřednictvím pouhých dvou vodičů: data a zem. Sběrnice má jeden řídící obvod (master) a jeden či více ovládaných zařízení (slave). Všechny obvody jsou zapojeny jednak na společnou zem, a jednak paralelně na společný datový vodič. Tato sběrnice je součástí obou verzí desky, slouží pro připojení externích čidel (nejčastěji teplotní čidla) a u obou verzí desky je vyvedena do modulárního konektoru.

\subsection{GPIO}
\label{KapGPIO}
GPIO jsou piny, které lze programovat pomocí softwaru. Do těchto pinů lze posílat elektrický signál nebo jej z nich naopak přijímat. Na RaspberryPi 1 je takových vývodů celkem 26, na RaspberryPi 2 a RaspberryPi 3 je vývodů 40. GPIO vývodů je~zde standardně 8, krom nich se zde nachází i dva piny pro UART, 2 pro I2C a~6 pro SPI, ty však jdou také přenastavit pro GPIO využití. Nesmíme zapomenout ani na dva výstupy s napětím (3,3\,V a 5\,V) a zem.
Obrázek \ref{ObrazekGPIO} demonstruje rozložení GPIO konektoru napříč verzemi RaspberryPi.

\begin{figure}[!ht]
\vspace{-20pt}
  \begin{center}
    \includegraphics[scale=1.0]{obrazky/unipi_gpio}
  \end{center}
	\vspace{-20pt}
  \caption{Zpětná kompatibilita GPIO konektoru}
	\label{ObrazekGPIO}
	\vspace{-10pt}
\end{figure}

Jak již bylo popsáno v předchozích kapitolách, GPIO konektor není ve všech verzích RaspberryPi shodný. Model RaspberryPi B má 26 pinový konektor, zatímco verze B+, 2 a 3 mají konektor 40 pinový. Rozdíl je v tom, že u 40 pinového konektoru je prvních 26 pinů shodných a konektor je na zbývajících 14 pinech rozšířen o další vstupy a výstupy. Je tedy zpětně kompatibilní.


%%%%%%%%%%%%%%%%%%%%%%%%%%%%%%%%%%%%%%%
%%%%%%%%%%%%%%%%%%%%%%%%%%%%%%%%%%%%%%%
%%%%%%%%%%%%%%%%%%%%%%%%%%%%%%%%%%%%%%%


\section{Software pro UniPi}

Hlavní výhodou otevřené platformy RaspberryPi je možnost použít zákazníkem zvolený libovolný software. Neexistují omezení ze strany výrobce, proto si může každý svoje řešení postavit na míru.

Výrobce poskytuje vlastní software EVOK, který se stará o komunikaci desky UniPi přes virtuální server či API (Application Programming Interface) s uživatelem. Většina dalších open-source programů využívá toto API pro svůj provoz. Výrobcem je taktéž podporován software Mervis \cite{MervisWeb}, který z UniPi dělá plnohodnotné PLC. Dále je k dipozici několik open-source programů:
\begin{itemize}
\item EVOK - oficiální Python API s websocket a REST podporou.
\item PiDome - platforma pro domácí automatizaci.
\item pimmatic - platforma pro domácí automatizaci založená na node.js.
\item Node-RED - platforma založená na node.js s integrací do společnosti IBM Cloud Bluemix.
\item Wyliodrin - programování automatizace na bázi prohlížeče.
\item FHEM.de - domácí automatizační projekt napsaný v jazyce Perl.
\item JEEDOM - automatizační projekt napsaný v jazyce PHP.
\end{itemize}

\vspace{10pt}
A tři komerční:
\begin{itemize}
\item Mervis - profesionální domácí automatizační řešení s on-line SCADA.
\item REX - profesionální PLC s podporou mnoha průmyslových protokolů.
\item HomeSeer - odhlehčená platforma pro automatizaci domácnosti.
\end{itemize}

\vspace{10pt}
V době psaní této práce byl EVOK k dispozici i pro druhou verzi desky, avšak bez podpory komunikačních modulů. Implementace protokolu tedy bude prováděna jako samostatná aplikace s vlastní formou vizualizace naměřených dat.





 

%% Vložení kapitoly o MODULu
\chapter{Komunikační modul Wireless M-Bus}

Spolu se zařízením UniPi Neuron S103 byl zapůjčen i modul IQRF TR-72DC-WMB, který do budoucna bude součástí tohoto produktu a bude rozšiřovat konektivitu zařízení o protokol Wireless M-Bus. 

\section{Obecný popis modulu TR-72D-WMB}

Modul IQRF TR-72DA-WMB je bezdrátový komunikační modul velikosti SIM karty z výroby české firmy \href{http://microrisc.com/cs/}{MICRORISC s.r.o.}. Vychází z řady produktů technologie IQRF, s tím rozdílem, že místo IQRF softwaru má přímo implementovaný Wireless M-Bus protokol \cite{ModulIQRF}. 

 \begin{figure}[!ht]
\vspace{-1,pt}
  \begin{center}
    \includegraphics[scale=0.76]{obrazky/modul_modul}
  \end{center}
	\vspace{-20pt}
  \caption{Modul IQRF TR-72DA-WMB \cite{ModulIQRF}}
	\vspace{-10pt}
\end{figure}

Na malém prostoru se nachází vše potřebné pro uskutečnění bezdrátového přenosu: mikrokontrolér, externí EEPROM, teplotní senzor, kontrolní LED, 6 pinů a~anténa dle typu komunikačního modulu (Obr.~\ref{BlokovkaIQRF}).



Modul podporuje módy přenosu S1, S2, T1 a T2. Napájecí napětí modulu je~v~rozsahu 3.1 až 5.3\,V se spotřebou 1\,$\mu$A v režimu spánku a 8-22\,mA ve vysílacím režimu, dle nastavení výstupního výkonu, jehož maximální hodnota je 12.5\,W.
V~České republice je využíván pro přenos v bezlicenčním pásmu 868\,MHz, případně 433\,MHz nebo 169\,MHz.\newline

\newpage

 \begin{figure}[!ht]
	\vspace{-20pt}
  \begin{center}
    \includegraphics[scale=0.6]{obrazky/modul_block}
  \end{center}
	\vspace{-30pt}
  \caption{Blokové schéma modulu TR-72D-WMB \cite{ModulIQRF}}
	\label{BlokovkaIQRF}
	\vspace{-5pt}
\end{figure}

Modul je vyráběn ve třech verzích (viz Obr.~\ref{ObrazekAnteny}) dle připojení antény:
\begin{itemize}
		\item TR-72D-WMB má zdířku pro připájení antény.
		\item TR-72DC-WMB	má vyveden koaxiální anténí konektor U.FL. 
		\item TR-72DA-WMB má integrovanou anténu přímo na desce modulu. Dosah signálu toto typu je až 320\,m v módu T a 365\,m v módu S.
\end{itemize}

\begin{figure*}[!ht]
    \centering
			\subfigure[TR-72D-WMB]{\includegraphics[width=0.28\textwidth]{obrazky/modul_antenaa}\label{modul_antenaa}}
			\hspace*{5mm}
			\subfigure[TR-72DC-WMB]{\includegraphics[width=0.28\textwidth]{obrazky/modul_antenab}\label{modul_antenab}}
			\hspace*{5mm}
			\subfigure[TR-72DA-WMB]{\includegraphics[width=0.28\textwidth]{obrazky/modul_antenac}\label{modul_antenac}}
		\caption{Přehled typu modulu dle antény \cite{ModulIQRF}}
		\label{ObrazekAnteny}
		\vspace{-20pt}
\end{figure*}

\section{Komunikační módy}

Modul může být v závislosti na použité topologii nastaven do jednoho ze tří provozních módů: měřič, koncentrátor, skener \cite{ModulIQRF}. 

V módu měřiče může být modul přes UART sběrnici zapojen k mikrokontroléru, který zajistí zpracování dat od senzorů. Může tedy sloužit k sestavení vlastních měřicích zařízení postavených na protokolu Wireless M-Bus.

V módu koncentrátoru slouží modul jako komunikační zařízení pro sběr dat z~meřičů. Aktuální firmware podporuje pouze obousměrnou komunikaci s měřiči v~režimu S a T a je zatím ve fázi vývoje a do produkce nasazen jako experimentální. Z tohoto důvodu bude při implementaci samotné aplikace modul nasazen v režimu skeneru.

V módu skeneru modul zachytává veškerou dostupnou komunikaci daného módu přenosu. Díky vnitřní implementaci Wireless M-Bus protokolu je modul schopen zachytávat a dešifrovat šifrovanou komunikaci, je však nutné počítat s tím, že současný firmware není stavěný na vyžití modulu pro příjem šifrované komunikace v módu skeneru od více zařízení zároveň. Modul totiž automaticky veškeré zachycené šifrované telegramy automaticky rozšifruje pomocí jediného interního AES klíče. Jedná se však o klíč daného modulu, nikoliv vyčítaného zařízení. Při vyčítání šifrovaných dat je tedy nutné postupovat složitěji a provést nejdříve zpětné zašifrování daných dat tímto klíčem a až poté provést rozšifrování dat dle normy.

Praktické využíti jednotlivých módu zobrazuje Obr.~\ref{TopologieIQRF}.

 \begin{figure}[!ht]
\vspace{-20pt}
  \begin{center}
    \includegraphics[scale=0.65]{obrazky/modul_topologie}
  \end{center}
	\vspace{-30pt}
  \caption{Různé módy dle použité topologie \cite{ModulIQRF}}
	\label{TopologieIQRF}
	\vspace{-20pt}
\end{figure}

\section{Komunikační protokol}

S řídícím mikrokontrolérem modul komunikuje pomocí rozhraní UART, jehož parametry jsou 19200\,Bd, 8\,bitů, žádná parita a 1 stop bit.
		
Modul podporuje jednoduchý formát příkazů sloužící k nastavení konfiguračních parametrů modulu i ke samotné komunikaci s modulem. Každý příkaz začíná znakem \textgreater. Každá zpráva odpověďi začíná znakem \textless. Každému příkazu musí předcházet budící znak NULL (0x00) následovaný 2\,ms pauzou a každý paket příkazů je ukončen znakem CR (0x0D). 

\newpage
Obecná struktura~\cite{ModulIQRF}~paketu příkazu je 

\begin{figure}[!ht]
\begin{centerverbatim}
			>[CC][RW][DATA][CR]
\end{centerverbatim}
\end{figure}

kde CC je jednobajtový kód daného příkazu, RW je jednobajtový příznak zápisu (\textbf{:}) či čtení dat (\textbf{?}) dat, DATA jsou zapisovaná data, pokud se jedná o zápis a CR je znak ukončení.

Obecná skruktura odpovědi je následující

\begin{figure}[!ht]
\begin{centerverbatim}
	<[DATA][CR]
\end{centerverbatim}
\end{figure}

kde DATA obsahuje přenášená data či návratové kódy (OK pro správné dokončení příkazu, ERR1 pro chybu syntaxe a ERR2 pro neplatnou vstupní hodnotu). Některé bajty jsou kódovány v šestnáckové soustavě či využívají uložení BigEndian.

Například dotaz a odpověd pro aktuální AES klíč je:
\label{KapitolaStazeniKlice}

\begin{figure}[!ht]
\begin{centerverbatim}
	>03?[CR]
	<010203040506070809a0b0c0d0e0f[CR]  
\end{centerverbatim}
\end{figure}

a pro případnou změnu AES klíče:

\begin{figure}[!ht]
\begin{centerverbatim}
>03:112233445566778899aabbccddeeff[CR]
>OK[CR]
\end{centerverbatim}
\end{figure}

Ukázku jednoduché komunikace s modulem v jazyce Python obsahuje~Kód~\ref{CodeSerial}.

\begin{lstlisting}[caption={Komunikace s modulem přes sériový port},captionpos=b,label=CodeSerial,style=MyCodePython]
import serial

# Setup a serial port
	ser = serial.Serial(
		port='/dev/ttyAMA0',
		baudrate=19200,
		parity=serial.PARITY_NONE,
		stopbits=serial.STOPBITS_ONE,
		bytesize=serial.EIGHTBITS,
		timeout=1
	)
  output("SCR", "Device is ready: " + str(ser.isOpen()))

# Set default AES key
	ser.write("\x00\x00>03:" + AES_IQRF_DEFAULT + "\x0D")
	y = ser.readline()
	output("SCR", "Default AES key set: " + y[1:])
\end{lstlisting}





%% Vložení kapitoly o WM-BUS protokolu
\chapter{Wireless M-Bus protokol}
\label{ChapterWMBus}
Wireless M-Bus je v Evropě perspektivní otevřený standard pro automatické měření, který pracuje v sub-gigahertzovém bezlicenčním pásmu v okolí 868 MHz. Wireless M-Bus se primárně zaměřuje na použití v SRD (Short Range Device) zařízeních pro bezdrátovou komunikaci s měřiči energií, jako jsou: voda, plyn, teplo, elektřina, atd. Měřiče energií, vybavené bezdrátovým rozhraním Wireless M-Bus jsou schopny komunikovat jak se stacionárními, tak i s mobilními čtecími zařízeními. Předpokládá se, že rádiová část měřiče je napájena z baterie a je schopna provozu po dobu 10-15 let bez výměny baterie. Na čtecích zařízeních, ať už stacionárních nebo mobilních, není takový požadavek na dobu provozu na baterie a čtecí zařízení mohou být napájena i z externího zdroje.

Wireless M-Bus má svůj původ v rámci norem Meter-Bus. Wireless Meter Bus je~bezdrátovou variantou drátového Meter-Bus. To je standard zaměřený na aplikace pro sběr dat měřiče plynu, elektřiny a vody. Sběrnice je specifikována v evropské normě EN 13757~\cite{Norma1}. Tato specifikace je rozdělena do pěti částí (viz Tab.~\ref{TableNorma}), z~nichž jedna se zaměřuje na Wireless M-Bus.

\begin{table}[!ht]
\centering
\caption{Popis standardu EN-13757~\cite{WmbusTables}}
\label{TableNorma}
\resizebox{\textwidth}{!}{%
\begin{tabular}{|l|l|}
\hline
{\textbf{Standard}} & \multicolumn{1}{c|}{\textbf{Podrobnosti}} \\ \hline \hline
EN 13757-1 & \begin{tabular}[c]{@{}l@{}}Část 1 standardu definuje výměnu dat, která podrobně popisuje základní \\ komunikaci mezi vodoměry a centrálním sběračem dat. Poskytuje přehled \\ komunikačního systému.\end{tabular}\\ \hline
EN 13757-2 & \begin{tabular}[c]{@{}l@{}}Tato část normy Meter Bus řeší fyzickou a spojovou vrstvu pro fyzický přenos \\ dat pomocí kabelových spojů. Také popisuje protokol používaný pro přenos dat.\end{tabular} \\ \hline
EN 13757-3 & \begin{tabular}[c]{@{}l@{}}Část 3  se týká speciální aplikační vrstvy. Ta popisuje standardní aplikační\\  protokol používaný k tomu, aby se zachovala kompatibilita výrobců, což \\ umožňuje zařízení od několika různých dodavatelů působit v jednom systému.\end{tabular} \\ \hline
EN 13757-4 & \begin{tabular}[c]{@{}l@{}}Oddíl 4 popisuje bezdrátový systém. Jedná se o radiový odečet pro provoz v pásmu \\ 868\,MHz až 870\,MHz. Tato část normy se zabývá fyzickou a linkovou vrstvou pro\\  bezdrátová zařízení.\end{tabular} \\ \hline
EN 13757-5 & \begin{tabular}[c]{@{}l@{}}Tato část definuje adresy předávání. To zahrnuje celou řadu návrhů na předávání \\ datových rámců jako prostředek komunikace mezi měřičem a koncentrátorem.\end{tabular} \\ \hline \hline
\end{tabular}}
\vspace{-10pt}
\end{table}

%%%%%%%%%%%%%%%%%%%%%%%%%%%%%%%%%%%%%%%%%%%%%%%%%=
%%%%%%%%%%%%%%%%%%%%%%%%%%%%%%%%%%%%%%%%%%%%%%%%%=
%%%%%%%%%%%%%%%%%%%%%%%%%%%%%%%%%%%%%%%%%%%%%%%%%=
%%%%%%%%%%%%%%%%%%%%%%%%%%%%%%%%%%%%%%%%%%%%%%%%%=

\section{Princip komunikace}

Bezdrátová komunikace Wireless M-Bus fyzicky probíhá ve 12 kanálech v bezlicenčním vysílacím pásmu ISM (industrial, scientific and medical) okolo frekvence 868\,MHz (2~kanály\,868,3 a 868,95\,MHz jsou využívány režimem S a T, 10 uživatelem volitelných kanálů 868,03 + n x 0,06\,MHz v režimu R2), přičemž každý z výše uvedených režimů vyžaduje různé požadavky. Těmi například jsou specifikovaný kanál, přesnost frekvence, toleranci přenosové rychlosti atd. Velmi dobrá je stabilita frekvence až 27 let (dle údaje výrobce). V případě použití čtvrtvlné antény (délky 8,2 cm), tak na přímou viditelnost vysílacího a přijímacího modulu je komunikační dosah 500 až 600\,m.

Komunikace má hvězdicovitou strukturu, kdy několik měřících jednotek/snímačů přenáší svá naměřená data jedné centrální jednotce, obvykle tvořené koncentrátorem. Ten tedy obvykle slouží pro příjem a shromaždování dat z několika měřících míst, z dále uvedených důvodů nikdy neinicializuje (nezahajuje) vzájemnou komunikaci. Pracuje tedy jako server (Master), tzn. že stále naslouchá a čeká na navázání komunikace měřící jednotkou a jí inicializovaný přenos dat. Ta tedy pracuje jako klient (Slave). V případě nastavené obousměrné komunikace přechází měřič/snímač do přijímacího režimu pouze po krátký čas jím navázané komunikace. Pouze v tomto momentu může koncentrátor vyslat nějaké jednotce řídící data. Časování je rozdílné pro různé režimy a je přesně specifikováno ve standardu.

Adresování ve Wireless M-Bus sběrnici je převzato z klasické drátové verze M-BUSu. Zde však pouze klientské jednotky (měřiče/snímače) mají přidělenou adresu a využívají ji jak při příjmu, tak při vysílání. Každý koncentrátor by měl obsahovat tabulku adres, se kterými může komunikovat, resp. od kterých má přijímat data. Tato tabulka se obvykle vytváří automaticky během instalace/registrování nové jednotky do sítě. Samozřejmě je možné se obejít i bez ní, ale pak lze přijímat všechny snímače či měřiče v dosahu. Toho se dá využít jen v malých sítích. 

%%%%%%%%%%%%%%%%%%%%%%%%%%%%%%%%%%%%%%%%%%%%%%%%%=
%%%%%%%%%%%%%%%%%%%%%%%%%%%%%%%%%%%%%%%%%%%%%%%%%=
%%%%%%%%%%%%%%%%%%%%%%%%%%%%%%%%%%%%%%%%%%%%%%%%%=
%%%%%%%%%%%%%%%%%%%%%%%%%%%%%%%%%%%%%%%%%%%%%%%%%=

\section{Režimy přenosu}
Nejdůležitější vlastností technologie WM-Bus je možnost bateriového napájení měřicích zařízení. V případě bezdrátové komunikace je výhodné například měřiče tepla nebo vodoměry napájet jen bateriově a tím eliminovat jakoukoliv nutnost pokládání kabelů. To ale znamená velmi omezenou spotřebu elektrické energie, aby baterie vydržely co nejdéle, alespoň několik let. V současné době v případě napájení modulu lze dosáhnout životnost na jednu baterii až 12~let~\cite{CidloWeptech,CidloBonega}. Aby to však bylo možné, řízení přenosu dat musí co nejčastěji přecházet do nízkopříkonového stavu (sleep mode) a vysílat data jen v nutných případech v co nejkratších časových slotech. Proto také centrální zařízení (koncentrátor), který obvykle slouží pro příjem a shromaždování dat z několika měřících míst, nikdy nesmí inicializovat vzájemnou komunikaci.

Protokol podporuje několik režimů přenosu, lišících se dle požadavků na konkretní aplikaci. Je definováno několik režimů označených jako S, T a R představující 3 různé různé přenosové rychlosti, které se dále dělí na režim 1 a 2, což značí jednosměrný či obousměrný přenos dat. U některých zařízení mohou být doplněny o~režimy N, C a F. Tyto režimy jsou shrnuty v Tab.~\ref{TableRezimy}.

\begin{table}[!ht]
\centering
\caption{Režimy přenosu WM-Bus protokolu~\cite{WmbusTables}}
\label{TableRezimy}
\resizebox{\textwidth}{!}{%
\begin{tabular}{|c|l|l|l|l|l|}
\hline
\textbf{Mód} & \multicolumn{1}{c|}{\textbf{Mód přenosu}} & \multicolumn{1}{c|}{\textbf{Směr}} & \multicolumn{1}{c|}{\textbf{Frekvence}} & \multicolumn{1}{c|}{\textbf{Kódování}} & \multicolumn{1}{c|}{\textbf{Rychlost}} \\ \hline \hline
S & Stacionární & \begin{tabular}[c]{@{}l@{}}Jednosměrný,\\ i obousměrný\end{tabular} & 868\,MHz & Manchester & 32,768\,kbps \\ \hline
T & Častý vysílací & \begin{tabular}[c]{@{}l@{}}Jednosměrný,\\ i obousměrný\end{tabular} & 868\,MHz & \begin{tabular}[c]{@{}l@{}}Manchester \\  a 3 z 6\end{tabular} & 100\,kbps \\ \hline
R & Častý přijímací & \begin{tabular}[c]{@{}l@{}}Jednosměrný,\\ i obousměrný\end{tabular} & 868\,MHz & Manchester & 4,8\,kbps \\ \hline
N & Úzkopásmový & \begin{tabular}[c]{@{}l@{}}Jednosměrný,\\ i obousměrný\end{tabular} & 169\,MHz & NRZ &  \\ \hline
C & Kompaktní & \begin{tabular}[c]{@{}l@{}}Jednosměrný,\\ i obousměrný\end{tabular} & 868\,MHz & Manchester & 50\,kbps \\ \hline
F & \begin{tabular}[c]{@{}l@{}}Častý vysílací \\ i přijímací mód\end{tabular} & Obousměrný & 433\,MHz & NRZ &  \\ \hline \hline
\end{tabular}}
\end{table}

Mód S je určen pro jednosměrnou nebo obousměrnou komunikaci mezi pevnými nebo mobilními zařízeními. Centrální frekvence tohoto módu je 868,3\, MHz s dobou provozu 0,02\,\% za hodinu. Přenosová rychlost je pro tento mód 32,768\,kbps. Pro~operační mód S jsou definovány tři submódy: S1, S1-m a S2. Submód S1 lze~použít pro jednosměrnou komunikaci nevyžadující potvrzení o přijetí rámce a~je~určen pro aplikace, kdy se vysílá několikrát za den ke statickému přijímači. Pro kódování používají všechny submódy módu S kódování Manchester.  
Submód S1-m je~modifikací submódu S1 pro komunikaci mezi čidlem a koncentrátorem, zasílaný rámec obsahuje zkrácenou hlavičku.
Submód S2M podporuje oboustranou komunikaci v~kontinuálních cyklech bez nutnosti probouzet zařízení.

V	módu T měřič samostatně odesílá data, buď periodicky nebo aperiodicky (když jsou k dispozici). Pro přenos rámce z měřiče k dalším zařízením je použita přenosová rychlost 100 kbps s kódováním 3 z 6, zatímco komunikace v opačném směru má přenosovou rychlost 32,768 kbps a kódování je použito Manchester. Submód T1 je definován jako jednosměrná komunikace, při které měřič nevyžaduje potvrzení od příjemce o přijatém rámci. Měřič odešle data a přepne se do úsporného režimu. Zatímco submód T2 je definován jako obousměrná komunikace. Měřič po odeslání rámce krátkou dobu vyčkává na potvrzení od příjemce. Pokud měřič neobdrží odpověď přepne se do úsporného režimu. Pokud ve stanoveném čase příjemce odpoví, naváže se obousměrná komunikace mezi měřičem a koncentrátorem.

V	módu R měřič samostatně neodesílá změřená data, ale vyčkává na výzvu od~koncentrátoru. Měřič je v úsporném režimu a v pravidelných úsecích se periodicky probouzí do režimu přijmu a očekává rámec. Když není přijat žádný validní wake-up rámec, měřič se přepne zpět do úsporného režimu. V	opačném případě se naváže obousměrná komunikace mezi měřičem a koncentrátorem.

V režimech S, T a R je každý bajt vysílán s nejvíce důležitým bitem (MSB - Most
Significant Bit) na prvním místě. Implementace MSB v jazyce Python je zobrazena v Kódu~\ref{CodeMSB}.

\begin{lstlisting}[caption={Implementace vyčítaní uložení MSB},captionpos=b,label=CodeMSB,style=MyCodePython]
def MSB(bytes):
    new = ""
    size = len(bytes)
    while (size>0):
        new = new + bytes[size-2:size]
        size=size-2
    return new
\end{lstlisting}

%%%%%%%%%%%%%%%%%%%%%%%%%%%%%%%%%%%%%%%%%%%%%%%%%=
%%%%%%%%%%%%%%%%%%%%%%%%%%%%%%%%%%%%%%%%%%%%%%%%%=
%%%%%%%%%%%%%%%%%%%%%%%%%%%%%%%%%%%%%%%%%%%%%%%%%=
%%%%%%%%%%%%%%%%%%%%%%%%%%%%%%%%%%%%%%%%%%%%%%%%%=

\section{Struktura zasílaných dat}
Komunikace probíhá následovně: nadřazené aplikace realizující aplikační vrstvu standardu M-Bus vyšlou svá data do RF modemu v podobě datové jednotky, která je~zobrazena v Tab.~\ref{PaketWm1}:

\begin{table}[!ht]
\vspace{-10pt}
\centering
\begin{tabular}{ccc}
1 Bajt & 1 Bajt & n Bajtů \\ \hline
\multicolumn{1}{|c|}{Length} & \multicolumn{1}{c|}{CI} & \multicolumn{1}{c|}{AppLayer} \\ \hline
\end{tabular}
\caption{Formát datové jednotky~\cite{FormatDatoveJednotky}}
\label{PaketWm1}
\vspace{-10pt}
\end{table}

Komunikační modul pracující jako modem dle požadavků standardu Wireless M-Bus automaticky přidá následující pole:

\begin{itemize}
	\item Řídicí pole.
\item Označení výrobce dle~\cite{WmbusVendors}.
\item Unikátní komunikační adresy založené na parametrech uložených v paměti modulu.
\item Případně se ještě na závěr přidá informace o síle přijímaného signálu RSSI (Received Signal Strength Indication).
\end{itemize}



\begin{table}[!ht]
\centering
\begin{tabular}{ccccccc}
1 Bajt & 1 Bajt & 2 Bajty & 6 Bajtů & 1 Bajt & n Bajtů & 1 Bajt \\ \hline
\multicolumn{1}{|c|}{Legth} & \multicolumn{1}{c|}{C} & \multicolumn{1}{c|}{ManID} & \multicolumn{1}{c|}{Address} & \multicolumn{1}{c|}{CI} & \multicolumn{1}{c|}{AppLayer} & \multicolumn{1}{c|}{RSSI} \\ \hline
\end{tabular}
\caption{Formát datové jednotky protokolu Wireless M-Bus~\cite{FormatDatoveJednotky}}
\label{PaketWm2}
\vspace{-5pt}
\end{table}

Takovýto paket se pak zašifruje (obvykle algoritmem AES-128) a přenáší se vzduchem. V případě, že se realizuje jen bezdrátové tunelování přenosu mezi dvěma Wireless M-Bus modemy, je povolen i režim bez zasílání adresy a jí přidružených informacích o měřící jednotce. Rámec se pak výrazně zjednoduší a jeho struktura je~zobrazena v Tab.~\ref{PaketWm3}.

			
			\begin{table}[!ht]
			\vspace{-10pt}
\centering
\begin{tabular}{cccc}
1 Bajt & 1 Bajt & n Bajtů & 1 Bajt \\ \hline
\multicolumn{1}{|c|}{Legth} & \multicolumn{1}{c|}{CI} & \multicolumn{1}{c|}{AppLayer} & \multicolumn{1}{c|}{RSSI} \\ \hline
\end{tabular}
\caption{Zkrácený formát datové jednotky~\cite{FormatDatoveJednotky}}
\label{PaketWm3}
\vspace{-20pt}
\end{table}
			
Obsah pole AppLayer je již dán aplikační vrstvou definovanou ve standardu M-Bus, které se používá jako mechanizmus komunikace z linkové vrstvy do vyšších protokolových vrstev, a je tedy shodný s obsahem pro klasický drátový M-Bus přenos.  Data následující za polem Cl jsou již závislá na aplikační vrstvě M-Bus. Komunikace mezi měřící jednotkou a RF modemem či mezi koncentrátorem a RF modem obvykle probíhá prostřednictvím sériového přenosu UART, například s využitím RS-232, RS-485 či USB.

Při přenosu datové jednotky uvedené v Tab.~\ref{PaketWm3} IQRF modulem dochází k jejímu rozšíření o položky uvedené v Tab.~\ref{PaketWm4}.

\begin{table}[!h]
\centering
\begin{tabular}{ccccccc}
1 Bajt & 1 Bajt & 12\,-\,n Bajtu & 1 Bajt & 1 Bajt & 1 Bajt & 1 Bajt\\ \hline
\multicolumn{1}{|c|}{Length} & \multicolumn{1}{c|}{Status} & \multicolumn{1}{c|}{...} & \multicolumn{1}{c|}{CRC} & \multicolumn{1}{c|}{RSSI} & \multicolumn{1}{c|}{CR} & \multicolumn{1}{c|}{0A}\\ \hline
\end{tabular}
\caption{Formát datové jednotky po přijetí modulem IQRF}
\label{PaketWm4}
\end{table}




%%%%%%%%%%%%%%%%%%%%%%%%%%%%%%%%%%%%%%%%%%%%%%%%%=
%%%%%%%%%%%%%%%%%%%%%%%%%%%%%%%%%%%%%%%%%%%%%%%%%=
%%%%%%%%%%%%%%%%%%%%%%%%%%%%%%%%%%%%%%%%%%%%%%%%%=
%%%%%%%%%%%%%%%%%%%%%%%%%%%%%%%%%%%%%%%%%%%%%%%%%=


\section{Popis jednotlivých vrstev}

Norma EN 13757-4 specifikuje fyzickou a linkovou vrstvu. Na ně následně navazuje aplikační vrstva, která je shodná s původním M-Bus protokolem.

\subsection{Fyzická vrstva Wireless M-Bus}
Fyzická vrstva definuje jak mají být bity kódovány a vysílány, tedy radiofrekvenční charakteristiky a radiofrekvenční parametry. Fyzická vrstva je realizována hardwarem, případně v kombinaci s firmwarem daného hardware.

Wireless M-Bus dle normy ČSN~EN~13757-4~\cite{Norma4} využívá tři pásma pro tři různé módy komunikace: 868,3\,MHz pro módy Sx, 868,95\,MHz pro módy Tx a 868,33\,MHz pro mód R2 jsou definovány tři různé operační módy komunikace. Všechny tři módy používají modulaci 2-FSK, tedy dvoustavovou frekvenční modulaci. Pro některé módy jsou některé parametry fyzické vrstvy stejné, proto je fyzické zařízení schopné s nezměněným hardwarem komunikovat v různých operačních módech.


\subsubsection{Kódování používaná ve Wireless M-Bus}
Wireless M-Bus definuje dvojí možné kódování: 
\begin{itemize}
	\item kódování Manchester,
	\item kódování 3 ze 6. 
\end{itemize}

Kódování Manchester (viz Obr. \ref{ObrazekManechester}) slučuje datový a hodinový signál do jediného signálu. Toto kódování se krom bezdrátových přenosů používá i v sítích LAN, konkrétně v síti Ethernet. Výhodou kódu Manchester je konstantní střední hodnota takového signálu, která je 50\,\% z maximální hodnoty. Náběžné hrany ohraničují jeden bit dat a sestupné hrany určují kód Manchester. Logická jednička je reprezentována náběžnou hranou a logická nula hranou sestupnou. 

				\begin{figure}[!ht]
				\vspace{-20pt}
 \begin{center}
    \includegraphics[scale=1.0]{obrazky/wmbus_manchester}
  \end{center}
	\vspace{-40pt}
  \caption{Princip kódování Manchester}
	\label{ObrazekManechester}
	\vspace{-10pt}
\end{figure}

Pokud nejsou vysílána žádná data, výstup kódování Manchester je hodinový signál. Nevýhodou použití Manchester kódování je to, že na přenos jednoho bitu informace je potřeba dvou hodinových taktů.

Princip kódování 3 ze 6 (viz Tab.~\ref{Table3out6}) spočívá v tom, že každé 4 bity (nibble) jsou zakódovány jako 6ti bitová data, přičemž zakódované slovo obsahuje stejné množství nul a jedniček. Zároveň v kódu musí být alespoň dvě změny, tzn. není možné použít \uv{000111} nebo \uv{111000}. Takto zakódovaná data jsou přenášené s nejvýznamnějším bitem jako prvním. Toto kódování by mělo být aplikováno při použití módu častého vysílání (módy T1 a T2) a při komunikaci měřiče s koncentrátorem. Koncentrátor může odpovědět měřiči zprávou kódovanou kódováním Manchester.


\begin{table}[!ht]
\centering
\caption{Tabulka kódování 3 ze 6 \cite{WMencodeing}}
\label{Table3out6}
\begin{tabular}{|c|c|c|c|c|}
\hline
\textbf{NRZ kód} & \textbf{Desítkově} & \textbf{3 ze 6} & \textbf{Desítkově} & \textbf{Počet změn v kódu} \\ \hline \hline
0                & 0                  & 10110               & 22                 & 4                          \\ \hline
1                & 1                  & 1101                & 13                 & 3                          \\ \hline
10               & 2                  & 1110                & 14                 & 2                          \\ \hline
11               & 3                  & 1011                & 11                 & 3                          \\ \hline
100              & 4                  & 11100               & 28                 & 2                          \\ \hline
101              & 5                  & 11001               & 25                 & 3                          \\ \hline
110              & 6                  & 11010               & 26                 & 4                          \\ \hline
111              & 7                  & 10011               & 19                 & 3                          \\ \hline
1000             & 8                  & 101100              & 44                 & 3                          \\ \hline
1001             & 9                  & 100101              & 37                 & 4                          \\ \hline
1010             & 10                 & 100110              & 38                 & 3                          \\ \hline
1011             & 11                 & 100011              & 35                 & 2                          \\ \hline
1100             & 12                 & 110100              & 52                 & 3                          \\ \hline
1101             & 13                 & 110001              & 49                 & 2                          \\ \hline
1110             & 14                 & 110010              & 50                 & 3                          \\ \hline
1111             & 15                 & 101001              & 41                 & 4                          \\ \hline \hline
\end{tabular}
\end{table}

\subsection{Linková vrstva Wireless M-Bus}

Linková vrstva poskytuje rozhraní mezi fyzickou a aplikační vrstvou. Její hlavní funkce jsou:
\begin{itemize}
	\item Poskytování služeb převádějících data mezi fyzickou a aplikační vrstvou.
	\item Generování CRC pro odchozí zprávy.
	\item Detekování CRC chyb v příchozích zprávách.
	\item Poskytování adresování fyzické vrstvy.
	\item Kontrola ACK u obousměrných přenosů.
	\item Vytváření rámců.
	\item Kontrola chyb rámců v příchozích zprávách.
\end{itemize}

Rámec linkové vrstvy se skládá z bloků dat. Každý blok dat obsahuje 16bitové CRC pole.  První blok má pevnou délku 12 bajtů a obsahuje L, C, M a A pole.

\subsubsection{L-Pole}
\begin{itemize}
	\item Určuje velikost přenášených dat, ale bez samotného L-pole a kontrolního součtu.	
\end{itemize}

\subsubsection{C-Pole}
\begin{itemize}
	\item Identifikuje typ rámce (SEND, CONFIRM, REQUEST, RESPONSE).
	\item Používá se pro zasílání základních příkazů.
\end{itemize}

\subsubsection{M-Pole}
\begin{itemize}
	\item Obsahuje identifikaci výrobce zařízení.
	\item Je kódováno jako třípísmenný kód, který se získává následovně:
			\begin{figure}[!ht]
				\begin{centerverbatim}
				Manufacturer ID = [ASCII(Znak1) - 64] + 32 + 32
												+ [ASCII(Znak2) - 64] + 32
												+ [ASCII(Znak3) - 64]
				\end{centerverbatim}
			\end{figure}
\end{itemize}
\vspace{-30pt}

\subsubsection{A-Pole}
\begin{itemize}
	\item Obsahuje 6 bajtů určující adresu zařízení.
	\item U rámců SEND a REQUEST je zde adresa vysílajícího zařízení.
	\item U rámců CONFIRM a RESPONSE je zde adresa zařízení, které je paket určen.
	\item Je tvořen následovně:
		\begin{itemize}
			\item 4 bajty (identifikační číslo) kódované jako 8 BCD znaků. Jedná se o unikátní identifikaci stanovenou výrobcem.
			\item 2 bajty (verze zařízení) určující generaci daného zařízení ve výrobním procesu výrobce.
			\item 2 bajty (typ zařízení), kódované dle Tab.~\ref{def_device_type_identification}.
		\end{itemize}
\end{itemize}

			\begin{table}[!ht]
			\centering
			\vspace{-10pt}
			\caption{Identifikace typu zařízení}
			\label{def_device_type_identification}
			\begin{tabular}{|c|c|c|c|c|c|}
			\hline
			\textbf{Hodnota} & \textbf{Bit 16}   & \textbf{Bit 15}   & \textbf{Bit 8}   & \textbf{Bit 7}  & \textbf{Médium  }                      \\ \hline\hline
			0           & 0        & 0        & 0       & 0      & Ostatní                   \\ \hline
			1           & 0        & 0        & 0       & 1      & Olej                     \\ \hline
			2           & 0        & 0        & 1       & 0      & Elektřina             \\ \hline
			3           & 0        & 0        & 1       & 1      & Benzín                     \\ \hline
			4           & 0        & 1        & 0       & 0      & Vytápění                    \\ \hline
			5           & 0        & 1        & 0       & 1      & Pára                   \\ \hline
			6           & 0        & 1        & 1       & 0      & Horká voda               \\ \hline
			7           & 0        & 1        & 1       & 1      & Voda                   \\ \hline
			8           & 1        & 0        & 0       & 0      & Tepelné čerpadlo                \\ \hline
			9           & 1        & 0        & 0       & 1      & Rezervováno                \\ \hline
			A           & 1        & 0        & 1       & 0      & Benzín 2              \\ \hline
			B           & 1        & 0        & 1       & 1      & Vytápění 2             \\ \hline
			C           & 1        & 1        & 0       & 0      & Horká voda 2        \\ \hline
			D           & 1        & 1        & 0       & 1      & Voda 2            \\ \hline
			E           & 1        & 1        & 1       & 0      & Tepelné čerpadlo 2           \\ \hline
			F           & 1        & 1        & 1       & 1      & Rezervováno                 \\ \hline\hline
			\end{tabular}
			\vspace{-5pt}
			\end{table}

\vspace{-10pt}
\subsubsection{CI-Pole}
\begin{itemize}
	\item Určuje typ přenášených dat.
	\item Nejčastější typy jsou uvedeny v Tab.~\ref{def_ci_pole}.
\end{itemize}

		\begin{table}[!ht]
		\centering
		\vspace{-30pt}
		\caption{Kódování CI-Pole}
		\label{def_ci_pole}
		\begin{tabular}{|c|c|c|}
\hline
\textbf{Hodnota} & \multicolumn{1}{c|}{\textbf{Direction}} & \multicolumn{1}{c|}{\textbf{Protokol}} \\ \hline \hline
50h               & Výběr aplikace zařízení           & pouze M-Bus    \\ \hline
51h               & Požadavek na zařízení                              & pouze M-Bus      \\ \hline
52h               & Výběr zařízení                    & pouze M-Bus      \\ \hline
5Ah               & Požadavek na zařízení                           & (W)M-Bus          \\ \hline
5Bh               & Požadavek na zařízení                            & M-Bus         \\ \hline
60h               & Požadavek na zařízení                            & DLMS         \\ \hline
61h               & Požadavek na zařízení                            & DLMS        \\ \hline
64h               & Požadavek na zařízení                            & SML     \\ \hline
65h               & Požadavek na zařízení                            & SML a   \\ \hline
6Ch               & Synchonizace času zařízení                     & všechny OMS     \\ \hline
6Dh               & Synchonizace času zařízení                       & všechny OMS        \\ \hline
6Eh               & Chyba zařízení                      & všechny OMS       \\ \hline
6Fh               & Chyba zařízení                        & všechny OMS       \\ \hline
70h               & Chyba zařízení                       & pouze M-Bus                    \\ \hline
71h               & Alarm zařízení                       & pouze M-Bus                    \\ \hline
72h               & Odpověd zařízení                     & (W)M-Bus                                              \\ \hline
74h               & Alarm zařízení                         & všechny OMS                                            \\ \hline
75h               & Alarm zařízení                         & všechny OMS                                            \\ \hline
78h               & Odpověd zařízení                   & (W)M-Bus                           \\ \hline
7Ah               & Odpověd zařízení                     & (W)M-Bus                                              \\ \hline
7Ch               & Odpověd zařízení                    & DLMS                                          \\ \hline
7Dh               & Odpověd zařízení                     & DLMS                                         \\ \hline
7Eh               & Odpověd zařízení                     & SML                                           \\ \hline
7Fh               & Odpověd zařízení                     & SML                                           \\ \hline \hline
\end{tabular}
\vspace{-10pt}
\end{table}

\subsubsection{CRC}
\begin{itemize}
	\item CRC obsahuje kontrolní součet pro kontrolu správnosti přenosu. 
	\item Jako kontrolní polynom se dle specifikace používá x\textsuperscript{16} + x\textsuperscript{13} + x\textsuperscript{12} + x\textsuperscript{11} + x\textsuperscript{10} + x\textsuperscript{8} +\textsuperscript{6} + x\textsuperscript{5} + x\textsuperscript{2} + 1.
\end{itemize}

\subsubsection{RSSI}
\begin{itemize}
	\item Received Signal Strength Indication.
	\item Určuje sílu signálu při přijetí paketu.
	\item Pro převod je využita lineární konverze:
\end{itemize}	
				\begin{figure}[!ht]
				\begin{centerverbatim}
				RSSI [dBm] = RSSI_LEVEL/2 - 130
				\end{centerverbatim}
			\end{figure}


%%%%%%%%%%%%%%%%%%%%%%%%%%%%%%%%%%%%%%%%%%%%%%%%%%%%%%%%%%%%%%%%%%%%%%%%%%%%%%%%%%%%%%%%%%
%%%%%%%%%%%%%%%%%%%%%%%%%%%%%%%%%%%%%%%%%%%%%%%%%%%%%%%%%%%%%%%%%%%%%%%%%%%%%%%%%%%%%%%%%%
%%%%%%%%%%%%%%%%%%%%%%%%%%%%%%%%%%%%%%%%%%%%%%%%%%%%%%%%%%%%%%%%%%%%%%%%%%%%%%%%%%%%%%%%%%
%%%%%%%%%%%%%%%%%%%%%%%%%%%%%%%%%%%%%%%%%%%%%%%%%%%%%%%%%%%%%%%%%%%%%%%%%%%%%%%%%%%%%%%%%%

\subsection{Aplikační vrstva Wireless M-Bus}

V souladu se specifikací OMS~(Open Metering Standard)~3.0.1~\cite{NormaOMS}, která vychází z normy EN 13757-4~\cite{Norma4} pro bezdrátový protokol WM-Bus, jsou některé položky aplikační vstvy shodné pro většinu zařízení protokolu WM-Bus.


\subsubsection{Access Number}
\begin{itemize}
	\item Binárně kódované pořadí přístupu.
	\item Při každém odeslání paketu je jeho hodnota zvýšena o jedničku.
	\item Po dosažení hodnoty 254 se začíná odznova.
\end{itemize}

\subsubsection{Status}
\begin{itemize}
	\item Obsahuje chyby vysílajícího zařízení.
	\item Může nastat i několik chyb zároveň.
	\item Definované chyby jsou uvedeny v Tab.~\ref{def_status}.
\end{itemize}

				\begin{table}[!ht]
				\centering
				\vspace{-10pt}
				\caption{Hodnoty Status pole}
				\label{def_status}
				\begin{tabular}{|c|c|l|}
				\hline
				\textbf{Bit}       & \textbf{Hex hodnota} & \textbf{Význam}        \\ \hline \hline
				\multirow{2}{*}{0} & 00h                  & Žádná chyba            \\ \cline{2-3} 
													 & 01h                  & Aplikace zaneprázdněna \\ \hline
				\multirow{2}{*}{1} & 02h                  & Obecná chyba aplikace  \\ \cline{2-3} 
													 & 03h                  & Neočekávaný stav       \\ \hline
				2                  & 04h                  & Vybitá baterie         \\ \hline
				3                  & 08h                  & Trvalá chyba           \\ \hline
				4                  & 10h                  & Dočasná chyba          \\ \hline
				5                  & 20h                  & Specifický kód výrobce \\ \hline
				6                  & 40h                  & Specifický kód výrobce \\ \hline
				7                  & 80h                  & Specifický kód výrobce \\ \hline \hline
				\end{tabular}
				\end{table}


Struktura zbytku aplikační vrstvy je dána opakováním určité sekvence (viz Tab.~\ref{TableStrukturaDat}) bajtů, určující typ a hodnotu přenášených dat.
\begin{table}[!ht]
\centering
\caption{Struktura dat aplikační vrstvy}
\label{TableStrukturaDat}
\begin{tabular}{|c|c|c|c|c|}
\hline
\textbf{Byte 1}                                                                & \textbf{Byte 2}                                                                             & \textbf{Byte 3}                                                                & \textbf{Byte 4}                                                                             & \textbf{Byte 5-n}                                                      \\ \hline \hline
\multicolumn{2}{|c|}{\begin{tabular}[c]{@{}c@{}}Data Information \\ Block (DIB)\end{tabular}}                                                                                & \multicolumn{2}{c|}{\begin{tabular}[c]{@{}c@{}}Value Information \\ Block (VIB)\end{tabular}}                                                                                & \multirow{2}{*}{\begin{tabular}[c]{@{}c@{}}Data\\ Values\end{tabular}} \\ \cline{1-4}
\begin{tabular}[c]{@{}c@{}}Data  \\ Information \\ Field\\  (DIF)\end{tabular} & \begin{tabular}[c]{@{}c@{}}Data \\ Information \\ Field \\ Extension \\ (DIFE)\end{tabular} & \begin{tabular}[c]{@{}c@{}}Value \\ Information \\ Field \\ (VIF)\end{tabular} & \begin{tabular}[c]{@{}c@{}}Data \\ Information \\ Field \\ Extension \\ (VIFE)\end{tabular} &                                                                        \\ \hline \hline
\end{tabular}
\end{table}

\vspace{-10pt}
\subsubsection{Data Information Block (DIB)}
DIB definuje typ přenášených dat a skládá se z DIF a z nepovinného DIFE.

\subsubsection{Data information Field (DIF)}
DIF definuje datový typ přenášených dat a má strukturu dle Tab.~\ref{KodovaniDIFu}.
	\begin{table}[!ht]
	\centering
	\caption{Kódování DIF Pole}
	\label{KodovaniDIFu}
	\begin{tabular}{|c|c|c|c|c|c|c|c|}
	\hline \hline
	\textbf{Bit 1}                                           & \textbf{Bit2}                                                    & \textbf{Bit 3}                         & \textbf{Bit 4}                        & \textbf{Bit 5} & \textbf{Bit 6} & \textbf{Bit 7} & \textbf{Bit 8} \\ \hline 
	\begin{tabular}[c]{@{}c@{}}Rozšiřující \\ Bit\end{tabular} & \begin{tabular}[c]{@{}c@{}}LSB uložení\end{tabular} & \multicolumn{2}{c|}{\begin{tabular}[c]{@{}c@{}}Funkční \\ Položka\end{tabular}} & \multicolumn{4}{c|}{Data}                                         \\ \hline \hline
	\end{tabular}
	\end{table}

Rozšiřující bit pole určuje jaký blok bajtů následuje po DIF. Možnosti jsou shrnuty v Tab.~\ref{KodovaniRozsBituDIFu}.

\begin{table}[!ht]
\centering
\caption{Kódování rozšiřujícího bitu DIF pole}
\label{KodovaniRozsBituDIFu}
\begin{tabular}{|c|c|}
\hline
\textbf{Bit} & \textbf{Další informace je obsažena v} \\ \hline \hline
0            & DIF                                    \\ \hline
1            & DIFE                                   \\ \hline \hline
\end{tabular}
\end{table}

Funkční pole definuje typ přenášené hodnoty z hlediska její aktuálnosti či limitnosti. Možnosti jsou shrnuty v Tab.~\ref{KodovaniFunkPoleDIFu}.

\begin{table}[!ht]
\centering
\caption{Kódování funkčního pole DIF pole}
\label{KodovaniFunkPoleDIFu}
\begin{tabular}{|c|l|}
\hline
\textbf{Hodnota} & \multicolumn{1}{c|}{\textbf{Význam}} \\ \hline \hline
00b              & Okamžitá hodnota                     \\ \hline
01b              & Minimální hodnota                    \\ \hline
10b              & Maximální hodnota                    \\ \hline
11b              & Hodnota při chybovém stavu         \\ \hline \hline
\end{tabular}
\end{table}

Data pole určuje datový typ přenášené hodnoty. Možnosti jsou shrnuty v Tab.~\ref{KodovaniDataPoleDIFu}.

\begin{table}[!ht]
\centering
\caption{Kódování Data pole DIF pole}
\label{KodovaniDataPoleDIFu}
\begin{tabular}{|c|c|c|c|c|}
\hline
\textbf{Délka hodnoty {[}b{]}} & \textbf{Kód} & \textbf{Význam} & \textbf{Kód} & \textbf{Význam}   \\ \hline \hline
0                              & 0000         & Žádná data      & 1000         & Volba pro hodnotu \\ \hline
8                              & 0001         & 8-bit Integer   & 1001         & 2 cifry BCD       \\ \hline
16                             & 0010         & 16-bit Integer  & 1010         & 4 cifry BCD       \\ \hline
24                             & 0011         & 24-bit Integer  & 1011         & 6 cifer BCD       \\ \hline
32                             & 0100         & 32-bit Integer  & 1100         & 8 cifer BCD       \\ \hline
32                             & 0101         & 32-bit Real     & 1101         & Proměnlivá délka  \\ \hline \hline
\end{tabular}
\end{table}

\subsubsection{Data Information Field Extension (DIFE)}
DIFE obsahuje upřesnění veličiny či informace o tarifu dle struktury zobrazené v Tab.~\ref{KodovaniDIFE}.

\begin{table}[!ht]
\centering
\caption{Kódování DIFE Pole}
\label{KodovaniDIFE}
\begin{tabular}{|c|c|c|c|c|c|c|c|}
\hline \hline
\textbf{Bit 7}  & \textbf{Bit 6} & \textbf{Bit 5} & \textbf{Bit 4} & \textbf{Bit 3} & \textbf{Bit 2} & \textbf{Bit 1} & \textbf{Bit 0} \\ \hline 
Rozšiřující bit & Jednotka       & \multicolumn{2}{c|}{Tarif}      & \multicolumn{4}{c|}{Hodnota}                                      \\ \hline \hline
\end{tabular}
\end{table}

\subsubsection{Value Information Block (VIB)}
VIB definuje typ přenášené hodnoty a skládá se z VIF a z nepovinného VIFE.

\subsubsection{Value Information Field (VIF)}
VIF definuje veličinu přenášených dat a má strukturu dle Tab.~\ref{KodovaniVIFu}.

\begin{table}[!ht]
\centering
\caption{Kódování VIF Pole}
\label{KodovaniVIFu}
\begin{tabular}{|c|c|c|c|c|c|c|c|}
\hline \hline
\textbf{Bit 1}  & \textbf{Bit2} & \textbf{Bit 3} & \textbf{Bit 4} & \textbf{Bit 5} & \textbf{Bit 6} & \textbf{Bit 7} & \textbf{Bit 8} \\ \hline 
Rozšiřující bit & \multicolumn{7}{c|}{Data}                                                                                           \\ \hline \hline
\end{tabular}
\end{table}

Rozšiřující bit pole určuje jaký blok bajtů následuje po VIF. Možnosti jsou shrnuty v Tab.~\ref{KodovaniRozsBituVIFu}.

\begin{table}[!ht]
\centering
\caption{Kódování rozšiřujícího bitu VIF pole}
\label{KodovaniRozsBituVIFu}
\begin{tabular}{|c|c|}
\hline
\textbf{Bit} & \textbf{Další informace je obsažena v} \\ \hline \hline
0            & VIF                                    \\ \hline
1            & VIFE                                   \\ \hline \hline
\end{tabular}
\end{table}

Data pole určuje datový typ přenášené hodnoty. Možnosti jsou shrnuty v Tab.~\ref{KodovaniDataPoleVIFu}.

\begin{table}[!ht]
\centering
\caption{Kódování Data pole VIF pole}
\label{KodovaniDataPoleVIFu}
\resizebox{\textwidth}{!}{%
\begin{tabular}{|l|l|l|l|}
\hline
\multicolumn{1}{|c|}{\textbf{Bity}} & \multicolumn{1}{c|}{\textbf{Veličina}} & \multicolumn{1}{c|}{\textbf{Jednotka}}                                                                                & \multicolumn{1}{c|}{\textbf{Rozsah}} \\ \hline \hline
E000 0nnn                             & Energie                                    & 10\textsuperscript{(nnn-3)}\,Wh                                                                                                              & 0.001\,Wh - 10000\,Wh \\ \hline
E000 1nnn                             & Energie                                    & 10\textsuperscript{(nnn)}\,J                                                                                                                 & 0.001\,kJ - 10000 \,kJ                \\ \hline
E001 0nnn                             & Objem                                    & 10\textsuperscript{(nnn-6)}\,m\textsuperscript{3}                                                                                                              & 0.001\,l - 10000\,l                    \\ \hline
E001 1nnn                             & Hmotnost                                      & 10\textsuperscript{(nnn-3)}\,kg                                                                                                              & 0.001\,kg - 10000\,kg                \\ \hline
\multirow{2}{*}{E010 00nn}            & \multirow{2}{*}{Provozní čas}                  & \multirow{4}{*}{\begin{tabular}[c]{@{}l@{}}nn = 00 sekundy\\ nn = 01 minuty\\ nn = 10 hodiny\\ nn = 11 dny\end{tabular}} & \multirow{2}{*}{}                   \\
                                      &                                           &                                                                                                                           &                                     \\ \cline{1-2} \cline{4-4} 
\multirow{2}{*}{E010 01nn}            & \multirow{2}{*}{Operační čas}            &                                                                                                                           & \multirow{2}{*}{}                   \\
                                      &                                           &                                                                                                                           &                                     \\ \hline
E010 1nnn                             & Výkon                                     & 10\textsuperscript{(nnn-3)}\,W                                                                                                               & 0.001\,W - 10000\,W                  \\ \hline
E011 0nnn                             & Výkon                                     & 10\textsuperscript{(nnn)}\,J/h                                                                                                               & 0.001\,kJ/h - 10000\,kJ/h            \\ \hline
E011 1nnn                             & Průtok                              & 10\textsuperscript{(nnn-6)}\,m\textsuperscript{3}/h                                                                                                            & 0.001\,l/h - 10000 l/h              \\ \hline
E100 0nnn                             & Průtok                           & 10\textsuperscript{(nnn-7)}\,m\textsuperscript{3}/min                                                                                                          & 0.0001\,l/min - 1000 l/min          \\ \hline
E100 1nnn                             & Průtok                           & 10\textsuperscript{(nnn-9)}\,m\textsuperscript{3}/s                                                                                                            & 0.001\,ml/s - 10000 ml/s            \\ \hline
E101 0nnn                             & Protok (hmotnosti)                              & 10\textsuperscript{(nnn-3)} kg/h                                                                                                            & 0.001\,kg/h - 10000 kg/h            \\ \hline
E101 10nn                             & Teplota (průtoku)                           & 10\textsuperscript{(nn-3)}\,°C                                                                                                               & 0.001\,°C - 1\,°C                    \\ \hline
E101 11nn                             & Teplota (návratová)                         & 10\textsuperscript{(nn-3)}\,°C                                                                                                               & 0.001\,°C - 1\,°C                    \\ \hline
E110 00nn                             & Teplota (rozdíl)                    & 10\textsuperscript{(nn-3)}\,K                                                                                                                & 1\,mK - 1000\,mK                     \\ \hline
E110 01nn                             & Temperature (externí)                       & 10\textsuperscript{(nn-3)}\,°C                                                                                                               & 0.001\,°C - 1\,°C                    \\ \hline
E110 10nn                             & Tlak                                 & 10\textsuperscript{(nn-3)}\,bar                                                                                                              & 1\,mbar - 1000\,mbar                 \\ \hline
\multirow{2}{*}{E110 110n}                             & \multirow{2}{*}{Datum a čas}                                & \multirow{2}{*}{\begin{tabular}[c]{@{}l@{}}n = 0 datum\\ n = 1 datum a čas\end{tabular}}                                                                                            & \multirow{2}{*}{Datový typ F a G}             \\ 
                                      &                                           &                                                                                                                           &                                     \\ \hline
E110 1110                             & Tepelná výměna                          &                                                                                                                           & bezrozměrné                 \\ \hline
E110 1111                             & Rezervováno                                 &                                                                                                                           &                                     \\ \hline
\multirow{2}{*}{E111 00nn}            & \multirow{2}{*}{Průměrné trvání}        & \multirow{4}{*}{\begin{tabular}[c]{@{}l@{}}nn = 00 sekundy\\ nn = 01 minuty\\ nn = 10 hodiny\\ nn = 11 dny\end{tabular}} & \multirow{2}{*}{}                   \\
                                      &                                           &                                                                                                                           &                                     \\ \cline{1-2} \cline{4-4} 
\multirow{2}{*}{E111 01nn}            & \multirow{2}{*}{Aktuální trvání}        &                                                                                                                           & \multirow{2}{*}{}                   \\
                                      &                                           &                                                                                                                           &                                     \\ \hline
E111 1000                             & Výrobní číslo                            &                                                                                                                           &                 \\ \hline
E111 1001                             & Rozšířená identifikace                &                                                                                                                           & Datový typ C                    \\ \hline
E111 1010                             & Adresa sběrnice                               &                                                                                                                           & Datový typ C                    \\ \hline \hline
\end{tabular}}
\end{table}

\subsubsection{Value Information Field Extension (VIFE)}
VIFE obsahují upřesnění, doplňující informaci či přenos chybového stavu dané položky. Jejich kompletní přehled je uveden ve specifikaci~\cite{WmBusSpecka}.

\begin{table}[!ht]
\centering
\caption{Kódování VIFE Pole}
\label{KodovaniVIFE}
\begin{tabular}{|c|c|c|c|c|c|c|c|}
\hline \hline
\textbf{Bit 7}  & \textbf{Bit 6} & \textbf{Bit 5} & \textbf{Bit 4} & \textbf{Bit 3} & \textbf{Bit 2} & \textbf{Bit 1} & \textbf{Bit 0} \\ \hline 
Rozšiřující bit & Jednotka       & \multicolumn{2}{c|}{Tarif}      & \multicolumn{4}{c|}{Hodnota}                                      \\ \hline \hline
\end{tabular}
\end{table}

\subsubsection{Data Value}
Pole \texttt{Data Value} již obsahuje přenášenou hodnotu, definovanou dle DIB a VIB.


\subsubsection{Datové typy F a G}
V protokolu Wireless M-Bus je datum kódováno ve formátu G, viz Tab.~\ref{KodovaniDataG} a datum i čas ve formátu F, viz Tab.~\ref{KodovaniDataF}.  

\begin{table}[!ht]
\centering
\caption{Kódování data ve formátu G}
\label{KodovaniDataG}
\begin{tabular}{|c|c|c|c|c|c|c|c|c|c|c|c|c|c|c|c|}
\hline
\rotatebox[origin=c]{90}{\parbox[b]{1.75cm}{\hspace{10pt}\textbf{Bit 7}}} & 
\rotatebox[origin=c]{90}{\parbox[b]{1.75cm}{\hspace{10pt}\textbf{Bit 6}}} & 
\rotatebox[origin=c]{90}{\parbox[b]{1.75cm}{\hspace{10pt}\textbf{Bit 5}}} &
\rotatebox[origin=c]{90}{\parbox[b]{1.75cm}{\hspace{10pt}\textbf{Bit 4}}} & 
\rotatebox[origin=c]{90}{\parbox[b]{1.75cm}{\hspace{10pt}\textbf{Bit 3}}} & 
\rotatebox[origin=c]{90}{\parbox[b]{1.75cm}{\hspace{10pt}\textbf{Bit 2}}} &
\rotatebox[origin=c]{90}{\parbox[b]{1.75cm}{\hspace{10pt}\textbf{Bit 1}}} & 
\rotatebox[origin=c]{90}{\parbox[b]{1.75cm}{\hspace{10pt}\textbf{Bit 0}}} & 
\rotatebox[origin=c]{90}{\parbox[b]{1.75cm}{\hspace{10pt}\textbf{Bit 7}}} &
\rotatebox[origin=c]{90}{\parbox[b]{1.75cm}{\hspace{10pt}\textbf{Bit 6}}} & 
\rotatebox[origin=c]{90}{\parbox[b]{1.75cm}{\hspace{10pt}\textbf{Bit 5}}} & 
\rotatebox[origin=c]{90}{\parbox[b]{1.75cm}{\hspace{10pt}\textbf{Bit 4}}} &
\rotatebox[origin=c]{90}{\parbox[b]{1.75cm}{\hspace{10pt}\textbf{Bit 3}}} & 
\rotatebox[origin=c]{90}{\parbox[b]{1.75cm}{\hspace{10pt}\textbf{Bit 2}}} & 
\rotatebox[origin=c]{90}{\parbox[b]{1.75cm}{\hspace{10pt}\textbf{Bit 1}}} &
\rotatebox[origin=c]{90}{\parbox[b]{1.75cm}{\hspace{10pt}\textbf{Bit 0}}} \\ \hline \hline
\multicolumn{8}{|c|}{Bajt 1} & \multicolumn{8}{c|}{Bajt 2} \\ \hline
\multicolumn{3}{|c|}{Rok (1/2)} & \multicolumn{5}{c|}{Den} & \multicolumn{4}{c|}{Rok (2/2)} & \multicolumn{4}{c|}{Měsíc} \\ \hline \hline
\end{tabular}
\end{table}


\begin{table}[!ht]
\centering
\caption{Kódování data a času ve formátu F}
\label{KodovaniDataF}
\begin{tabular}{|c|c|c|c|c|c|c|c|c|c|c|c|c|c|c|c|c|c|}
\hline
\rotatebox[origin=c]{90}{\parbox[b]{1.75cm}{\hspace{10pt}\textbf{Bit 7}}} & 
\rotatebox[origin=c]{90}{\parbox[b]{1.75cm}{\hspace{10pt}\textbf{Bit 6}}} & 
\rotatebox[origin=c]{90}{\parbox[b]{1.75cm}{\hspace{10pt}\textbf{Bit 5}}} &
\rotatebox[origin=c]{90}{\parbox[b]{1.75cm}{\hspace{10pt}\textbf{Bit 4}}} & 
\rotatebox[origin=c]{90}{\parbox[b]{1.75cm}{\hspace{10pt}\textbf{Bit 3}}} & 
\rotatebox[origin=c]{90}{\parbox[b]{1.75cm}{\hspace{10pt}\textbf{Bit 2}}} &
\rotatebox[origin=c]{90}{\parbox[b]{1.75cm}{\hspace{10pt}\textbf{Bit 1}}} & 
\rotatebox[origin=c]{90}{\parbox[b]{1.75cm}{\hspace{10pt}\textbf{Bit 0}}} & 
\rotatebox[origin=c]{90}{\parbox[b]{1.75cm}{\hspace{10pt}\textbf{Bit 7}}} &
\rotatebox[origin=c]{90}{\parbox[b]{1.75cm}{\hspace{10pt}\textbf{Bit 6}}} & 
\rotatebox[origin=c]{90}{\parbox[b]{1.75cm}{\hspace{10pt}\textbf{Bit 5}}} & 
\rotatebox[origin=c]{90}{\parbox[b]{1.75cm}{\hspace{10pt}\textbf{Bit 4}}} &
\rotatebox[origin=c]{90}{\parbox[b]{1.75cm}{\hspace{10pt}\textbf{Bit 3}}} & 
\rotatebox[origin=c]{90}{\parbox[b]{1.75cm}{\hspace{10pt}\textbf{Bit 2}}} & 
\rotatebox[origin=c]{90}{\parbox[b]{1.75cm}{\hspace{10pt}\textbf{Bit 1}}} &
\rotatebox[origin=c]{90}{\parbox[b]{1.75cm}{\hspace{10pt}\textbf{Bit 0}}} & & \\ \hline \hline
\multicolumn{8}{|c|}{Bajt 1}                                  & \multicolumn{8}{c|}{Bajt 2}                                   & Bajt 3           & Bajt 4          \\ \hline
0     & 0     & \multicolumn{6}{c|}{Hodina}                   & 0     & 0     & 0     & \multicolumn{5}{c|}{Minuta}           & \multicolumn{2}{c|}{Dle formátu G} \\ \hline \hline
\end{tabular}
\end{table}

Ukázky implementace obou formátů v jazyce Python uvádí Kód~\ref{CodeFG}.

\begin{lstlisting}[caption={Implementace F a G formátu},captionpos=b,label=CodeFG,style=MyCodePython]
# Get date in G format
def get_date(date_bytes):
    date = str(bin(int(date_bytes[0:2], 16))[2:]).zfill(8) + str(bin(int(date_bytes[2:4], 16))[2:]).zfill(8)
    year = str(int(date[0:4] + date[8:11], 2))
    month = str(int(date[4:8], 2))
    day = str(int(date[11:16], 2))
    vysledek = day + "." + month + ".20" + year
    return vysledek

# Get time from F format
def get_time(time_bytes):
    time = str(bin(int(time_bytes[0:2], 16))[2:]).zfill(8) + str(bin(int(time_bytes[2:4], 16))[2:]).zfill(8)
    hour = str(int(time[3:8], 2))
    minute = str(int(time[10:16], 2)).zfill(2)
    vysledek = hour + ":" + minute
    return vysledek
\end{lstlisting}
	
\newpage{}	
	
\section{Šifrování dat}
Pro šifrování přenášených dat se v protokolu Wireless M-Bus používají tři šifrovací algoritmy:
\begin{itemize}
	\item DES (Data Encryption Standard) bez inicializačního vektoru,
	\item DES s inicializačním vektorem a
	\item AES (Advanced Encryption Standard) s inicializačním vektorem.
\end{itemize}

Šifrování DES dnes již není moc využívané, je již nedostačující a zastaralé. Drtivá většina dnešních zařízení umožnujících šifrovaný přenos využívá šifrování AES, konkrétně verzi AES128 CBC.

\subsection{Šifrovací algoritmus DES}
Data Encryption Standard je v kryptografii symetrická šifra vyvinutá v 70. letech. V roce 1977 byla zvolena za standard FIPS 46~\cite{NormaFIPS46}. V současnosti je tato šifra považována za nespolehlivou, protože používá klíč pouze o délce 64 bitů, z toho 8 je kontrolních a 56 efektivních. Navíc algoritmus obsahuje slabiny, které dále snižují bezpečnost šifry. Díky tomu je možné šifru prolomit útokem hrubou silou za méně než 24 hodin.

\subsection{Šifrovací algoritmus AES}
Advanced Encryption Standard je symetrická bloková šifra (pro šifrování i dešifrování využívá stejný klíč na data s pevně danou délkou bloku), která nahradila dříve užívanou šifru DES~\cite{NormaFIPS}. AES šifra je rychlá v softwaru i hardwaru a na rozdíl od~svého předchůdce DES nepoužívá Feistelovu síť. AES má pevně danou velikost bloku na 128 bitů a velikost klíče na 128, 192 nebo 256 bitů.   Pokud jsou šifrovaná data delší, zpracovávají se po jednotlivých blocích. 

\begin{figure}[!ht]
\vspace{-10pt}
 \begin{center}
    \includegraphics[scale=0.65]{obrazky/wmbus_aes_cbc}
  \end{center}
	\vspace{-30pt}
  \caption{Princip algoritmu AES v módu CBC}
	\label{SchemaAEScbc}
	\vspace{-10pt}
\end{figure}

Pro šifrovaný přenos dat v protokolu WM-Bus se využívá AES, kontrétně mód s~inicializačním vektorem (CBC - Cipher Block Chaining). Ten funguje (viz~Obr.~\ref{SchemaAEScbc}) tak, že před zašifrováním se odpovídající blok otevřeného textu xoruje předcházejícím blokem zašifrovaného textu. To znamená, že jednotlivé bloky jsou na sobě závislé, aby došlo k dešifrování konkrétního bloku, je nutné  dešifrovat i všechny předchozí. Je tedy nutné mít nějaký nulový blok dat pro zašifrování prvního bloku dat. K tomu se využívá inicializační vektor. Tomuto bloku se pak říká inicializační vektor (IV). Tento blok se použije k dešifrování prvního bloku a pak zahodí.



\subsection{Inicializační vektor}
\label{KapitolaInicializacniVektor}
Inicializační vektor má délku 16 bajtů (128 bitů, odtud označení AES-128) a~v~případě protokolu WM-Bus je tvořený dynamicky z nešifrovaných bajtů polí paketu, způsobem popsaným v Tab.~\ref{TabulkaInicializacniVektor} a implemetovaných dle Kódu~\ref{CodeIV}.

\begin{lstlisting}[caption={Sestavení inicializačního vektoru},captionpos=b,label=CodeIV,style=MyCodePython]
# Build Initialization Vector from incoming packet data
device = parsedstring[8:24].upper()
access = str(parsedstring[26:28])
AES_IV = binascii.unhexlify(device + access * 8)
\end{lstlisting}

\begin{table}[!ht]
	%\vspace{-30pt}
  \caption{Formát inicializačního vektoru}
	\label{TabulkaInicializacniVektor}
	\vspace{-10pt}
	\begin{center}
\begin{tabular}{|c|c|l|}
\hline
\textbf{Bit} & \textbf{Obsah}                 & \textbf{Význam}                        \\ \hline \hline
LSB          & \multirow{2}{*}{M-Pole}        & \multirow{2}{*}{Identifikace výrobce}  \\ \cline{1-1}
1            &                                &                                        \\ \hline
2            & \multirow{6}{*}{A-Pole}        & \multirow{6}{*}{Identifikace zařízení} \\ \cline{1-1}
3            &                                &                                        \\ \cline{1-1}
4            &                                &                                        \\ \cline{1-1}
5            &                                &                                        \\ \cline{1-1}
6            &                                &                                        \\ \cline{1-1}
7            &                                &                                        \\ \hline
8            & \multirow{2}{*}{Access Number} & \multirow{8}{*}{Identifikace paketu}   \\ \cline{1-1}
9            &                                &                                        \\ \cline{1-2}
10           & \multirow{2}{*}{Access Number} &                                        \\ \cline{1-1}
11           &                                &                                        \\ \cline{1-2}
12           & \multirow{2}{*}{Access Number} &                                        \\ \cline{1-1}
13           &                                &                                        \\ \cline{1-2}
14           & \multirow{2}{*}{Access Number} &                                        \\ \cline{1-1}
MSB          &                                &                                        \\ \hline \hline
\end{tabular}
\vspace{-20pt}
\end{center}
\end{table}

První 2 bajty obsahují přidělené identifikační údaje výrobce, další čtyři obsahují sériové číslo daného zařízení, následující dva obsahují verzi zařízení a zbylých osm Bajtů je tvořeno opakováním se přístupového čísla. Vzhledem k faktu, že přístupové číslo se s každým vysláním telegramu změní, je nutné inicializační vektor přepočítat pro každý přijatý paket. Tím je zajištěna dynamičnost šifrování danou metodou.

\subsection{Šifrovací klíč}
Šifrovací klíč AES je sekvence bajtů o velikosti 128, 192 nebo 256 bitů. Tento klíč slouží pro šifrování a dešifrování přenášených dat a je unikátní pro každé vyčítané zařízení. Bez znalosti tohoto klíče nelze tedy vyčítat zařízení se šifrovaným přenosem dat.

\newpage{}

\subsection{Určení šifrovaných dat}
\label{KapitolaConfigurationWord}
Aplikační vrstva protokolu WM-Bus obsahuje položku \texttt{ConfigurationWord} případně \texttt{SignatureField}, která deklaruje typ použitého šifrovací algoritmu, délku šifrované části a způsob datového šifrování. Pole je složeno ze dvou bajtů. První bajt obsahuje \texttt{NNNNCCHHb} a druhý bajt obsahuje \texttt{BAS0MMMMb}. Význam jednotlivých položek je shrnut v Tab.~\ref{TableConfigurationWord}.


\begin{table}[!ht]
\centering
\caption{Význam bitů pole ConfigurationWord}
\label{TableConfigurationWord}
\begin{tabular}{|c|c|l|}
\hline
\textbf{Bit} & \textbf{Označení} & \textbf{Význam}        \\ \hline \hline
MSB          & B                 & Obousměrnost           \\ \hline
14           & A                 & Dostupnost             \\ \hline
13           & S                 & Synchronizace          \\ \hline
12           & 0                 & Synchronizace          \\ \hline
11           & M                 & Šifrování              \\ \hline
10           & M                 & Šifrování              \\ \hline
9            & M                 & Šifrování              \\ \hline
8            & M                 & Šifrování              \\ \hline
7            & N                 & Počet kódovaných bloků \\ \hline
6            & N                 & Počet kódovaných bloků \\ \hline
5            & N                 & Počet kódovaných bloků \\ \hline
4            & N                 & Počet kódovaných bloků \\ \hline
3            & C                 & Obsah telegramu        \\ \hline
2            & C                 & Obsah telegramu        \\ \hline
1            & H                 & Počítač skoků          \\ \hline
LSB          & H                 & Počítač skoků          \\  \hline  \hline
\end{tabular}
%\vspace{-5pt}
\end{table}

Bity šifrování nabývají těchto hodnot:
\begin{itemize}
	\item 4 pro AES se statickým inicializačním vektorem,
	\item 5 pro AES s dynamickým inicializačním vektorem,
	\item 6 je rezervovaná,
	\item 7 až 15 jsou pro využití výrobcem.
\end{itemize}
Pro režim AES s dynamickým inicializačním vektorem bity jsou 0101 a vyjadřují hodnotu 5.

\newpage{}

\subsection{Princip dešifrování}
Pro dešifrování přijatých dat je nutná znalost šifrovacího algoritmu, šifrovacího klíče a sestavení inicializačního vektoru. Potom lze aplikací dešifrovacího algoritmu získat přenášená data. Obecné schéma dešifrování AES-128 CBC je znázorněno na~Obr.~\ref{SchemaAESobecne} a~implementováno v~Kódu~\ref{CodeCrypto}.
\begin{figure}[!ht]
\vspace{-20pt}
 \begin{center}
    \includegraphics[scale=0.8]{obrazky/wmbus_aes_schema}
  \end{center}
	\vspace{-30pt}
  \caption{Obecné schéma dešifrování AES-128 CBC}
	\label{SchemaAESobecne}
	\vspace{-20pt}
\end{figure}

\begin{lstlisting}[caption={Implementace AES},captionpos=b,label=CodeCrypto,style=MyCodePython]
from Crypto.Cipher import AES

encryptor = AES.new(AES_KEY_IQRF, AES.MODE_CBC, IV=AES_IV)
OUTPUT_DECRYPTED = encryptor.decrypt(INPUT_ENCRYPTED)
\end{lstlisting}

\subsection{Kontrola rozšifrování dat}
Ke kontrole správnosti dešifrovaných dat slouží definovaná počáteční sekvence dat. U algoritmu DES začínají dešifrovaná data dvěma bajty obsahujícími datum a čas. Pro algoritmus AES jsou první dva bajty šestnáckové a oba obsahují znak 2Fh, jak znázorňuje Kód~\ref{CodeVerify}.

\begin{lstlisting}[caption={Ověření kontrolní sekvence AES},captionpos=b,label=CodeVerify,style=MyCodePython]
# Verify control sequence after decrypt
aes_control = binascii.hexlify(TELEGRAM_ORIGINAL[0:2]).upper()
if (aes_control == b'2F2F'):
		binascii.hexlify(TELEGRAM_ORIGINAL).upper().decode('ascii'))
\end{lstlisting}



%% Vložení kapitoly o vycitanem zarizeni
\chapter{Vyčítaná Wireless M-Bus zařízení}

Pro účely testování komunikace bylo využito několik typů dostupných zařízení:
\begin{itemize}
	\item pokojové čidlo teploty a vlhkosti Weptech OMST-868A~\cite{CidloWeptech},
	\item modul pro vodoměry Bonega~\cite{CidloBonega},
	\item ultrazvukový měřič tepla a chladu Kamstrup Multical 402~\cite{CidloKamstrup},
	\item třífázový elektroměr ZPA ZE302~\cite{CidloZpa},
	\item obecná zařízení Pikkerton~\cite{CidloPikkerton}.
\end{itemize}

Všechna tyto zařízení poskytují formát dat dle platné specifikace OMS~(Open Metering Standard)~3.0.1~\cite{NormaOMS}, která vychází z normy EN 13757-4~\cite{Norma4} pro bezdrátový protokol WM-Bus.

Pro základní komunikaci bylo zvoleno čidlo teploty a vlhkosti Weptech OMST-868A, z důvodu volné dostupnosti kompletní dokumentace a možnosto nastavení parametrů vysílání včetně volitelného šifrování přenášených dat. Jako jediné z výše jmenovaných čidel nevyžaduje ke své činnosti žádná doplňující média či přístroje.
	
%%%%%%%%%%%%%%%%%%%%%%%%%%%%%%%%%%%%%%%%%%%%%%%%%%%%%%%%%%%%%%%%%%%%%%%%%%%%%%%%%%%%%%%%%%%%%%%%%%%%%%%%%%%%%%%%%%%%%%%%%%%%%%%%%%%%%%%%%
%%%%%%%%%%%%%%%%%%%%%%%%%%%%%%%%%%%%%%%%%%%%%%%%%%%%%%%%%%%%%%%%%%%%%%%%%%%%%%%%%%%%%%%%%%%%%%%%%%%%%%%%%%%%%%%%%%%%%%%%%%%%%%%%%%%%%%%%%
	
	
	\section{Weptech OMST-868A}
	
	Weptech OMST-868A je teplotní a vlhkostní čidlo podporující protokol Wireless M-Bus. Je určeno pro vnitřní využití a proto je dodáváno v pouzdře určeném pro montáž na zeď.
	
	 \begin{figure}[!h]
  \begin{center}
    \includegraphics[scale=0.35]{obrazky/zarizeni_weptech}
  \end{center}
  \caption{Čidlo Weptech OMST-868A~\cite{CidloWeptech}}
\end{figure}


\subsubsection{Parametry čidla}
\begin{itemize}
	\item Rozsah měření vlhkosti: 20 až 80\,\%.
	\item Přesnost měření vlhkosti: ± 2\,\%.
	\item Rozsah měření teploty: -10\,°C až 55\,°C.
	\item Přesnost měření teploty: ± 0,3\,°C.
	\item Teplotní hystereze: 0,1\,°C.
	\item Mód přenosu: S nebo T.
	\item Interval přenosu: konfiugurovatelný v rozsahu 5~sekund až 24~hodin.
	\item Šifrování přenosu: volitelný AES-128~mód~5.
	\item Napájení: 2~x~AA baterie.
	\item Výdrž baterie: dle módu a intervalu přenosu až 10~let.
\end{itemize}


\subsubsection{Formát telegramu}

Telegram má specifickou základní strukturu popsanou v Tab.~\ref{TabulkaTelegramWeptech}~\cite{CidloWeptech}:

\begin{table}[!ht]
\centering
\caption{Telegram ze zařízení Weptech 868A~\cite{CidloWeptech}}
\label{TabulkaTelegramWeptech}
\resizebox{\textwidth}{!}{%
\begin{tabular}{|c|l|c|}
\hline
\textbf{POLE}      & \textbf{POPIS}            & \textbf{HODNOTA} \\ \hline
L-Field            & Délka telegramu                                & 1Eh              \\ \hline
C-Field            & Typ telegramu                                  & 44h              \\ \hline
M-Field            & Výrobce zařízení                               & B0h              \\ \hline
M-Field            & Výrobce zařízení                               & 5Ch              \\ \hline
A-Field            & Sériové číslo                                  & 11h              \\ \hline
A-Field            & Sériové číslo                                  & 47h              \\ \hline
A-Field            & Sériové číslo                                  & 15h              \\ \hline
A-Field            & Sériové číslo                                  & 08h              \\ \hline
A-Field            & Verze zařízení                                 & 01h              \\ \hline
A-Field            & Typ zařízení                                   & 1Bh              \\ \hline
Ci-pole            & Odpověd od zařízení                            & 7Ah              \\ \hline
Access Number      & Číslo přístupu                                 & 41h              \\ \hline
Status             & Status zařízení                                & 00h              \\ \hline
Configuration word & Konfigurační řetězec AES                       & 00h              \\ \hline
Configuration word & Konfigurační řetězec AES                       & 00h              \\ \hline
AES verification   & Ověření AES                                  & 2Fh              \\ \hline
AES verification   & Ověření AES                                  & 2Fh              \\ \hline
DR1                & DIF: 4 cifry BCD                               & 0Ah              \\ \hline
DR1                & VIF: teplota ve stupních Celsia na mínus první & 66h              \\ \hline
DR1                & Hodnota teploty                                & 99h              \\ \hline
DR1                & Hodnota teploty                                & 01h              \\ \hline
DR2                & DIF: 4 cifry BCD                               & 0Ah              \\ \hline
DR2                & VIF: První rozšiřovací tabulka                 & FBh              \\ \hline
DR2                & VIFE: vlhkost procentech na mínus první        & 1Ah              \\ \hline
DR2                & Hodnota relativní vlhkosti                               & 93h              \\ \hline
DR2                & Hodnota relativní vlhkosti                               & 02h              \\ \hline
DR3                & DIF: 16bit integer/binary                      & 02h              \\ \hline
DR3                & VIF: Druhá rozšiřovací tabulka                 & FDh              \\ \hline
DR3                & VIFE0: Chybové stavy                           & 97h              \\ \hline
DR3                & VIFE1: Norma 																	& 1Dh              \\ \hline
DR3                & Příznak sabotáže                               & 00h              \\ \hline
DR3                & Příznak vybité baterie                         & 00h              \\ \hline
Fill               & Výplňové byty (13x)                                 & 2Fh              \\ \hline

\end{tabular}}
\end{table}

\newpage{}

Některé z položek je potřeba blíže vysvětlit:

\begin{itemize}
	\item Access number - Toto číslo se s každým požadavkem zvýší o hodnotu 1.
	\item Status - V případě úspěšného přenosu je zde uložena nula, v opačném případě je zde uložena logická jednička a nastal tedy chybový stav 'sabotáž' nebo 'vybitá baterie'.
	\item Configuration word - V případě zapnutého šifrovaní, první bajt obsahuje počet zašifrovaných bloků, obsah telegarmu a inkrement. Druhý bajt obsahuje záznam o obousměrnosti, dostupnosti, synchronizaci a šifrování. Pokud je šifrování zapnuto, je nastaven mód 5, v opačném případě jsou oba bajty nulové.
	\item Příznak sabotáže čidla - Pokud čidlo pomocí integrovaného spínače detekuje uvolnění krytu z montážní desky, pošle výstrahu přes rádio do přijímače, tedy změní pro nejbližší a všechny následující vysílání tamper bit v telegramu. Tento bit slouží jako ochrana před neoprávněnou manipulací s čidlem a může být vymazán pouze restartem zařízení. Tedy vyjmutím starých baterií, ponecháním zařízení několik minut bez napájení, aby došlo k vybití všech kondenzátorů a následným vložením baterií.
\item Příznak vybité baterie - Pokud elektronika v čidlu vyhodnotí úroveň nabití baterie jako nedostatečnou, nastaví bit vybití baterie do sekce chyb v telegramu. Tento bit ošetřuje stavy, kdy nedostatečně nabité baterie způsobí příliš velký rozptyl naměřených hodnot, v krajních případech i mimo měřící rozsah čidla. Tento bit může být vymazán také pouze restartem zařízení, jako v předchozím případě.
\item Položky hodnota teploty, hodnota vlhkosti, výrobce zařízení a sériové číslo jsou uloženy v kódování big-endian, tedy na paměťové místo s nejnižší adresou se uloží nejvíce významný bajt a za něj se ukládají ostatní bajty až po nejméně významný bajt na konci. Uživatelská hodnota se tedy vyčítá pozpátku pod jednotlivých bajtech.
\item Telegram je ukončen 13 výplňovými bajty, které nenesou žádnou informaci.
\end{itemize}


\subsubsection{Nastavení čidla}
Čidlo má k dispozici několik nastavení. Některé z nich lze nastavit pomocí čtyř přepínačů DIP na desce plošných spojů.
První přepínač zapíná AES-128 šifrování, druhý přepínač přepíná mezí módem vysílání S (poloha ON) a módem T (poloha OFF), třetí a čtvrtý přpínač určují interval zasílání telegramu, jejich nastavení shrnuje Tab.~\ref{TablukaDIP}.

\newpage{}

\begin{table}[!ht]
\centering
\caption{Konfigurace intervalu zasílání pomocí DIP přepínače~\cite{CidloWeptech}}
\label{TablukaDIP}
\begin{tabular}{|c|c|c|}
\hline
\textbf{Interval zasílání} & DIP 3 & DIP 4 \\ \hline
1 minuta & ON & ON \\ \hline
5 minut & OFF & ON \\ \hline
10 minut & ON & OFF \\ \hline
15 minut & OFF & OFF \\ \hline
\end{tabular}
\end{table}

Jiné mohou být nastaveny pouze během výroby daného setu, a to do příslušnch továrních nastavení, či uživatelsky vyžádaných nastavení, viz Tab.~\ref{TablukaSETUP}.

\begin{table}[!ht]
\centering
\caption{Přehled nastavení čidla \cite{CidloWeptech}}
\label{TablukaSETUP}
\begin{tabular}{|c|l|c|}
\hline
\textbf{Parametr} & \multicolumn{1}{c|}{\textbf{Popis}} & \textbf{\begin{tabular}[c]{@{}c@{}}DIP \\ přepínač\end{tabular}} \\ \hline
AES enable & \begin{tabular}[c]{@{}l@{}}Možnost zapnutí či vypnutí šifrování \\ přenášených dat.\end{tabular} & 1 \\ \hline
%AES key & \begin{tabular}[c]{@{}l@{}}AES klíč je zapsán při výrobě zařízení, \\ nelze ho uživatelsky měnit. Hodnota \\ klíče je 00 01 02 03 04 05 06 07 08 \\ 09 0A 0B 0C 0D 0E 0F.\end{tabular} &  \\ \hline
wM-Bus mode & \begin{tabular}[c]{@{}l@{}}Implementovány jsou módy S1-m \\ a T1. Ostatní módy lze nastavit\\ pouze při tovární výrobě.\end{tabular} & 2 \\ \hline
Transmission interval & \begin{tabular}[c]{@{}l@{}}Interval je výrobcem konfigurovatelný \\ v intervalu 2 až 65534 sekund. \\ Předvolby (60s, 300s, 600s, 900s) \\ jsou uživatelsky nastavitelné pomocí \\ DIP přepínače.\end{tabular} & 3 a 4 \\ \hline
%Address & \begin{tabular}[c]{@{}l@{}}Adresa zařízení je výrobcem udaná \\ hodnota, obsahující identifikaci \\ výrobce „WEP“, sériového číslo čidla, \\ verzi zařízení (1) a typ \\ zařízení (1Bh - pokojové čidlo)\end{tabular} &  \\ \hline
%Tamper & \begin{tabular}[c]{@{}l@{}}Odezva od čidla pro otevření \\ boxu může být povolena či \\ zakázána, samotné zařízení \\ existuje i ve verzích bez \\ tohoto čidla.\end{tabular} & \multicolumn{1}{l|}{} \\ \hline
%Config & \begin{tabular}[c]{@{}l@{}}Změna nastavení pomocí DIP \\ přepínače může být povolena \\ či zakázána, samotné zařízení \\ existuje i ve verzích bez\\  tohoto přepínače.\end{tabular} & \multicolumn{1}{l|}{} \\ \hline
\end{tabular}
\end{table}


%%%%%%%%%%%%%%%%%%%%%%%%%%%%%%%%%%%%%%%%%%%%%%%%%%%%%%%%%%%%%%%%%%%%%%%%%%%%%%%%%%%%%%%%%%%%%%%%%%%%%%%%%%%%%%%%%%%%%%%%%%%%%%%%%%%%%%%%%
%%%%%%%%%%%%%%%%%%%%%%%%%%%%%%%%%%%%%%%%%%%%%%%%%%%%%%%%%%%%%%%%%%%%%%%%%%%%%%%%%%%%%%%%%%%%%%%%%%%%%%%%%%%%%%%%%%%%%%%%%%%%%%%%%%%%%%%%%
	

	\section{Bonega}

Modul Bonega je bezdrátové čidlo podporující protokol Wireless M-Bus. Jedná se o samostatné zařízení, které je určené pro montáž na vodoměry Bonega. Na řadu vodoměrů podporujících tento modul je možná i dodatečná montáž. Elektronická část modulu slouží současně pro vyčítání dvou vodoměrů, na teplou i studenou vodu.
	
	 \begin{figure}[!h]
  \begin{center}
    \includegraphics[scale=0.70]{obrazky/zarizeni_bonega}
  \end{center}
  \caption{Bezdrátový modul Bonega~\cite{CidloBonega}}
\end{figure}


\subsubsection{Parametry modulu}
\begin{itemize}
	\item Rozsah měření: 0 až 65536 m\textsuperscript{2}.
	\item Přesnost měření: ± 1\,litr.
	\item Maximální detekovatelný průtok: 6\,m\textsuperscript{3}/hod.
	\item Mód přenosu: T1.
	\item Stupeň krytí: IP64.
	\item Interval přenosu: 20-24\,sekund v odpočtovém období (od 1.11. do 15.1.)
	\item Interval přenosu: 4\,minuty mimo odpočtové období.
	\item Šifrování přenosu: AES-128~mód~5.
	\item Napájení: integrovaná baterie.
	\item Výdrž baterie: až 12~let.
\end{itemize}


\subsubsection{Formát telegramu}
Zařízení vysílá postupně dva telegramy, s rozlišením šestým bajtem adresy zařízení, jeden pro vodoměr teplé vody a druhý pro vodoměr studené vody.
Telegram má specifickou základní strukturu popsanou v Tab. \ref{TabulkaTelegram2}~\cite{CidloBonega}:

\begin{table}[!ht]
\centering
\caption{Telegram z modulu Bonega~\cite{CidloBonega}}
\resizebox{\textwidth}{!}{%
\label{TabulkaTelegram2}
\begin{tabular}{|c|l|c|}
\hline
\textbf{POLE}      & \textbf{POPIS}            & \textbf{HODNOTA} \\ \hline
L-Field            & Délka telegramu                                & 1Eh              \\ \hline
C-Field            & Typ telegramu                                  & 44h              \\ \hline
M-Field            & Výrobce zařízení                               & B0h              \\ \hline
M-Field            & Výrobce zařízení                               & 5Ch              \\ \hline
A-Field            & Sériové číslo                                  & 11h              \\ \hline
A-Field            & Sériové číslo                                  & 47h              \\ \hline
A-Field            & Sériové číslo                                  & 15h              \\ \hline
A-Field            & Sériové číslo                                  & 08h              \\ \hline
A-Field            & Verze zařízení                                 & 01h              \\ \hline
A-Field            & Typ zařízení                                   & 1Bh              \\ \hline
Ci-pole            & Odpověd od zařízení                            & 7Ah              \\ \hline
Access Number      & Číslo přístupu                                 & 41h              \\ \hline
Status             & Status zařízení                                & 00h              \\ \hline
Signature field 	 & Konfigurační řetězec AES                       & 00h              \\ \hline
Signature field    & Konfigurační řetězec AES                       & 00h              \\ \hline
Data						   & Ověření AES                                    & 2Fh              \\ \hline
Data						   & Ověření AES                                    & 2Fh              \\ \hline
Data               & DIF: 2 cifry BCD (určení formátu průtoku)      & 04h              \\ \hline
Data               & VIF: Objemový průtok v m krychlových na mínus první & 13h         \\ \hline
Data               & Hodnota průtoku                               & 99h               \\ \hline
Data               & Hodnota průtoku                               & 99h               \\ \hline
Data               & Hodnota průtoku                               & 99h               \\ \hline
Data               & Hodnota průtoku                               & 99h               \\ \hline
Data               & DIF: 2 cifry BCD (určení formátu času)         & 6Dh              \\ \hline
Data               & DIF: 2 cifry BCD                               & 6Dh              \\ \hline
Data               & Čas odeslání měření                           & 99h               \\ \hline
Data               & Čas odeslání měření                           & 99h               \\ \hline
Data               & Čas odeslání měření                           & 99h               \\ \hline
Data               & Čas odeslání měření                           & 99h               \\ \hline	
Data						   & Ověření AES                                    & 2Fh              \\ \hline
Data						   & Ověření AES                                    & 2Fh              \\ \hline

\end{tabular}}
\end{table}

\newpage{}

Modul Bonega pracuje pouze v režimu šifrování přenášenných dat pomocí AES128~mód~5. Při přenosu je tedy celá sekce data šifrována, telegram popsaný v Tab. \ref{TabulkaTelegram2} je popisován po dešifrování.

Některé z položek je potřeba blíže vysvětlit:

\begin{itemize}
	\item Access number - Toto číslo se s každým požadavkem zvýší o hodnotu 1.
	\item Status - V případě úspěšného přenosu je zde uložena nula, v opačném případě je zde uložena logická jednička a nastal tedy chybový stav 'únik vody' nebo 'vybitá baterie'.
	\item Signature field - V případě zapnutého šifrovaní, první bajt obsahuje počet zašifrovaných bloků, obsah telegarmu a inkrement. Druhý bajt obsahuje záznam o obousměrnosti, dostupnosti, synchronizaci a šifrování. Pokud je šifrování zapnuto, je nastaven mód 5, v opačném případě jsou oba bajty nulové.
	\item Hodnota průtoku - aktuální hodnota průtoku je zde vyjádřena čtyři hexadecimálními bajty v LSB pořadí.
	\item Čas odeslání měření - Datum a čas provedení posledního měření. Nejedná se očas posledního odečtu či odeslání posledního telegramu.
\end{itemize}



%%%%%%%%%%%%%%%%%%%%%%%%%%%%%%%%%%%%%%%%%%%%%%%%%%%%%%%%%%%%%%%%%%%%%%%%%%%%%%%%%%%%%%%%%%%%%%%%%%%%%%%%%%%%%%%%%%%%%%%%%%%%%%%%%%%%%%%%%
%%%%%%%%%%%%%%%%%%%%%%%%%%%%%%%%%%%%%%%%%%%%%%%%%%%%%%%%%%%%%%%%%%%%%%%%%%%%%%%%%%%%%%%%%%%%%%%%%%%%%%%%%%%%%%%%%%%%%%%%%%%%%%%%%%%%%%%%%
	
	\section{Kamstrup}
	
	Kamstrup Multical 402 je kompatkní ultrazvukový měřič tepla a chladu, tedy kombinace kalorimetru a ultrazvukového průtokoměru. Je určen k měření tepla, chladu a kombinovanemu měření tepla a chladu ve všech systémech na bázi vody s rozmetím teplot 2\,$^{\circ}$C az 130\,$^{\circ}$C. Skládá se z kalkulátoru, průtokoměru a dvou teplotních snímačů. Měřič pracuje v režimech T1 a C1  povinným šifrováním přenášených dat pomocí AES128 v módu CBC. Při každém přenosu poskytuje 9 aktuálních hodnot a 2 měsíční souhrny.
	
	 \begin{figure}[!h]
  \begin{center}
    \includegraphics[scale=0.7]{obrazky/zarizeni_kamstrup}
  \end{center}
  \caption{Kamstrup Multical 402~\cite{CidloKamstrup}}
\end{figure}

\begin{table}[!ht]
\centering
\caption{Telegram ze zařízení Kamstrup Multical 402~\cite{CidloKamstrup}}
\label{TabulkaTelegramKamstrup}
%\resizebox{\textwidth}{!}
\end{table}

	
	%%%%%%%%%%%%%%%%%%%%%%%%%%%%%%%%%%%%%%%%%%%%%%%%%%%%%%%%%%%%%%%%%%%%%%%%%%%%%%%%%%%%%%%%%%%%%%%%%%%%%%%%%%%%%%%%%%%%%%%%%%%%%%%%%%%%%%%%%
%%%%%%%%%%%%%%%%%%%%%%%%%%%%%%%%%%%%%%%%%%%%%%%%%%%%%%%%%%%%%%%%%%%%%%%%%%%%%%%%%%%%%%%%%%%%%%%%%%%%%%%%%%%%%%%%%%%%%%%%%%%%%%%%%%%%%%%%%
	
	\section{ZPA}

\colorbox[rgb]{1,0,0}{doplnit text}

	 \begin{figure}[!h]
  \begin{center}
    \includegraphics[scale=0.5]{obrazky/zarizeni_zpa}
  \end{center}
  \caption{ZPA ZE.310~\cite{CidloZpa}}
\end{figure}

\colorbox[rgb]{1,0,0}{doplnit telegram}

\begin{table}[!ht]
\centering
\caption{Telegram ze zařízení ZPA ZE.302~\cite{CidloZpa}}
\label{TabulkaTelegramKZPA}
%\resizebox{\textwidth}{!}
\end{table}
	
	
%%%%%%%%%%%%%%%%%%%%%%%%%%%%%%%%%%%%%%%%%%%%%%%%%%%%%%%%%%%%%%%%%%%%%%%%%%%%%%%%%%%%%%%%%%%%%%%%%%%%%%%%%%%%%%%%%%%%%%%%%%%%%%%%%%%%%%%%%
%%%%%%%%%%%%%%%%%%%%%%%%%%%%%%%%%%%%%%%%%%%%%%%%%%%%%%%%%%%%%%%%%%%%%%%%%%%%%%%%%%%%%%%%%%%%%%%%%%%%%%%%%%%%%%%%%%%%%%%%%%%%%%%%%%%%%%%%%
	

\section{Pikkerton}

\colorbox[rgb]{1,0,0}{doplnit obecny text}
	
	\begin{table}[!ht]
\centering
\caption{Telegram ze zařízení Pikkerton~\cite{CidloPikkerton}}
\label{TabulkaTelegramPikkerton}
%\resizebox{\textwidth}{!}
\end{table}

\colorbox[rgb]{1,0,0}{doplnit telegram}




%% Vložení kapitoly o navrhu implementace
\chapter{Návrh implementace}
Jak již bylo zmíněno na začátku práce, samotná implementace je rozdělena do dvou částí:
\begin{enumerate}
	\item Komunikace RaspberryPi přes rozšiřující desku UniPi s bezdrátovým modulem a pomocí něj s poskytnutými WM-Bus zařízeními.
	\item Zachytávání šifrované i nešifrované komunikace s WM-Bus zařízeními. Uložení, analýza a následná vizualizace zachycených dat. 
\end{enumerate}

Jelikož žádný z dostupných softwarů pro UniPi nepodporuje daný bezdrátový modul, ani UART zařízení obecně, je nutné tuto komunikaci implementovat již na úrovni operačního systému.

%%%%%%%%%%%%%%%%%%%%%%%%%%%%%%%%%%%%%%%%%%%%%%%%%%%%%%%%%%%
%%%%%%%%%%%%%%%%%%%%%%%%%%%%%%%%%%%%%%%%%%%%%%%%%%%%%%%%%%%
%%%%%%%%%%%%%%%%%%%%%%%%%%%%%%%%%%%%%%%%%%%%%%%%%%%%%%%%%%%
%%%%%%%%%%%%%%%%%%%%%%%%%%%%%%%%%%%%%%%%%%%%%%%%%%%%%%%%%%%

\section{Výběr OS}
Jako operační sytém je využita aktuální verze Raspbianu Jessie s datem vydání 2017-01-11. UART rozhraní se na RaspberryPi verze 1 a 2 nachází v /dev/ttyAMA0. To se ale v případě RaspberryPi 3 odkazuje na integrovaný BT modul a původní sériový port je zde v /dev/ttyS0. Samotné UART rozhraní je ale ve výchozím nastavení Raspbianu zakázáno.

Pro zpřístupnění UART rozhraní je nutné provést drobné úpravy jeho konfigurace:


\begin{enumerate}
	\item Nejdříve je nutné provést kompletní aktualizaci Raspbianu, tedy v konzoli spustit posloupnost příkazů:
	
	\begin{lstlisting}[style=MyCodeBash]
		sudo apt-get update
		sudo apt-get upgrade
		sudo apt-get dist-upgrade
		sudo apt-get rpi-upgrade	
	\end{lstlisting}
					
	\item Poté je potřeba v /boot/config.txt změnit položku ENABLE\_UART na hodnotu 1. Tím dojde k zpřístupnení sběrnice UART. Tato položka může být v budoucnu při aktualizaci Raspbianu přepsána, proto při prvním náznaku nefunkčnosti, je potřeba tuto položku zkontrolovat jako první.
	\item V souboru /boot/cmdline.txt je potřeba odebrat úsek textu \textit{console=ttyAMA0, 115200}, aby při startování systému nedocházelo k výpisu do seríové linky. 
	\item V případě, že se jedná o RaspberryPi verze 3, je potřeba do /boot/config.txt dopsat položku \textit{dtoverlay=pi3-miniuart-bt}, která zakáže BT na mini-UART a provede přemapování zpět na /dev/ttyAMA0. Tento krok je takto řešený z důvodu kompatibility, kdy je sériová komunikace směrována přes /dev/ttyAMAO nezávisle na použité verzi RaspberryPi.
\end{enumerate}

Po každém z těchto kroků je doporučován restart zařízení. Kroky byly otestovány pouze na výše zmíněné verzi Raspbianu a v jiných distribucích se monou mírně lišit. Úspěšnost provedení těchto kroků lze zkontrolovat pomocí zadání příkazu konzole 
	\begin{lstlisting}[style=MyCodeBash]
			sudo dmesg | grep tty
	\end{lstlisting}

jehož výstup by měl být následující:
					
	\begin{lstlisting}[style=MyCodeBash]
[    0.000974] console [tty1] enabled
[    0.130442] 20201000.uart: ttyAMA0 at MMIO 0x20201000 (irq = 81, base_baud = 0) is a PL011 rev2
	\end{lstlisting}


%%%%%%%%%%%%%%%%%%%%%%%%%%%%%%%%%%%%%%%%%%%%%%%%%%%%%%%%%%%
%%%%%%%%%%%%%%%%%%%%%%%%%%%%%%%%%%%%%%%%%%%%%%%%%%%%%%%%%%%
%%%%%%%%%%%%%%%%%%%%%%%%%%%%%%%%%%%%%%%%%%%%%%%%%%%%%%%%%%%
%%%%%%%%%%%%%%%%%%%%%%%%%%%%%%%%%%%%%%%%%%%%%%%%%%%%%%%%%%%

\section{Výběr programovacího jazyka}
Jelikož primárním jazykem využívaným na platformě RaspberryPi je Python, který již obsahuje knihovny pro sériovou komunikaci, je současný kód napsán v jazyce Python 3.

%%%%%%%%%%%%%%%%%%%%%%%%%%%%%%%%%%%%%%%%%%%%%%%%%%%%%%%%%%%
%%%%%%%%%%%%%%%%%%%%%%%%%%%%%%%%%%%%%%%%%%%%%%%%%%%%%%%%%%%
%%%%%%%%%%%%%%%%%%%%%%%%%%%%%%%%%%%%%%%%%%%%%%%%%%%%%%%%%%%
%%%%%%%%%%%%%%%%%%%%%%%%%%%%%%%%%%%%%%%%%%%%%%%%%%%%%%%%%%%

\section{Nastavení komunikačního modulu a čidla}

Před samotným vyčítáním dat bylo potřeba zjistit či nastavit přenosové parametry všech použitých zařízení:

\begin{itemize}
	\item Komunikační modul IQRF nastaven do módu T ve funkci skeneru.
	\item Čidlo Weptech je nastaveno do módu T1 s intervalem zasílání 1 minuta. 
	\item Modul Bonega je nastaven do módu T1 se zapnutým šifrováním AES128 v módu 5 a s intervalem zasílání 20-24 sekund v odpočtovém období a 4 intervalem minuty mimo odpočtové období.
	\item \colorbox[rgb]{1,0,0}{Doplnit dle ZPA a Kamstrupu}
\end{itemize}

%%%%%%%%%%%%%%%%%%%%%%%%%%%%%%%%%%%%%%%%%%%%%%%%%%%%%%%%%%%
%%%%%%%%%%%%%%%%%%%%%%%%%%%%%%%%%%%%%%%%%%%%%%%%%%%%%%%%%%%
%%%%%%%%%%%%%%%%%%%%%%%%%%%%%%%%%%%%%%%%%%%%%%%%%%%%%%%%%%%
%%%%%%%%%%%%%%%%%%%%%%%%%%%%%%%%%%%%%%%%%%%%%%%%%%%%%%%%%%%

\section{Zajištění dedikovaného běhu}
Pro zajištění běhu aplikace nezávisle na typu provozu RaspberryPi bude daný program spouštěn ihned po startu operačního systému pomocí příkazu screen. Je tedy nutné ho doinstalovat:
 
\begin{lstlisting}[style=MyCodeBash]
		sudo apt-get update
		sudo install screen		
	\end{lstlisting}


%%%%%%%%%%%%%%%%%%%%%%%%%%%%%%%%%%%%%%%%%%%%%%%%%%%%%%%%%%%
%%%%%%%%%%%%%%%%%%%%%%%%%%%%%%%%%%%%%%%%%%%%%%%%%%%%%%%%%%%
%%%%%%%%%%%%%%%%%%%%%%%%%%%%%%%%%%%%%%%%%%%%%%%%%%%%%%%%%%%
%%%%%%%%%%%%%%%%%%%%%%%%%%%%%%%%%%%%%%%%%%%%%%%%%%%%%%%%%%%

\section{Zajištění podpory šifrování}
Některá ze zařízení používají pro přenos dat šifrování AES. Pro zajištění podpory šifrování byla zvolena knihovna PyCrypto, která podporuje jak šifrování DES tak i AES. 
Umoňuje pohodlnou implementaci AES128 pomocí jazyku Python3. Na rozdíl od ostatních knihoven není závislá na balíčku OpenSSL a je součástí repozitářů Raspbianu. 

Je nutné doinstalovat nezbytné balíčky:	
 
\begin{lstlisting}[style=MyCodeBash]
		sudo apt-get update
		sudo install python-crypto
		sudo install python-dev
	\end{lstlisting}

%%%%%%%%%%%%%%%%%%%%%%%%%%%%%%%%%%%%%%%%%%%%%%%%%%%%%%%%%%%
%%%%%%%%%%%%%%%%%%%%%%%%%%%%%%%%%%%%%%%%%%%%%%%%%%%%%%%%%%%
%%%%%%%%%%%%%%%%%%%%%%%%%%%%%%%%%%%%%%%%%%%%%%%%%%%%%%%%%%%
%%%%%%%%%%%%%%%%%%%%%%%%%%%%%%%%%%%%%%%%%%%%%%%%%%%%%%%%%%%

\section{Zpracování dat}

\subsection{Nešifrovaný přenos}

Jednoduchým spuštěním komunikačního modulu v módu snifferu, byl zachycen telegram

\begin{verbatim}
	32002E44B05C10000000021B7A620800002F2F0A6699010AFB1
	A930202FD971D01002F2F2F2F2F2F2F2F2F2F2F2F2F879e0D0A
\end{verbatim}

který byl pomoci datasheetu použitého komunikačního modulu~\cite{ModulIQRF} a~čidla~\cite{CidloWeptech} analyzován, a přehledně zobrazen do Tab. \ref{PacketTableAnalysis}.

\subsection{Šifrovaný přenos}

V okamžiku kdy bylo zařízení přepnuto do šifrovaného módu dle Tab.~\ref{TablukaSETUP} byl zachycen šifrovaný telegram

\begin{verbatim}
	32002e44b05c10000000021b7ac40820053ed44a38a9c3c86f5
	8210f9b979353c39dc1d5e0c873eb81994d28c099ef1d55b008
\end{verbatim}

který byl pomoci datasheetů použitého komunikačního modulu \cite{ModulIQRF} a normy~\cite{Norma1,NormaFIPS} analyzován a byly vyparsovány položky nezbytné pro dešifrování dat:
\begin{itemize}
	\item 30-33 pro informaci použitém šifrování,
	\item 8-25 pro sestavení inicializačního vektoru a
	\item 38-93 pro šifrovanou část dat.
\end{itemize}

Poté byla daná data v souladu s normou \cite{NormaFIPS} dešifrována:


..\colorbox[rgb]{1,0,0}{popsat prakticky jak se to dešifrovalo}..


 a byl získán dešifrovaný telegram:

\begin{verbatim}
	32002e44b05c10000000021b7ac40820052F2F0A6699010AFB1
	A930202FD971D01002F2F2F2F2F2F2F2F2F2F2F2F2F879e0D0A
\end{verbatim}

Poté lze dešifrovaná data vyparsovat jako při nešifrovaném přenosu popsaném v předchozí kapitole.

\begin{table}[!ht]
\centering
\caption{Rozklíčovaný zachycený paket}
\resizebox{\textwidth}{!}{%
\label{PacketTableAnalysis}
\begin{tabular}{|c|c|c|l|l|c|c|l|}
\hline
\textbf{Pozice} & \textbf{\begin{tabular}[c]{@{}c@{}}Tele-\\ gram\end{tabular}} & \multicolumn{1}{c|}{\textbf{Pole}} & \multicolumn{1}{c|}{\textbf{Popis}} & \textbf{Hodnota} & \textbf{\begin{tabular}[c]{@{}c@{}}Číselné\\ vyjádření\end{tabular}} & \multicolumn{1}{c|}{\textbf{Význam pro uživatele}} \\ \hline
4 & 2E & L-Pole & Délka telegramu & 1Eh & 46 & Paket má 46 bytů \\ \hline
6 & 44 & C-Pole & Typ telegramu & 44h & 44 & \begin{tabular}[c]{@{}l@{}}Paket je typu \\ SND-NR\end{tabular} \\ \hline
8 & B0 & M-Pole & Výrobce zařízení & B0h & 5CB0 & \begin{tabular}[c]{@{}l@{}}Výrobcem \\ je WEPtech\end{tabular} \\ \hline
10 & 5C & M-Pole & Výrobce zařízení & 5Ch &  &  \\ \hline
12 & 10 &  A-Pole & Sériové číslo & 11h & 10 & SN je 00000010 \\ \hline
14 & 00 & A-Pole & Sériové číslo & 47h &  &  \\ \hline
16 & 00 & A-Pole & Sériové číslo & 15h &  &  \\ \hline
18 & 00 & A-Pole & Sériové číslo & 08h &  &  \\ \hline
20 & 02 & A-Pole & Verze zařízení & 01h & 2 & \begin{tabular}[c]{@{}l@{}}Druhá verze \\ zařízení\end{tabular} \\ \hline
22 & 1B & A-Pole & Typ zařízení & 1Bh & 1B & \begin{tabular}[c]{@{}l@{}}Zařízení je \\ pokojové čidlo\end{tabular} \\ \hline
24 & 7A & Ci-Pole & Odpověd od zařízení & 7Ah & 7A & jedná se o M-Bus \\ \hline
26 & 62 & AccNo & Číslo přístupu & 41h & 214 & 214. přístup \\ \hline
28 & 08 & Status & Status zařízení & 00h & 8 &  \\ \hline
30 & 0000 & \begin{tabular}[c]{@{}l@{}}Config.\\ word\end{tabular} & Šifrování AES & 0000h &  & Telegram není šifrován \\ \hline
34 & 2F2F & \begin{tabular}[c]{@{}l@{}}AES \\ encr.\end{tabular} & Ověření AES & 2F2Fh &  &  \\ \hline
40 & 66 & DR1 & \begin{tabular}[c]{@{}l@{}}VIF: první \\ měřená veličina\end{tabular} & 66h & 66 & Teplota v \degree\,C\textsuperscript{-1} \\ \hline
42 & 99 & DR1 & hodnota teploty & 99h & 0199 & Teplota je 19.9\degree\,C \\ \hline
44 & 01 & DR1 & hodnota teploty & 01h &  &  \\ \hline
50 & 1A & DR2 & \begin{tabular}[c]{@{}l@{}}VIFE: druhá \\ měřená veličina\end{tabular} & 1Ah & 1A & Relativní vlhkost v \%\textsuperscript{-1} \\ \hline
52 & 93 & DR2 & hodnota vlhkosti & 93h & 0293 & Vlhkost je 29.3\,\% \\ \hline
54 & 02 & DR2 & hodnota vlhkosti & 02h &  &  \\ \hline
62 & 1D & DR3 & VIFE1: Norma & 1Dh &  &  \\ \hline
64 & 01 & DR3 & Příznak sabotáže & 00h & 1 & Čidlo bylo otevřeno \\ \hline
66 & 00 & DR3 & \begin{tabular}[c]{@{}l@{}}Příznak vybité\\  baterie\end{tabular} & 00h & 0 & Baterie je nabitá \\ \hline
94 & 87 & CRC & Kontrolní součet & 87h &  &  \\ \hline
96 & 9e & RSSI & \begin{tabular}[c]{@{}l@{}}Síla přijímaného \\ signálu\end{tabular} & 9Eh & 158 & \begin{tabular}[c]{@{}l@{}}Síla signálu \\ je -51dBm\end{tabular} \\ \hline
\end{tabular}}
\end{table}


Z tabulky je patrné, že nutné vyparsovat položky na následujících pozicích:
\begin{itemize}
	\item 8-23 pro informace o daném čidlu,
	\item 24-25 pro určení pořadí telegramu,
	\item 42-45 pro hodnotu naměřené teploty,
	\item 52-55 pro hodnotu naměřené vlhkosti,
	\item 64-67 pro kontrolu stavu čidla a
	\item 96 pro úroveň signálu.	
\end{itemize}

a jejich následnou správnou interpretací dle specifikace (zohlednění uložení LSB, převod hexadecimálních hodnot na dekadické) předat k dalšímu zpracování či uložení do databáze.


\section{Zajištění uložení dat}
Zachycená a naměřená data se ukládají do databáze k pozdějšímu zpracování. Zvolena byla databáze SqLite3 pro svoji jednoduchost, nenáročnost na sytémové prostředky a možností instalace z repozitáře Raspbianu:
 
\begin{lstlisting}[style=MyCodeBash]
		sudo apt-get update
		sudo apt-get install sqlite3
	\end{lstlisting}

Byla zvolena jedna databáze se třemi tabulkami:
\begin{itemize}
	\item DEVICES - evidence známých zařízení a jejich AES klíčů
	\item VALUES - uložení naměřených hodnot
	\item TELEGRAMS - uložení zachycených dat a AES klíče modulu
\end{itemize}

Struktura tabulek je popsána na obrázku... \colorbox[rgb]{1,0,0}{DOPSAT + SCHEMA}
	
\section{Zajištění vizualizace dat}	
\colorbox[rgb]{1,0,0}{POPSAT GOOGLE API}

\section{Struktura aplikace}
Vzhledem k výše uvedeným požadavakům a technologiím byla zvolena struktura aplikace znázorněná na Obr.~\ref{AplikaceDiagram}. Diagram je pro přehlednost odlišen barevnými bloky:
\begin{itemize}
	\item modrou barvou je znázorněna kostra programu,
	\item zelenou barvou je nekonečná smyčka naslouchání dat,
	\item růžovou barvou je případné dešifrování přenášených dat,
	\item červenou barvou jsou chyby znemožnující běh programu,
	\item oranžovou barvou jsou chyby znemožňující platnou analýzu či dešifrování daného telegramu a
	\item černou barvu je řízení samotného programu.
\end{itemize}

V následujících podkapitolách budou jednotlivé bloky aplikace představeny podrobněji.

\subsection{Start programu v rámci operačního systému}
Program je nyní spouštěn automaticky po startu operačního systému interpretem jazyka Python v příkazu screen. Tím je zajištěna nezávilost na typu nasazení RaspberryPi a případných restartech zařízení. Ukončení programu nastává pouze násilným ukončením aplikace, restartem zařízení nebo závažnou chybou při startu programu. 

 \begin{figure}[!ht]
  \begin{center}
    \includegraphics[scale=0.6]{obrazky/aplikace_diagram}
  \end{center}
  \caption{Vývojový diagram aplikace pro vyčítání dat}
	\label{AplikaceDiagram}
\end{figure}

\subsection{Start programu z pohledu aplikace}
Program při startu kontroluje, zdali  má k dispozici všechny potřebné komponenty pro svůj běh. Program je závislý na knihovně PyCrypto, Serial nebo SQLite databázi.
Dále program kontroluje přítomnost a možnost otevření sériového portu. V případě úspěšného otevření portu je na něj zaslán příznak pro probuzení komunikačního modulu z úsporného režimu. Po probuzení následujě příkaz, který nastaví modul do režimu skeneru v komunikačním módu T1. Pokud některá z operací selže, je zaznamenán chybový stav a dojde k ukončení programu.

\subsection{Základní kontrola a parsování dat}
\colorbox[rgb]{1,0,0}{ta cestin!}
Nasledne dojde ke spusteni smycky naslouchani prichozich telegramu. Kazdy prichozi telegram je podroben serii kontrol, zdali doslo ke spravnemu odposlechnuti. Jako prvni kontrola je delka telegramu odpovidajici sudemu nasobku. Telegram je dale podroben zakladni analyze na parsovani polozek aplikacni a linkove vrstvy. Zde dochazi ke kontrole delky telegramu vuci hodnote uvedene v hlavicce, dale je provedena kontrola platnosti CRC souctu. Je take provedena kontrola obsahu Cl-pole, Status pole a ConfigurationWord pole.

\subsection{Dešifrování dat}
\colorbox[rgb]{1,0,0}{ta cestin!}
 Na zaklade analyzy posledni hodnoty se program dale vetvi:
\begin{itemize}
	\item nesifrovane telegramy jsou zapsany do databaze, pote dle vyrobce daneho mericiho zarizeni podrobeny dukladne analyze prenasenych dat. Zjistene hodnoty doplnene o casovou znacku a informace o mericim zarizeni zapsany do databaze a sdeleny na graficky vstup.
	\item v pripade ze analyzou pole ConfigurationWord je zjisteno sifrovani prenasenych dat algoritmem AES128 v modu CBC, dojde k rozsifrovani dat dle postupu popsanem v kapitole \colorbox[rgb]{0,0,0}{LINK} a pote program s telegramem zachazi jako s nesifrovanym, popsanym v predchozim bode.
\end{itemize}
Pokud nektera z operaci nad telegramem selze, dojde k vyjimce, dany telegram je zaevidovan, v jeho zpracovani se jiz nepokracuje a aplikace se vrati do stavu cekani na prichod dalsiho telegramu.

\subsection{Parsování dat}
\colorbox[rgb]{1,0,0}{DOPSAT}
\begin{itemize}
\item - co to umí
\item - jaká je aplikační logika pro RSSI a VendorID
\item - jak se to dělá pro jednotlivé výrobce
\item - co se tam kontroluje (signatura, rozsifrovani, delka paketu, statove bajty)
\end{itemize}

\subsection{Uložení dat}
\colorbox[rgb]{1,0,0}{DOPSAT}
\begin{itemize}
\item - jak se osetri ktere chyby
\item - co davame kam jak na vedomi
\item - co se zapise do logu a co do db
\end{itemize}

\subsection{Export dat}
\colorbox[rgb]{1,0,0}{ta cestin!}
Dalsim krokem je vizualizace ziskanych dat. Kazdou pulnoc se automaticky spusti export databaze, ktery davkove vyexportuje data namerena za poslednich 24 hodin do Google SpreadSheetu. Pro kazdy senzor v danem dni existuje samostatny Worksheet.

\subsection{Vizualizace dat}
\colorbox[rgb]{1,0,0}{ta cestin!}
Nad temito daty se potom provadi vizualizace pomoci Google Graphs. Ten umoznuje interaktivni odecitani vykreslenych hodnot zachycenych danym senzorem v danem dni, viz Obr.~\ref{fig_sec4_chart_example}). 

 \begin{figure}[!ht]
  \begin{center}
    \includegraphics[scale=0.8]{obrazky/chart_weptech}
  \end{center}
  \caption{Ukazka vizualizace dat}
	\label{fig_sec4_chart_example}
\end{figure}

\subsection{Ošetření chyb a vyjímek}
\colorbox[rgb]{1,0,0}{DOPSAT}
\begin{itemize}
\item - jak se ukladaji vysledky
\item - jak to vypada v DB, kolik je tabulek s cim
\item - kdy se kam co uklada a v jakem formatu
\item - co se zapise do logu a co do db
\end{itemize}










%% Vložení kapitoly s mymi poznamkami
\chapter{ToDo Poznámky}

\colorbox[rgb]{0,1,0}{neni realna kapitola. tady mam nejake poznamky k dodelani, abych to nemel po papircich.}

\begin{itemize}
	\item presazet veci z clanku [CLANEK!]
	\item doplnit kecy o AES sifrovano (norma FIPS 197), popsat jednozlive mody, napsat logiku desifrovani, popsat tvorbu inicializacniho vektoru
	\item v kapitole 7.2.5 popsat jak se to prakticky obecne desiforvalo, dat ukazku desifrovani AES128 v modu CBC do prace [CLANEK!]
	\item okomentovat dukladne navrhnute schema aplikace, nezapomenout vysvetlit logiku:zasifruj desifrovane, zkontroluj, desifruj znova [CLANEK!]
	\item dopsat druhou aplikaci pro vizualizaci, popsat nastaveni a instalaci veci k google sheetum
	\item sepsat proc se tu bavime pouze o ramci typu A (jedeme pouze v modu T zatim) a pripadne zminit nebo sepsat i ramec B
	\item do popisu vstev wm-busu zminit i obalku od IQRF modulu a UART sbernice...co a proc to tam je [CLANEK!]
	\item do popisu vrstev doplnit prehledy moznosti dle dane specifikace
	\item doplnit vzorec pro vypocet RSSI a VENDOR\_ID
	\item popsat ulozeni dat LSB vs MSB
	\item provazat to cele se specifikaci OSMS 3 a popsat proc se o tom vubec bavime
	\item dopsat nekam ze iqrf umi aes pouze jako jednosmery meter. muc implementace stoji za hovno, snifferova taky, ale ta jde castecne obejit [CLANEK?]
	\item Pikkerton+ZPA+Kamstrup presazet jejich teoreticke kapitoly a datasheery
	\item prekreslit vsechny potrebne obrazky, vcetne toho vyvojoveho diagramu. 
	\item prekreslit vsechny ukazky telegramu - narvat je to vektoru nebo tabulky.
	\item presazet vsechny prikazy pro linux a zachycene telegramy do cislovanych rovnic [CLANEK!]
	\item doplnit neco malo k uvodni strance [CLANEK!]
	\item projit praci a sjednotit reference Obr. vs obrázek
	\item projit praci a sjednotit sniffer-skener, MUC->koncentrator, Meter->Meric
	\item korektura tentokrat zavcas!
	\item ocitovat zdroje:
		\begin{itemize}
			\item FIPS197
			\item OMS 1-4 vrstvy, telegram example, primary comunication
			\item Datashety: Bonega, Kamstup, ZPA
		\end{itemize}
\end{itemize}


%% Vložení souboru 'text/zaver' se závěrem
\chapter{Závěr}

V této diplomové práci byla popsána problematika M2M (Machine to Machine) komunikace pomocí protokolu Wireless M-bus a její implementace do produktu UniPi NEURON.


V první části práce byla popsána M2M komunikace z pohledu spotřebitelského a průmyslového Internetu věcí.

Druhá část se zabývá embedded zařízeními pro IoT (Internet of Things), přináší přehled nejznámějších z nich, popisuje jejich možnosti, uvádí možnosti připojení senzorů a zmiňuje nedostatky zařízení. Zařízení RaspberryPi je následně použité k samotné implementaci v praktické části. Jsou zde popsány předchozí verze, důvod výběru konkrétního modelu, design i kroky potřebné k implementaci.

Třetí část obsahuje popis rozšiřující desky UniPi a zařízení UniPi NEURON. Popisuje blíže parametry obou zařízení, možnosti jejich konektivity a softwarového vybavení. Zařízení bylo vyvinuto primárně jako rozhraní pro příjem vstupních signálů, jejich vyhodnocení a realizaci výstupní reakce na základě naprogramovaných algoritmů. Je vhodné pro monitorování, sběr a ukládání dat na vzdálený server, nebo jako výkonná a plně vybavená brána pro ostatní zařízení.

Čtvtrá část se zabývá Wireless M-Bus modulem TR-72D-WMB výrobce IQRF, komunikující přes sběrnici UART, a popisuje strukturu příkazů a formát dat pro komunikaci s tímto modulem.

Pátá část se zaměřila na protokol Wireless M-Bus, konkrétně na princip komunikace, režimy přenosu a jednotlivé vrstvy. 
Díky nutnosti znalosti fyzické a linkové vrstvy pro pozdější analýzu zachytávaných dat byly tyto vrstvy rozebrány podrobněji. 

\colorbox[rgb]{1,0,0}{AKTUALIZOVAT}

V šesté části bylo popsáno čidlo WEPTECH a struktura dat jeho telegramu.

Závěrečná (sedmá) část obsahuje návrh a samotnou implementaci vzorové aplikace pro vyčítání dat. Jsou popsány jednotlivé kroky nutné ke zprovoznění komunikace mezi RaspberryPi a vyčítaným senzorem, provedeno zachycení vzorového telegramu, jeho analýza a následné předání zvolených informací. Z výstupu aplikace (viditelném v konzoli) je patrné, že pakety obsahují příslušná data, komunikace mezi modulem a zařízením funguje, data ze senzoru se přenášejí, následně vyčítají a zobrazují.


Nakonec bych rád zmínil další možnosti vývoje aplikace. V aktuální verzi je aplikace schopná zachytávat nešifrovaný přenos dat od Wireless M-Bus zařízení výrobce WEPTECH. V navazující diplomové práci proto bude přistoupeno k rozšíření podpory senzorů od více výrobců, AES kódování přenášených dat či vizualizaci zachytávaných dat.



\colorbox[rgb]{1,0,0}{VNORIT DO ZAVERU + ta cestin!}
V teto zaverecne casti se budeme zabyvat aspekty, kterym jsme celili v prubehu vyvoje naseho reseni. 
Behem implementace aplikace jsme museli vyresit nekolik problemu:
\begin{itemize}
	\item Implementace byla provedena na embeded zarizeni RaspberryPi verze 3, bohuzel pristup k seriovemu portu tohoto zarizeni je odlisny od RaspberryPi predchozich verzi. Byly nutne zasahy do konfigurace na urovni zavadeni operacniho systemu.
	\item Bylo nutne implementovat UART komunikaci s bezdratovym modulem IRQF pomoci daneho protokolu, ktery je stale ve fazi vyvoje a jehoz dokumentace neni nyni kompletni.
	\item V pripade odposlechu zasifrovanych dat bylo nutne vyresit jejich zasifrovani AES klicem bezdratoveho modulu IRQF zpet do desifrovatelneho prenosoveho stavu a opetovne desifrovani prislusnym klicem.
	\item Pro zvolena merici zarizeni bylo potreba provest analyzu zachycenych telegramu a urcit schema vyparsovani zachycenych dat. To stezuje situaci pro plosene nasazeni daneho scenare.
	\item Pri vizualizaci dat pomoci Google bylo nezbytne implementovat autorziacni a komunikacni knihovny tretich stran.
\end{itemize}
Reseni techto dilcich problemu umoznilo postupne dosahnout techto cilu:
\begin{itemize}
	\item V pripade vyrobcu WepTech, ZPA, Kamstrup a Bonega jsme schopni vycitat jakekoliv zarizeni, vcetne zarizeni podporujici pouze sifrovany prenos.	
	\item Nejvyznamejsim dosazenym cilem bylo dosazeni analyzy, ulozeni a vizualizace prijatych dat.
\end{itemize}

Na zaklade nasich zjisteni jsme schopni ve skutecnosti analyzovat jen predem znama zarizeni. Pokud bychom chteli nase reseni plne vyuzit pro internet veci, bylo by nutne implementovat velke mnozstvi parsovacich schemat. Z toho vyplyva, ze dane reseni pro tento okamzik nema potrebnou funkcnost internetu veci v realnem svete.




%% Vložení souboru 'text/literatura' se seznamem literatury
\begin{literatura}{99}

% uvodni kapitola

\bibitem{uvod_prumysl_4_pdf} Národní iniciativa Průmysl 4.0 \textit{Ministerstvo průmyslu a obchodu}[online]. 2015 [cit. 2016-10-15]. Dostupné z: \url{http://www.spcr.cz/images/priloha001-2.pdf}

\bibitem{uvod_prumysl_4_web} KORBEL, Petr. \textit{Průmyslová revoluce 4.0: Za 10 let se továrny budou řídit samy a produktivita vzroste o třetinu}[online]. 2016 [cit. 2016-10-15]. Dostupné z: \url{http://byznys.ihned.cz/c1-64009970-prumyslova-revoluce-4-0-za-10-let-se-tovarny-budou-ridit-samy-a-produktivita-vzroste-o-tretinu}

\bibitem{uvod_google_charts_api} \colorbox[rgb]{0,1,0}{ Google Chart API}

% internet veci industrial

\bibitem{iot_svet_hardware_internet_veci} VITEK, Jan. \textit{Internet of Things: propojená budoucnost}[online]. 2016 [cit. 2016-10-15]. Dostupné z: \url{http://www.svethardware.cz/internet-of-things-propojena-budoucnost/39560}

\bibitem{iot_pohanka_internet_veci} POHANKA, Pavel. \textit{Internet věcí}[online]. 2016 [cit. 2016-10-15]. Dostupné z: \url{http://i2ot.eu/internet-of-things/}

% embedded zarizeni

\bibitem{embed_about_wiring_2011} BARRAGAN, Hernando \textit{About Wiring} [online]. 2016 [cit. 2016-10-15]. Dostupné z: \url{http://wiring.org.co/about.html}

\bibitem{embed_about_processing_2015} Processing. \textit{Arduino.cz} [online]. 2016 [cit. 2016-12-14]. Dostupné z: \url{http://arduino.cz/processing/}

\bibitem{ArduinoDuemilanove} ArduinoBoard Duemilanove. \textit{Arduino} [online]. 2016 [cit. 2016-12-14]. Dostupné z: \url{https://www.arduino.cc/en/Main/ArduinoBoardDuemilanove}

\bibitem{ArduinoUno} ArduinoBoard Uno. \textit{Arduino} [online]. 2016 [cit. 2016-12-14]. Dostupné z: \url{https://www.arduino.cc/en/Main/ArduinoBoardUno}

\bibitem{ArduinoLeonardo} ArduinoBoard Leondardo. \textit{Arduino} [online]. 2016 [cit. 2016-12-14].  Dostupné z: \url{https://www.arduino.cc/en/Main/ArduinoBoardLeondardo}

\bibitem{ArduinoMega} ArduinoBoard Mega. \textit{Arduino} [online]. 2016 [cit. 2016-12-14]. Dostupné z: \url{https://www.arduino.cc/en/Main/ArduinoBoardMega}

\bibitem{ArduinoDue} ArduinoBoard Due. \textit{Arduino} [online]. 2016 [cit. 2016-12-14]. Dostupné z: \url{https://www.arduino.cc/en/Main/ArduinoBoardDue}

\bibitem{ArduinoMini} ArduinoBoard Mini. \textit{Arduino} [online]. 2016 [cit. 2016-12-14]. Dostupné z: \url{https://www.arduino.cc/en/Main/ArduinoBoardMini}

\bibitem{ArduinoMicro} ArduinoBoard Micro. \textit{Arduino} [online]. 2016 [cit. 2016-12-14]. Dostupné z: \url{https://www.arduino.cc/en/Main/ArduinoBoardMicro}

\bibitem{ArduinoNano} ArduinoBoard Nano. \textit{Arduino} [online]. 2016 [cit. 2016-12-14]. Dostupné z: \url{https://www.arduino.cc/en/Main/ArduinoBoardNano}

\bibitem{ArduinoFio} ArduinoBoard Fio. \textit{Arduino} [online]. 2016 [cit. 2016-12-14]. Dostupné z: \url{https://www.arduino.cc/en/Main/ArduinoBoardFio}

\bibitem{ArduinoLilipad} ArduinoBoard Lilipad. \textit{Arduino} [online]. 2016 [cit. 2016-12-14]. Dostupné z: \url{https://www.arduino.cc/en/Main/ArduinoBoardLilipad}

\bibitem{ArduinoYun} ArduinoBoard Yun. \textit{Arduino} [online]. 2016 [cit. 2016-12-14]. Dostupné z: \url{https://www.arduino.cc/en/Main/ArduinoBoardYun}

\bibitem{ArduinoClonesWeb} Arduino and Arduino-Compatible Hardware. \textit{Arduino Playground} [online]. 2016 [cit. 2016-12-14]. Dostupné z: \url{http://playground.arduino.cc/main/similarBoards}

\bibitem{ArduinoTeensy} Teensy USB Development Board. \textit{PJRC} [online]. 2016 [cit. 2016-12-14]. Dostupné z: \url{http://www.pjrc.com/teensy/}

\bibitem{Raspi} RaspberryPi products. \textit{Raspberry Pi} [online]. 2016 [cit. 2016-12-14]. Dostupné z: \url{https://www.raspberrypi.org/products/}

\bibitem{GertBoard} Gertboard for Raspberry Pi \textit{ELEMENT14 Community} [online]. 2016 [cit. 2016-12-14]. Dostupné z: \url{https://www.element14.com/community/docs/DOC-69381/l/gertboard-for-raspberry-pi}

\bibitem{RaspiOne} RaspberryPi Model A+. \textit{Raspberry Pi} [online]. 2016 [cit. 2016-12-14]. Dostupné z: \url{https://www.raspberrypi.org/products/model-a-plus/}

\bibitem{RaspiTwo} RaspberryPi 2 model B. \textit{Raspberry Pi} [online]. 2016 [cit. 2016-12-14]. Dostupné z: \url{https://www.raspberrypi.org/products/raspberry-pi-2-model-b/}

\bibitem{RaspiThree} RaspberryPi 3 Model B. \textit{Raspberry Pi} [online]. 2016 [cit. 2016-12-14]. Dostupné z: \url{https://www.raspberrypi.org/products/raspberry-pi-3-model-b/}

\bibitem{RaspiZero} RaspberryPi Zero. \textit{Raspberry Pi} [online]. 2016 [cit. 2016-12-14]. Dostupné z: \url{https://www.raspberrypi.org/products/pi-zero/}

\bibitem{BananaPi} Open Source Hardware Products \textit{Banana Pi Products} [online]. 2016 [cit. 2016-12-14]. Dostupné z: \url{http://www.banana-pi.org/product.html}

\bibitem{OrangePi} What’s Orange Pi Plus? \textit{OrangePi Community} [online]. 2016 [cit. 2016-12-14]. Dostupné z: \url{http://www.orangepi.org/}

\bibitem{CubieBoards} A series of open source hardware \textit{CubieBoard} [online]. 2016 [cit. 2016-12-14]. Dostupné z: \url{http://cubieboard.org/model/}

\bibitem{UpBoard} UpBoard - Specifications \textit{UP - bidge the gap} [online]. 2016 [cit. 2016-12-14]. Dostupné z: \url{http://www.up-board.org/up/specifications/}

\bibitem{Pine64} SoC and Memory Specification \textit{PINE64} [online]. 2016 [cit. 2016-12-14]. Dostupné z: \url{http://wiki.pine64.org/index.php/Main\_Page\#SoC\_and\_Memory\_Specification}

\bibitem{HardKernel} ODROID Platforms\textit{HardKernel - Products} [online]. 2016 [cit. 2016-12-14]. Dostupné z: \url{http://www.hardkernel.com/main/products/prdt\_info.php}

\bibitem{BeagleBone} BeagleBone Black \textit{BeagleBoard.org Foundation} [online]. 2016 [cit. 2016-12-14]. Dostupné z: \url{https://beagleboard.org/black}

\bibitem{IntelGalileo} Intel Galileo. \textit{IoT Hardware Share} [online]. 2016 [cit. 2016-12-14]. Dostupné z: \url{http://www.intel.com/content/www/us/en/embedded/products/galileo/galileo-overview.html}

\bibitem{ArduinoGalileo} Arduino Galileo. \textit{Arduino} [online]. 2016 [cit. 2016-12-14]. Dostupné z: \url{https://www.arduino.cc/en/ArduinoCertified/IntelGalileo}

\bibitem{IntelEdison} Intel Edison. \textit{IoT Hardware Share} [online]. 2016 [cit. 2016-12-14]. Dostupné z: \url{http://www.intel.com/content/www/us/en/do-it-yourself/edison.html}

\bibitem{AmdGizmo1} GIZMO 1 \textit{GizmoSphere - Development unleashed} [online]. 2016 [cit. 2016-12-14]. Dostupné z: \url{http://www.gizmosphere.org/products/gizmo-explorer-kit/}

\bibitem{AmdGizmo2} GIZMO 2 \textit{GizmoSphere - Development unleashed} [online]. 2016 [cit. 2016-12-14]. Dostupné z: \url{http://www.gizmosphere.org/products/gizmo-2/}

% deska unipi

\bibitem{ZelenaData} ZELENÁ DATA - Datacentrum s inteligentní energií společnosti Faster CZ. \textit{ZELENÁ DATA - DATACENTRUM FASTER} [online]. 2016 [cit. 2016-12-14]. Dostupné z: \url{http://zelenadata.cz/cs/}

\bibitem{UniPiBoard} UniPi.technology \textit{UniPi.technology} [online]. 2016 [cit. 2016-12-14]. Dostupné z: \url{http://unipi.technology/}

\bibitem{UniPiBoard1} UniPi board \textit{UniPi.technology} [online]. 2016 [cit. 2016-12-14]. Dostupné z: \url{http://unipi.technology/product/unipi/}
				
\bibitem{UniPiBoard2} UniPi Neuron S103 \textit{UniPi.technology} [online]. 2016 [cit. 2016-12-14]. Dostupné z: \url{http://unipi.technology/product/unipi-neuron-s103/}



\bibitem{Boswarthick2012} BOSWARTHICK, David, Omar ELLOUMI a Olivier HERSENT. \textit{M2M communications: a systems approach}. 1.~vydání. Hoboken, N.J.: Wiley, 2012. ISBN 978-1-119-99475-6.

\bibitem{Monk[2013]} MONK, Simon. \textit{Programming the Raspberry Pi: getting started with Python}. 1.~vydání. New York: McGraw-Hill, 2013. ISBN 0071807837.
	
\bibitem{Anton-Haro2015} ANTON-HARO, C. \textit{Machine-to-machine (m2m) communications}. 1.vydání. Boston, MA: Elsevier, 2015. ISBN 9781782421023.	

\bibitem{Hersent2012} HERSENT, Olivier, David BOSWARTHICK a Omar ELLOUMI. \textit{The internet of things: applications to the smart grid and building automation}. 1.~vydání. Hoboken, NJ: Wiley, 2012. ISBN 9781119994350.

\bibitem{WMencodeing} SILICON LABS. WIRELESS M-BUS SOFTWARE IMPLEMENTATION. 2010, 14 s. Dostupné z: \url{https://www.silabs.com/Support\%20Documents/TechnicalDocs/AN451.pdf}

\bibitem{FormatDatoveJednotky} Sběrnice Wireless M-BUS - jde to i bezdrátově... Automatizace.hw.cz [online]. 2010 [cit. 2014-10-28]. Dostupné z: \url{http://automatizace.hw.cz/sbernice-wireless-mbus-jde-i-bezdratove}

\bibitem{Norma1} EN 13757-1. \textit{Communication system for and remote reading of meters - Part 1: Data exchange}. Wien: Austrian Standards Institute. [online] 2016 [cit. 2016-12-14]  Dostupné z: \url{https://shop.austrian-standards.at/Preview.action;jsessionid=4B46107107AC62A5CB24E33F6A51A5E4?preview=\&dokkey=467673\&selectedLocale=en}
				
\bibitem{Norma4} EN 13757-4. C\textit{ommunication systems for meters and remote reading of meters - Part 4: Wireless meter readout (Radio Meter reading for operation in the 868-870 MHz SRD band)}. [online] 2016 [cit. 2016-12-14] Brusel: EUROPEAN COMMITTEE FOR STANDARDIZATION, 2003. Dostupné z: \url{http://oldfjarrvarme.unc.se/download/1309/fj}








				



				













				


\bibitem{WmbusTables} Wireless Meter Bus, WM-Bus Technology \textit{Radio-Electronics.com - Adrio Communications Ltd} [online]. 2016 [cit. 2016-12-14]. Dostupné z: \url{http://www.radio-electronics.com/info/wireless/wireless-m-bus/basics-tutorial.php}
				
\bibitem{MervisWeb} MERVIS \textit{UniPi.technology} [online]. 2016 [cit. 2016-12-14]. Dostupné z: \url{http://unipi.cz/software/mervis/}


				
\bibitem{WeptechCidlo} Wireless M-Bus / OMS Humidity and temperature sensor WEP-OMSF-868A \textit{WEPTECH elektronik GmbH} [online]. 2016 [cit. 2016-12-14]. Dostupné z: \url{https://www.weptech.de/products/oms-humidity-and-temperature-sensor-wep-omsf-868a.html}

\bibitem{WmbusVendors} 	 	FLAG - Registered Manufacturers Identification Characters \textit{FLAG Association Limited} [online]. 2008 [cit. 2016-12-14]. Dostupné z: \url{http://www.dlms.com/flag/INDEX.HTM}

\bibitem{sberniceUART} Sběrnice USART \textit{Wikipedie - otevřená encyklopedie} [online]. 2016 [cit. 2016-12-14]. Dostupné z: \url{https://cs.wikipedia.org/wiki/USART}

\bibitem{raspiGPIOs} PiBrella compatibility \textit{ModMyPi} [online]. 2016 [cit. 2016-12-14]. Dostupné z: \url{http://forum.modmypi.com/technical-support/pibrella-compatibility-t181.html}

\bibitem{iqrfmodul} TR-72D-WMB series \textit{IQRF - Technology for Wireless} [online]. 2016 [cit. 2016-12-14]. Dostupné z: \url{http://www.iqrf.org/products/transceivers/tr-72d-wmb}

\bibitem{Manchester} Kódování Manchester \textit{Wikipedie - otevřená encyklopedie} [online]. 2016 [cit. 2016-12-14]. Dostupné z: \url{https://cs.wikipedia.org/wiki/K\%C3\%B3dov\%C3\%A1n\%C3\%AD\_Manchester}

\bibitem{BonegaCidlo} Ultra-antimagnetické bytové vodoměry s bezdrátovým přenosem dat \textit{Vodoměry BONEGA} [online]. 201 [cit. 2017-2-20]. Dostupné z: \url{http://www.bonega.cz/vodomery/index.htm}

%%%%%%%%%%%%%%%%%%%%%%%%%

	\bibitem{bib_sec2_iqrf_modul} TR-72D-WMB series \textit{IQRF - Technology for Wireless} [online]. 2016 [cit. 2017-03-11]. Dostupne z: \url{http://www.iqrf.org/products/transceivers/tr-72d-wmb}
	\bibitem{bib_sec3_device_weptech} Wireless M-Bus / OMS Humidity and temperature sensor WEP-OMSF-868A \textit{WEPTECH elektronik GmbH} [online]. 2016 [cit. 2017-03-11]. Dostupne z: \url{https://www.weptech.de/products/oms-humidity-and-temperature-sensor-wep-omsf-868a.html}
	\bibitem{bib_sec3_device_bonega} Ultra-antimagneticke bytove vodomery s bezdratovym prenosem dat \textit{Vodomery BONEGA} [online]. 2016 [cit. 2017-03-11]. Dostupne z: \url{http://www.bonega.cz/vodomery/index.htm}
	\bibitem{bib_sec3_device_zpa} Trifazovy elektromer ZE310 \textit{ZPA Smart Energy a.s.} [online]. 2016 [cit. 2017-03-11]. Dostupne z: \url{https://www.zpa.cz/produkty-a-reseni/elektromery:c1/ze-310:p4.htm}
	\bibitem{bib_sec3_device_kamstrup} Nejflexibilnejši meric na trhu MULTICAL 403 \textit{Kamstup ČR} [online]. 2016 [cit. 2017-03-11]. Dostupné z: \url{https://www.kamstrup.com/cs-cz/products-and-solutions/thermal-energy-meters/multical-403}
	\bibitem{bib_sec3_specs_oms} The Open Metering System specification \textit{OMS-Group} [online]. 2016 [cit. 2017-03-11]. Dostupne z: \url{http://oms-group.org/en/oms-group/about-oms-group/}
	\bibitem{bib_sec3_specs_wmbus} EN 13757-4. \textit{ommunication systems for meters and remote reading of meters - Part 4: Wireless meter readout (Radio Meter reading for operation in the 868-870 MHz SRD band)}. [online] 2016 [cit. 2017-03-11] Brusel: EUROPEAN COMMITTEE FOR STANDARDIZATION, 2003. Dostupne z: \url{http://oldfjarrvarme.unc.se/download/1309/fj}

%%%%%%%%%%%%%%%%%%%%%%%%%

\end{literatura}



%% Vložení souboru 'text/zkratky' se seznam použitých symbolů, veličin a zkratek
\begin{seznamzkratek}{KolikMista}
	\novazkratka{zkACK}{ACK}{Acknowledgement}
	\novazkratka{zkAES}{AES}{Advanced Encryption Standard}
	\novazkratka{zkAMR}{AMR}{Automated Meter Reading}
	\novazkratka{zkAPI}{API}{Application Programming Interface}
	\novazkratka{zkAPU}{APU}{Accelerated Processing Unit}
	\novazkratka{zkARM}{ARM}{Advanced (Acorn) RISC Machine}
	\novazkratka{zkAVR}{AVR}{Alf \& Vegard Risc procesor}
	\novazkratka{zkB}{bd}{baud}
	\novazkratka{zkB}{B}{Byte}
	\novazkratka{zkCAN}{CAN}{Controller Area Network}
	\novazkratka{zkCBC}{CBC}{Cipher Block Chaining}
	\novazkratka{zkcIoT}{cIoT}{Customer IoT}
	\novazkratka{zkCPU}{CPU}{Central Processing Unit}
	\novazkratka{zkCR}{CR}{Carriage Return}
	\novazkratka{zkCRC}{CRC}{Cyclic Redundancy Check}
	\novazkratka{zkCSI}{CSI}{Camera Serial Interface}
	\novazkratka{zkCVBS}{CVBS}{Color Video Blank Sync}
	\novazkratka{zkdBm}{dBm}{decibel on milliwatt}
	\novazkratka{zkDES}{DES}{Data Encryption Standard}
	\novazkratka{zkDIB}{DIB}{Data Information Block}
	\novazkratka{zkDIF}{DIF}{Data Information Field}
	\novazkratka{zkDIFE}{DIFE}{Data Information Field Extended}
	\novazkratka{zkDPA}{DPA}{Direct Peripheral Access}	
	\novazkratka{zkDSI}{DSI}{Display Serial Interface}
	\novazkratka{zkEEPROM}{EEPROM}{Electrically Erasable Programmable Read-Only Memory}
	\novazkratka{zkeMMC}{eMMC}{embedded MultiMedia Card}
	\novazkratka{zkESD}{ESD}{ElectroStatic Discharge}
	\novazkratka{zkFPU}{FPU}{Floating-Point Unit}
	\novazkratka{zkGPIO}{GPIO}{General Purpose Input Output}
	\novazkratka{zkGB}{GB}{GigaByte}
	\novazkratka{zkGHz}{GHz}{GigaHertz}
	\novazkratka{zkGPU}{GPU}{Graphical Processing Unit}
	\novazkratka{zkHDMI}{HDMI}{High-Definition Multimedia Interface}
	\novazkratka{zkHCA}{HCA}{Heat Cost Allocator}
	\novazkratka{zkHz}{Hz}{Hertz}
	\novazkratka{zkH2H}{H2H}{Human to Human}	
	\novazkratka{zkICSP}{ICSP}{In Circuit Serial Programming}	
	\novazkratka{zkIDE}{IDE}{Integrated Development Enviroment}	
	\novazkratka{zkiIoT}{iIoT}{Industry IoT}
	\novazkratka{zkIoP}{IoP}{Internet of People}
	\novazkratka{zkIoS}{IoS}{Internet of Services}
	\novazkratka{zkIoT}{IoT}{Internet of Things}
	\novazkratka{zkISM}{ISM}{Industrial, Scientific and Medical}
	\novazkratka{zkIV}{IV}{Initialization Vector}
	\novazkratka{zkI/O}{I/O}{Input / Output}	
	\novazkratka{zkI2C}{I2C}{Inter-Integrated Circuit}	
	\novazkratka{zkI2S}{I2S}{Inter-Integrated Sound}	
	\novazkratka{zkJTAG}{JTAG}{Joint Test Action Group}
	\novazkratka{zkKB}{KB}{KiloByte}
	\novazkratka{zkKHz}{KHz}{KiloHertz}
	\novazkratka{zkLRADC}{LRADC}{Low Resolution Analog to Digital Converter}
	\novazkratka{zkLSB}{LSB}{Least Significant Bit}
	\novazkratka{zkLVDS}{LVDS}{Low-Voltage Differential Signaling}	
	\novazkratka{zkMB}{MB}{MegaByte}
	\novazkratka{zkMBus}{M-Bus}{Meter Bus}
	\novazkratka{zkMCU}{MCU}{Micro-Controller Unit}
	\novazkratka{zkMHz}{MHz}{MegaHertz}
	\novazkratka{zkMicroSD}{MicroSD}{Micro Secure Digital}
	\novazkratka{zkMISO}{MISO}{Master Input Slave Output}
	\novazkratka{zkMOSI}{MOSI}{Master Output Slave Input}
	\novazkratka{zkMSB}{MSB}{Most Significant Bit}
	\novazkratka{zkMTCD}{MTC}{Machine Type Communication}
	\novazkratka{zkMTCD}{MTCD}{Machine Type Communication Device}
	\novazkratka{zkMTCG}{MTCG}{Machine Type Communication Gateway}
	\novazkratka{zkM2M}{M2M}{Machine to Machine}	
	\novazkratka{zkNRZ}{NRZ}{Non Return to Zero}
	\novazkratka{zkOMS}{OMS}{Open Metering Standard}
	\novazkratka{zkPCI}{PCI}{Peripheral Component Interconnect}
	\novazkratka{zkPLC}{PLC}{Programmable Logic Controller}
	\novazkratka{zkPoE}{PoE}{Power over Ethernet}	
	\novazkratka{zkPS2}{PS2}{Personal System/2 }
	\novazkratka{zkPWM}{PWM}{Pulse Width Modulation}
	\novazkratka{zkQoE}{QoE}{Quality of Experience}
	\novazkratka{zkREQ}{REQ}{Request}
	\novazkratka{zkRSSI}{RSSI}{Received Signal Strength Indication}
	\novazkratka{zkRTC}{RTC}{Real Time Clock}
	\novazkratka{zkRF}{RF}{Radio Frequency}
	\novazkratka{zkRX}{RX}{Receive}
	\novazkratka{zkRZ}{RZ}{Return to Zero}
	\novazkratka{zkSATA}{SATA}{Serial Advanced Technology Attachment}
	\novazkratka{zkSCLK}{SCLK}{Serial Clock}
	\novazkratka{zkSD}{SD}{Secure Digital}	
	\novazkratka{zkSDIO}{SDIO}{Serial Data Input Output}
	\novazkratka{zkSoc}{SoC}{System on Chip}
	\novazkratka{zkSRAM}{SRAM}{Static Random Access Memory}
	\novazkratka{zkSRD}{SRD}{Short Range Device}
	\novazkratka{zkSS}{SS}{Slave Select}
	\novazkratka{zkTX}{TX}{Transmit}
	\novazkratka{zkUART}{UART}{Universal Asynchronous Receiver/Transmitter}
	\novazkratka{zkUSB}{USB}{Universal Serial Bus}	
	\novazkratka{zkVIB}{VIB}{Value Information Block}
	\novazkratka{zkVIF}{VIF}{Value Information Field}
	\novazkratka{zkVIFE}{VIFE}{Value Information Field Extended}
	\novazkratka{zkWiFi}{Wi-Fi}{Wireless Fidelity}
	\novazkratka{zkWMBus}{WM-Bus}{Wireless Meter Bus}
\end{seznamzkratek}



%% Začátek příloh
\prilohy

%% Vysázení seznamu příloh
\seznampriloh

%% Vložení souboru 'text/prilohy' s přílohami
%%%%%%%%%%%%%%%%%%%%%%%%%%%%%%%%%%%%%%%%%%%%%%%%%%%%%%%%%%%%%%%%%%%%%%%%%%%%%%%%%%%%%%%%%%%%%%%%%%%%%%%%%%%%%%%
\begin{landscape}
\chapter{Přehled parametrů jednotlivých jednodeskových počítačů}
\label{PrilohaTabulkaTabulka}
\centering
\vspace{5pt}
\resizebox{\paperwidth}{!}{%
\begin{tabular}{|c|c|c|c|c|c|c|c|c|c|c|c|c|c|c|c|c|c|c|c|c|c|c|c|c|c|}
\hline

\rotatebox[origin=c]{90}{\parbox[b]{4cm}{\hspace{10pt}\textbf{Výrobce desky}}} &                        
\rotatebox[origin=c]{90}{\parbox[b]{4cm}{\hspace{10pt}\textbf{Označení modelu}}} & 
\rotatebox[origin=c]{90}{\parbox[b]{4cm}{\hspace{10pt}\textbf{Mikrokontrolér}}} & 
\rotatebox[origin=c]{90}{\parbox[b]{4cm}{\hspace{10pt}\textbf{Platforma}}} & 
\rotatebox[origin=c]{90}{\parbox[b]{4cm}{\hspace{10pt}\textbf{Frekvence}}} & 
\rotatebox[origin=c]{90}{\parbox[b]{4cm}{\hspace{10pt}\textbf{Flash [KiB]}}} & 
\rotatebox[origin=c]{90}{\parbox[b]{4cm}{\hspace{10pt}\textbf{EEPROM [KiB]}}} & 
\rotatebox[origin=c]{90}{\parbox[b]{4cm}{\hspace{10pt}\textbf{RAM [KiB]}}} & 
\rotatebox[origin=c]{90}{\parbox[b]{4cm}{\hspace{10pt}\textbf{Digitální piny}}} &
\rotatebox[origin=c]{90}{\parbox[b]{4cm}{\hspace{10pt}\textbf{PWM kanály}}} &
\rotatebox[origin=c]{90}{\parbox[b]{4cm}{\hspace{10pt}\textbf{Analogové vstupy}}} & 
\rotatebox[origin=c]{90}{\parbox[b]{4cm}{\hspace{10pt}\textbf{UART sběrnice}}} &
\rotatebox[origin=c]{90}{\parbox[b]{4cm}{\hspace{10pt}\textbf{I2C sběrnice}}} &
\rotatebox[origin=c]{90}{\parbox[b]{4cm}{\hspace{10pt}\textbf{SPI sběrnice}}} &
\rotatebox[origin=c]{90}{\parbox[b]{4cm}{\hspace{10pt}\textbf{Ethernet}}} &
\rotatebox[origin=c]{90}{\parbox[b]{4cm}{\hspace{10pt}\textbf{Wi-Fi}}} &
\rotatebox[origin=c]{90}{\parbox[b]{4cm}{\hspace{10pt}\textbf{USB master}}} &
\rotatebox[origin=c]{90}{\parbox[b]{4cm}{\hspace{10pt}\textbf{MicroSD}}} &
\rotatebox[origin=c]{90}{\parbox[b]{4cm}{\hspace{10pt}\textbf{Mini PCI-e}}} &
\rotatebox[origin=c]{90}{\parbox[b]{4cm}{\hspace{10pt}\textbf{SATA}}} &
\rotatebox[origin=c]{90}{\parbox[b]{4cm}{\hspace{10pt}\textbf{Grafický výstup}}} &
\rotatebox[origin=c]{90}{\parbox[b]{4cm}{\hspace{10pt}\textbf{Bluetooth}} }&
\rotatebox[origin=c]{90}{\parbox[b]{4cm}{\hspace{10pt}\textbf{HDMI}}} &
\rotatebox[origin=c]{90}{\parbox[b]{4cm}{\hspace{10pt}\textbf{Podpora shieldů}} }&
\rotatebox[origin=c]{90}{\parbox[b]{4cm}{\hspace{10pt}\textbf{USB rozhraní}}} &
\rotatebox[origin=c]{90}{\parbox[b]{4cm}{\hspace{10pt}\textbf{Rozměry}}} \\ \hline \hline



\multirow{13}{*}{\textbf{Arduino}}        & \textbf{Diecimila}              & ATmega168                & ARM       & 16 MHz              & 16              & 0.5              & 1              & 14             & 6          & 6                & ano           & ano          & ano          & ne       & ne   & ne         & ne      & ne        & ne   & ne              & ne        & ne   & ano             & FTDI          & 68.6 x 53.3 mm   \\ \cline{2-26}
                & \textbf{Duemilanove (v2)}       & ATmega328P               & ARM       & 16 MHz              & 32              & 1                & 2              & 14             & 6          & 6                & ano           & ano          & ano          & ne       & ne   & ne         & ne      & ne        & ne   & ne              & ne        & ne   & ano             & FTDI          & 68.6 x 53.3 mm   \\ \cline{2-26} 
                & \textbf{Uno (R3)}               & ATmega328P               & ARM       & 16 MHz              & 32              & 1                & 2              & 14             & 6          & 6                & ano           & ano          & ano          & ne       & ne   & ne         & ne      & ne        & ne   & ne              & ne        & ne   & ano             & ATmega8U2     & 68.6 x 53.3 mm   \\ \cline{2-26} 
                & \textbf{Due}                    & ATMEL SAM3U              & ARM       & 84 MHz              & 512             & 1                & 96             & 54             & 12         & 16               & ano           & ano          & ano          & ne       & ne   & ne         & ne      & ne        & ne   & ne              & ne        & ne   & ano             & Prog + Native & 101.5 x 53.3 mm  \\ \cline{2-26}
                & \textbf{Mega (2560)}            & ATmega2560               & ARM       & 16 MHz              & 256             & 4                & 8              & 54             & 14         & 16               & ano           & ano          & ano          & ne       & ne   & ne         & ne      & ne        & ne   & ne              & ne        & ne   & ano             & FTDI          & 101.5 x 53.3 mm  \\ \cline{2-26}
                & \textbf{Leonardo}              & ATmega32u4               & ARM       & 16 MHz              & 32              & 1                & 2              & 14             & 6          & 12               & ano           & ano          & ano          & ne       & ne   & ne         & ne      & ne        & ne   & ne              & ne        & ne   & ano             & Atmega32u4    & 68.6 × 53.3mm    \\ \cline{2-26}
                & \textbf{Fio}                   & ATmega328P               & ARM       & 8 MHz               & 32              & 1                & 2              & 14             & 6          & 8                & ano           & ano          & ano          & ne       & ne   & ne         & ne      & ne        & ne   & ne              & ne        & ne   & ne              & FTDI          & 40.6 x 27.9 mm   \\ \cline{2-26}
                & \textbf{Mini}                  & ATmega328P               & ARM       & 16 MHz              & 32              & 1                & 2              & 14             & 6          & 8                & ano           & ano          & ano          & ne       & ne   & ne         & ne      & ne        & ne   & ne              & ne        & ne   & ne              & UART          & 30.5 x 18.0 mm   \\ \cline{2-26}
                & \textbf{Micro}                  & ATmega32u4               & ARM       & 16 MHz              & 32              & 1                & 2.5            & 20             & 7          & 12               & ano           & ano          & ano          & ne       & ne   & ne         & ne      & ne        & ne   & ne              & ne        & ne   & ne              & Atmega32u4    & 50.0 x 13.0 mm   \\ \cline{2-26}
                & \textbf{Nano v2}               & ATmega328                & ARM       & 16 MHz              & 32              & 1                & 2              & 14             & 6          & 8                & ano           & ano          & ano          & ne       & ne   & ne         & ne      & ne        & ne   & ne              & ne        & ne   & ne              & FTDI          & 43.0 x 18.0 mm   \\ \cline{2-26}
                & \textbf{LilyPad (v2)}           & ATmega328V               & ARM       & 8 MHz               & 16              & 0.5              & 1              & 14             & 6          & 6                & ano           & ne           & ne           & ne       & ne   & ne         & ne      & ne        & ne   & ne              & ne        & ne   & ne              & UART          & ø 50mm           \\ \cline{2-26}
                & \multirow{2}{*}{\textbf{Ýun}}                    & Atheros AR9331           & x86       & 400 MHz             & 16 MiB          & \multirow{2}{*}{1}                & 64 MB          & \multirow{2}{*}{-}               & \multirow{2}{*}{7}           & \multirow{2}{*}{12}               & \multirow{2}{*}{ano}            & \multirow{2}{*}{ano}           & \multirow{2}{*}{ano}           & \multirow{2}{*}{ano}       & \multirow{2}{*}{ano}  & \multirow{2}{*}{ano}         & \multirow{2}{*}{ano}      & \multirow{2}{*}{ne}         & \multirow{2}{*}{ne}   & \multirow{2}{*}{ne}               & \multirow{2}{*}{ne}         & \multirow{2}{*}{ne}    & \multirow{2}{*}{ano}             & \multirow{2}{*}{Atmega32u4}    & \multirow{2}{*}{68.6 x 53.3 mm}    \\ \cline{3-6} \cline{8-8}
                &                        & ATmega32u4               & ARM       & 16 MHz              & 32              &                  & 1              &                &            &                  &               &              &              &          &      &            &         &           &      &                 &           &      &                 &               &                  \\ \hline
\multirow{10}{*}{\textbf{Arduino klony}}   & \textbf{Teensy (v 3.2)}         & MK20DX256                & ARM       & 72 MHz              & 256             & 64               & 2              & 34             & 21         & 1                & ano           & ano          & ano          & ne       & ne   & ne         & ne      & ne        & ne   & ne              & ne        & ne   & ne              & FTDI          & 30.5 x 18.0 mm   \\ \cline{2-26}
                & \textbf{Freeduino }             & ATmega168                & ARM       & 16 MHz              & 16              & 0.5              & 1              & 14             & 6          & 6                & ano           & ano          & ano          & ne       & ne   & ne         & ne      & ne        & ne   & ne              & ne        & ne   & ano             & FTDI          & 68.6 x 53.3 mm   \\ \cline{2-26}
                & \textbf{LABduino }              & ATmega328P               & ARM       & 16 MHz              & 32              & 1                & 2              & 14             & 6          & 6                & ano           & ano          & ano          & ne       & ne   & ne         & ne      & ne        & ne   & ne              & ne        & ne   & ano             & FTDI          & 51.0 x 51.0 mm   \\ \cline{2-26}
                & \textbf{Arduelo Libero}         & ATmega168                & ARM       & 16 MHz              & 16              & 0.5              & 1              & 14             & 6          & 6                & ano           & ano          & ano          & ne       & ne   & ne         & ne      & ne        & ne   & ne              & ne        & ne   & ano             & FTDI          & 68.6 x 53.3 mm   \\ \cline{2-26}
                & \textbf{Bare Bones Board }      & ATmega328P               & ARM       & 16 MHz              & 32              & 1                & 2              & 20             & 6          & 6                & ano           & ano          & ano          & ne       & ne   & ne         & ne      & ne        & ne   & ne              & ne        & ne   & ne              & FTDI          & 59.7 x 40.6 mm   \\ \cline{2-26}
                & \textbf{Nanode}                 & ATmega328P               & ARM       & 16 MHz              & 32              & 1                & 2              & 14             & 6          & 6                & ano           & ano          & ano          & ne       & ne   & ne         & ne      & ne        & ne   & ne              & ne        & ne   & ano             & ATmega8U2     & 68.6 x 53.3 mm   \\ \cline{2-26}
                & \textbf{Freaduino}              & ATmega328P               & ARM       & 16 MHz              & 32              & 1                & 2              & 14             & 6          & 6                & ano           & ano          & ano          & ne       & ne   & ne         & ne      & ne        & ne   & ne              & ne        & ne   & ano             & ATmega8U2     & 68.6 x 53.3 mm   \\ \cline{2-26}
                & \textbf{Seeeduino }             & ATmega1280               & ARM       & 16 MHz              & 128             & 4                & 8              & 54             & 14         & 16               & ano           & ano          & ano          & ne       & ne   & ne         & ne      & ne        & ne   & ne              & ne        & ne   & ano             & FTDI          & 68.6 x 53.3 mm   \\ \cline{2-26}
                & \textbf{Diavolino }             & ATmega328P               & ARM       & 16 MHz              & 32              & 1                & 2              & 14             & 6          & 6                & ano           & ano          & ano          & ne       & ne   & ne         & ne      & ne        & ne   & ne              & ne        & ne   & ano             & ATmega8U2     & 68.6 x 53.3 mm   \\ \cline{2-26}
                & \textbf{Boarduino}              & ATmega328P               & ARM       & 16 MHz              & 16              & 0.5              & 1              & 14             & 6          & 6                & ano           & ano          & ano          & ne       & ne   & ne         & ne      & ne        & ne   & ne              & ne        & ne   & ano             & FTDI          & 75.0 x 20.0 mm   \\  \hline
\multirow{2}{*}{\textbf{Intel}}            & \textbf{Galileo}                & Intel Quark X1000        & x86       & 400 MHz             & 8 MB            & 8                & 256 MB         & 14             & 6          & 6                & ano           & ano          & ano          & ano      & ne   & ano        & ano     & ano       & ne   & ano             & ne        & ne   & ano             & 2 x USB       & 123.8 x 72.0 mm  \\ \cline{2-26} 
                & \textbf{Edison}                 & Intel Atom + Intel Quark & x86       & 500 MHz             & 4 GB            & 8                & 1 GB           & 20             & 4          & 6                & ano           & ano          & ano          & ne       & ano  & ano        & ano     & ne        & ne   & ne              & ano       & ne   & ano             & dle desky     & 35.5 x 25.0 mm   \\  \hline
\multirow{2}{*}{\textbf{AMD}}            & \textbf{Gizmo 1}                & AMD GX210HA              & amd64     & 1 GHz               & ne              & -                & 1 GB           & ano            & ano        & ano              & ano           & ano          & ano          & ano      & ne   & ano        & ano     & ne        & ano  & ano             & ne        & ne   & ne              & 2x USB        & 101.6 x 101.6 mm \\ \cline{2-26} 
                & \textbf{Gizmo 2}                & AMD GX210HA              & amd64     & 1 GHz               & ne              & -                & 1 GB           & ano            & ano        & ano              & ano           & ano          & ano          & ano      & ne   & ano        & ano     & ano       & ano  & ano             & ne        & ano  & ne              & 2x USB        & 101.6 x 101.6 mm \\  \hline
\multirow{6}{*}{\textbf{Raspberry}}       & \textbf{Raspbery Pi A+  }       & Broadcom ARM 1176        & ARM       & 700 MHz             & ne              & -                & 256 MB         & 8              & 1          & ne               & ano           & ano          & ano          & ne       & ne   & ano        & ano     & ne        & ne   & ano             & ne        & ano  & ano             & 1x USB SoC    & 65.0 x 56.0 mm   \\ \cline{2-26} 
                & \textbf{Raspbery Pi B }         & Broadcom BCM 2835        & ARM       & 700 MHz             & ne              & -                & 512 MB         & 8              & 1          & ne               & ano           & ano          & ano          & ano      & ne   & ano        & SD      & ne        & ne   & ano             & ne        & ano  & ano             & 4x USB SoC    & 85.6 x 53.98 mm  \\ \cline{2-26}
                & \textbf{Raspbery Pi B+}         & Broadcom BCM 2835        & ARM       & 700 MHz             & ne              & -                & 512 MB         & 13             & 1          & ne               & ano           & ano          & ano          & ano      & ne   & ano        & ano     & ne        & ne   & ano             & ne        & ano  & ano             & 4x USB SoC    & 85 x 56 x 17 mm  \\ \cline{2-26}
                & \textbf{Raspbery Pi 2 }         & Broadcom BCM 2836        & ARM       & 900 MHz             & ne              & -                & 1 GB           & 13             & 1          & ne               & ano           & ano          & ano          & ano      & ne   & ano        & ano     & ne        & ne   & ano             & ne        & ano  & ano             & 4x USB SoC    & 86 x 56 x 17 mm  \\ \cline{2-26}
                & \textbf{Raspbery Pi 3 }         & Broadcom BCM 2837        & ARM       & 1.2 Ghz             & ne              & -                & 1 GB           & 13             & 1          & ne               & ano           & ano          & ano          & ano      & ano  & ano        & ano     & ne        & ne   & ano             & ne        & ano  & ano             & 4x USB SoC    & 87 x 56 x 17 mm  \\ \cline{2-26}
                & \textbf{Raspbery Pi Zero }      & Broadcom BCM 2835        & ARM       & 1 GHz               & ne              & -                & 512 MB         & 13             & 1          & ne               & ano           & ano          & ano          & ne       & ano  & ano        & ano     & ne        & ne   & ano             & ano       & ano  & ano             & 1x USB SoC    & 65  x 30  x 5 mm \\  \hline
\multirow{7}{*}{\textbf{Raspberry klony}}  & \textbf{Banana Pi (M3) }        & Allwinner A83T           & ARM       & 1.8 GHz             & 8 GB            & -                & 2 GB           & 13             & 1          & ne               & ano           & ano          & ano          & ano      & ano  & ano        & ano     & ne        & ano  & ano             & ano       & ano  & ano             & 2x USB SoC    & 92.0 x 60.0 mm   \\ \cline{2-26}
                & \textbf{OrangePi (+2) }         & Allwinner H3             & ARM       & 1.6 GHz             & 16 GB           & -                & 2 GB           & 13             & 1          & ne               & ano           & ano          & ano          & ano      & ano  & ano        & ano     & ne        & ano  & ano             & ano       & ano  & ano             & 4x USB SoC    & 108 × 67.0 mm    \\ \cline{2-26}
                & \textbf{CubieBoard (v5) }       & Allwinner H8             & ARM       & 2 GHz               & 8 GB            & -                & 2 GB           & ano            & ano        & ne               & ano           & ano          & ano          & ano      & ano  & ano        & ano     & ano       & ano  & ano             & ano       & ano  & ano             & 3x USB SoC    & 110 x 80 mm      \\ \cline{2-26}
                & \textbf{UpBoard (v1)}           & Intel X5-Z8350           & ARM       & 1.92 GHz            & 64 GB           & -                & 4 GB           & ano            & ano        & ne               & ano           & ano          & ano          & ano      & ne   & ano        & ne      & ne        & ne   & ano             & ne        & ano  & ne              & 4x USB SoC    & 85.6  × 56.5 mm  \\ \cline{2-26}
                & \textbf{PINE64 (+2)}            & Cortex A53               & ARM       & 1.2 GHz             & ne              & -                & 2 GB           & ano            & ano        & ano              & ano           & ano          & ano          & ano      & ano  & ano        & ano     & ne        & ne   & ano             & ano       & ano  & ano             & 2x USB SoC    & 127 x 79 mm      \\ \cline{2-26}
                & \textbf{HardKernel Odroid (C2)} & Amlogic S905             & ARM       & 1.5 GHz             & ne              & -                & 2 GB           & ano            & ano        & ne               & ano           & ano          & ano          & ano      & ne   & ano        & ano     & ano       & ne   & ano             & ne        & ano  & ano             & 4x USB SoC    & 85.0 x 56.0 mm   \\ \cline{2-26}
                & \textbf{BeagleBoard (Black)}    & Sitara AM3358/9          & ARM       & 1 GHz               & 2 GB            & -                & 512 MB         & ano            & 8          & ano              & 4             & ano          & ano          & ano      & ano  & ano        & ano     & ne        & ne   & ano             & ano       & ano  & ano             & 2x USB SoC    & 86.4 x 53.3 mm   \\  \hline  \hline
\end{tabular}}

\end{landscape}

%%%%%%%%%%%%%%%%%%%%%%%%%%%%%%%%%%%%%%%%%%%%%%%%%%%%%%%%%%%%%%%%%%%%%%%%%%%%%%%%%%%%%%%%%%%%%%%%%%%%%%%%%%%%%%%
\begin{landscape}
\chapter{Ukázka zachycených dat}
\label{PrilohaVystup}
	\begin{lstlisting}[style=MyCodeC]
19/04/2017 20:47:48  Running in: Real sniffer mode.
19/04/2017 20:47:48  Device is on AMA0: True
19/04/2017 20:47:49  Device is waked up: OK
19/04/2017 20:47:50  Device is set as Sniffer T: OK
19/04/2017 20:47:51  Default AES key set: OK
19/04/2017 20:47:51  Sniffing now:
19/04/2017 20:48:14  AccNo: 181  Device: BON.06.00000121.01  RSSI: -45dB  AES: True   Volume: 31567l 		
19/04/2017 20:48:46  AccNo: 182  Device: BON.07.00000121.01  RSSI: -46dB  AES: True   Volume: 28678l 
19/04/2017 20:48:47  AccNo: 203  Device: WEP.1b.00000010.02  RSSI: -39dB  AES: True   Temp.: 20.4C  Hum.: 35.1%
19/04/2017 20:49:12  AccNo:  71  Device: ZPA.02.01754247.01  RSSI: -74dB  AES: False  Tar.1: 3.0kWh  Tar.2: 11.0kWh
19/04/2017 20:49:19  AccNo: 182  Device: BON.06.00000121.01  RSSI: -48dB  AES: True   Volume: 31567l 
19/04/2017 20:49:48  AccNo: 204  Device: WEP.1b.00000010.02  RSSI: -44dB  AES: True   Temp.: 20.4C  Hum.: 35.0%		
19/04/2017 20:49:49  AccNo: 183  Device: BON.07.00000121.01  RSSI: -47dB  AES: True   Volume: 28678l
19/04/2017 20:50:12  AccNo:  71  Device: ZPA.02.01754247.01  RSSI: -75dB  AES: False  Tar.1: 3.0kWh  Tar.2: 11.0kWh
19/04/2017 20:50:18  AccNo: 183  Device: BON.06.00000121.01  RSSI: -49dB  AES: True   Volume: 31567l
19/04/2017 20:50:48  AccNo: 205  Device: WEP.1b.00000010.02  RSSI: -42dB  AES: True   Temp.: 20.4C  Hum.: 35.2%
19/04/2017 20:50:50  AccNo: 184  Device: BON.07.00000121.01  RSSI: -48dB  AES: True   Volume: 28678l
19/04/2017 20:51:12  AccNo:  71  Device: ZPA.02.01754247.01  RSSI: -76dB  AES: False  Tar.1: 3.0kWh  Tar.2: 11.0kWh
19/04/2017 20:51:22  AccNo: 184  Device: BON.06.00000121.01  RSSI: -46dB  AES: True   Volume: 31567l
19/04/2017 20:51:48  AccNo: 206  Device: WEP.1b.00000010.02  RSSI: -38dB  AES: True   Temp.: 20.4C  Hum.: 35.0%
19/04/2017 20:51:52  AccNo: 185  Device: BON.07.00000121.01  RSSI: -48dB  AES: True   Volume: 28678l
19/04/2017 20:52:12  AccNo:  71  Device: ZPA.02.01754247.01  RSSI: -72dB  AES: False  Tar.1: 3.0kWh  Tar.2: 11.0kWh
19/04/2017 20:52:24  AccNo: 185  Device: BON.06.00000121.01  RSSI: -50dB  AES: True   Volume: 31567l 
19/04/2017 20:52:49  AccNo: 207  Device: WEP.1b.00000010.02  RSSI: -40dB  AES: True   Temp.: 20.4C  Hum.: 34.9%
19/04/2017 20:52:52  AccNo: 186  Device: BON.07.00000121.01  RSSI: -51dB  AES: True   Volume: 28678l
19/04/2017 20:53:12  AccNo:  71  Device: ZPA.02.01754247.01  RSSI: -78dB  AES: False  Tar.1: 3.0kWh  Tar.2: 11.0kWh
	\end{lstlisting}
\end{landscape}
	
%%%%%%%%%%%%%%%%%%%%%%%%%%%%%%%%%%%%%%%%%%%%%%%%%%%%%%%%%%%%%%%%%%%%%%%%%%%%%%%%%%%%%%%%%%%%%%%%%%%%%%%%%%%%%%%
\begin{landscape}
\chapter{Ukázka vizualizace dat}
\label{PrilohaGrafy}
	 \begin{figure}[!ht]
  \begin{center}
    \includegraphics[scale=0.7]{graphs/VygenerujGrafZpa}
  \end{center}
	\vspace{-20pt}
  \caption{Vizualizace měření elektroměrem ZPA (interval 24 hodin)}
	\label{GrafPriloha3}
\end{figure}
	\begin{figure}[!ht]
  \begin{center}
    \includegraphics[scale=0.7]{graphs/VygenerujGrafBonega}
  \end{center}
	\vspace{-20pt}
  \caption{Vizualizace měření vodoměry Bonega (interval 24 hodin)}
	\label{GrafPriloha2}
\end{figure}
	 \begin{figure}[!ht]
  \begin{center}
    \includegraphics[scale=0.7]{graphs/VygenerujGrafWeptech}
  \end{center}
	\vspace{-20pt}
  \caption{Vizualizace měření čidlem Weptech (interval 24 hodin)}
	\label{GrafPriloha1}
\end{figure}
\end{landscape}

%%%%%%%%%%%%%%%%%%%%%%%%%%%%%%%%%%%%%%%%%%%%%%%%%%%%%%%%%%%%%%%%%%%%%%%%%%%%%%%%%%%%%%%%%%%%%%%%%%%%%%%%%%%%%%%
\chapter{Obsah přiloženého CD}
\label{PrilohaMedium}
K diplomové práci je přiloženo CD, obsahující bitový obraz MicroSD karty se systémem Raspbian, ve kterém je nainstalováno a nastaveno vše potřebné ke spuštění vzorové aplikace a zahájení komunikace s vyčítanými Wireless M-Bus zařízeními. Taktéž jsou zde uloženy zdrojové kódy vyčítačí i vizualizační aplikace.

\vspace{10pt}
Médium obsahuje následující strukturu: 
\vspace{-20pt}
	 \begin{figure}[!h]
  \begin{center}
    \includegraphics[scale=0.8]{obrazky/priloha_medium}
  \end{center}
	\vspace{-30pt}
\end{figure}

Návod pro spuštění aplikace:
		\begin{enumerate}
			\item Pomocí aplikace Win32DiskImager z \texttt{$\backslash$ObrazKarty$\backslash$Win32diskimager.exe} zapište obraz \texttt{$\backslash$ImageKarty$\backslash$UniPiRaspiWmBus.img} na pamětovou kartu typu MicroSD minimální velikosti 4GB.
			\item Kartu zasuňte do jednotky UniPi Neuron a zapněte tuto jednotku. Po startu jednotky dojde k aktivaci aplikace a zachytávání WM-Bus komunikace.
			\item Jednotka očekává přidělení IP adresy z DHCP serveru. Po přidělení IP adresy lze provádět vizualizaci zachytávaných dat pomocí aplikace na adrese \texttt{http:$\backslash$$\backslash$ip-adresa-jednotky$\backslash$}
			\item Případně po ssh (port 22022) přihlášení [\texttt{unipiraspiwmbus$\backslash$wmbusunipiraspi}] a zadání příkazu \texttt{screen -r} lze sledovat přímo výstup aplikace v konzoli.
		\end{enumerate}

	\vspace{10pt}
Vyčítací aplikace může být spuštěna samostatně, bez přítomnosti RaspberryPi, rozšiřující desky UniPi či IQRF komunikačního modulu. Je implementován demostrační režim s předpřipravenou sadou dříve zachycených telegramů:
\begin{itemize}
	\item Režim příjmu zašifrovaných telegramů modulem IQRF: 
		\newline 
		\texttt{python MainApplication.py aes\_iqrf}
	\item Režim příjmu zašifrovaných obecných telegramů: 
		\newline 
		\texttt{python MainApplication.py aes\_clean}
	\item Režim příjmu nešifrovaných telegramů: 
		\newline 
		\texttt{python MainApplication.py clean}
\end{itemize}




%% Konec dokumentu
\end{document}
